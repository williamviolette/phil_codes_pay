\documentclass[12pt]{article}

% TEMPLATE DEFAULT PACKAGES
\usepackage{amssymb,amsmath,amsfonts,eurosym,geometry,ulem,graphicx,color,setspace,sectsty,comment,natbib,pdflscape,array,adjustbox,threeparttable}

% ADDED PACKAGES FOR THIS MANUSCRIPT
\usepackage{palatino,newtxmath,multirow,titlesec,threeparttable,tabu,booktabs,titlesec,threeparttable,mathtools,bm,bbm,subcaption,pdflscape,tcolorbox,mathrsfs,float}
% endfloat,

%\usepackage{kbordermatrix}% http://www.hss.caltech.edu/~kcb/TeX/kbordermatrix.sty
%\usepackage{amsmath}% http://ctan.org/pkg/amsmath

\usepackage[colorinlistoftodos]{todonotes}

\usepackage{afterpage}
\usepackage[hyphens]{url}
\usepackage[margin=1cm]{caption}

\usepackage[draft]{hyperref}
\newcommand{\tim}{$\,\times\,$}
% FIGURES & TABLES CAPTION STYLING
\captionsetup[figure]{labelfont={bf},name={Figure},labelsep=period}
\captionsetup[table]{labelfont={bf},name={Table},labelsep=period}

% SECTION TITLE SETTINGS
\titlelabel{\thetitle.\enskip}
\titleformat*{\section}{\large\bfseries}
\titleformat*{\subsection}{\normalsize\bfseries}

% COLUMN TYPES
\newcolumntype{L}[1]{>{\raggedright\let\newline\\\arraybackslash\hspace{0pt}}m{#1}}
\newcolumntype{C}{>{\centering\arraybackslash}p{5.2em}}
\newcolumntype{D}{>{\centering\arraybackslash}p{5em}}
\newcolumntype{R}[1]{>{\raggedleft\let\newline\\\arraybackslash\hspace{0pt}}m{#1}}


% MARGINS AND SPACING
\normalem
\geometry{left=1.1in,right=1.1in,top=1.0in,bottom=1.0in}
\setlength{\parskip}{2.5pt}

% SPECIAL CELL 
\newcommand{\specialcell}[2][c]{%
	\begin{tabular}[#1]{@{}l@{}}#2\end{tabular}}

% NO INDENT ON FOOTNOTES
\usepackage[hang,flushmargin]{footmisc}

\begin{document}

\begin{titlepage} 
%\title{{Borrowing with Unpaid Bills}\thanks{}}
\title{{Microcredit from Delayed Bill Payments}}
\author{\\[3em]
  William Violette\thanks{Federal Trade Commission, Washington, DC. E-mail: william.j.violette@gmail.com   Any opinions and conclusions expressed herein are those of the author and do not necessarily represent the views of the Federal Trade Commission or its Commissioners.} \\
 \\ 
  }
\vspace{30mm}
\date{\vspace{5mm}This Version: \today}
\maketitle
\begin{abstract}


Delaying bill payments to public utilities may provide an important source of credit to households.  With billing data from a water utility in Manila, Philippines, this paper builds a consumption and savings model to estimate demand for both water and credit.  Estimates suggest that households value billing flexibility\todo[color=green]{What kind of flexibility, temporal?} \todo[color=red]{A does this mean being able to not pay? under what terms?} from the water utility as much as \input{tables/U_nl_per}\unskip\% of an average water bill.  I then analyze the welfare effects of popular proposals to reduce delinquency by requiring upfront payments.  Simulations find that upfront payments do not produce enough cost savings \todo[color=red]{A to whom?} to justify their negative impacts on household consumption smoothing.\todo[color=green]{``Recoup less revenue then it would take to compensate for their loss of consumption smoothing''}


\vspace{2in}
\textbf{Keywords:} credit constraints; consumption smoothing; water utilities. \\
\textbf{JEL Codes:} O13; E21; L95. \\
\bigskip
\end{abstract}
\setcounter{page}{0}
\thispagestyle{empty}
\end{titlepage}
\pagebreak \newpage

%\spacing{1.5}
\onehalfspacing

\section{Introduction}

Households face a dynamic problem of how to smooth their consumption over time, especially in low-income settings where households have little access to formal credit and risky income streams (\cite{morduch1995income}).  Households often resort to costly money lenders or informal arrangements with family members (\cite{banerjee2007economic}).  Firms may provide an additional channel for consumption smoothing by allowing households to delay their payments for goods and services.  In particular, public utilities often tolerate high levels of delinquent bills each month.  Delinquent bills provide an economically important source of credit for households since public utilities cover broad populations and take up at least \todo[color=red]{A approximately?} 5\% of household incomes in developing countries.\footnote{\cite{komives2006distributional} find that households spend 1-2\% of their incomes on water and 4\% on electricity.} 

When market failures limit formal lending, public utilities may provide efficient, second-best sources of credit.\todo[color=red]{A second best to whom? maybe add a footnote?}  First, poor access to collateral often restricts the availability of \todo[color=red]{A formal?} lending options to low-income households (\cite{jack2016borrowing}).  Public utilities overcome this barrier by threatening disconnection from future service to enforce repayment.  Second, poor infrastructure and mobile populations increase the transaction costs of screening and monitoring debtors, which contribute to high interest rates (\cite{jack2014risk}).  As part of their business practices, public utilities invest in meter reading and billing systems that may lower these transaction costs.  Third, moneylenders and peer-to-peer lending groups often cater to small pools of debtors while public utilities are able to spread default risk across a wide customer base.  Despite these advantages, linking credit to consumption from public utilities creates an inherent inefficiency by incentivizing overconsumption.  For example, households may substitute toward piped water and away from other beverages in months where they are credit constrained.\todo[color=red]{A absent externalities from water consumption? mention in footnote}\todo[color=green]{This seems so minimal, do you know if hh sell water when credit constrained or other direct evidence?  I need a more compelling example of overconsumption. A: wash clothes at home or outside the home}  Since governments actively regulate public utilities, the extent to which public utilities \todo[color=red]{A maybe whether or not is a better way to frame it}should provide credit remains an open policy question (\cite{laffont2005regulation}).  \todo[color=red]{A both of water/electricity and other stuff through credit? i'm just saying that even without the credit story, forcing you to pay would make you lose from not smoothing your water consumption. and then with credit, you also lose from not smoothing other stuff, right?}

Technologies that eliminate delayed payments have recently dominated utility policy discussions.    Both researchers and policymakers recommend new prepaid metering technologies, which require upfront payments before releasing any services, as effective strategies to prevent nonpayment.\footnote{\cite{kojima2016making} and \cite{heymans2014limits} point to prepaid metering as a successful strategy to reduce nonpayment for electricity and water utilities (respectively).  \cite{jack2016charging} find that prepaid meters for electricity in South Africa substantially increase utility revenues.}    Yet, existing analyses have yet to consider how these technologies may impact household welfare by reducing consumption smoothing.

This paper incorporates delayed payments to public utilities into household consumption and savings decisions in order to evaluate the welfare effects of offering credit through public utilities.  I build a dynamic model where each period, households have the option of borrowing with a costly asset as well as with delayed payments to a public utility.  Low interest rates on delayed payments induce households to increase their borrowing by overconsuming utility services.\todo[color=red]{A so there's 2 things going on here, right? 
1. i get credit by not paying my bill on my fixed consumption c
2. more credit by not paying and consuming c'>c cause it's cheaper than buying other beverages }  The extent to which households value payment flexibility depends on the borrowing costs of other assets, which are often difficult to measure in traditional survey data. \todo[color=red]{A  i'm still having trouble here. is this just the option of not paying? but what's the risk? what are the terms of this "loan"? how long can i do it? etc. 
i think i get what you're saying but i'm hung up on how this may be different in many scenarios. for example, in mex it's illegal to cut off people's water supply. so if they don't pay, i think the most you can do is decrease their flow or something}

\todo[color=red]{A formal or utilities?
oh ok. borrowing costs from i guess formal lenders and also roscas and all those informal ones}To estimate borrowing costs, I take a structural approach using billing data from a regulated water utility in Manila, Philippines.  The identification strategy leverages months where utility workers visit households and disconnect them if they do not pay their outstanding balances.  If borrowing were costless, households would never be disconnected during these visits because they would pay their outstanding balances with cheap loans.  Therefore, long and frequent disconnections in response to these visits --- especially among households with large outstanding balances --- provide evidence of borrowing costs.\todo[color=green]{Or they have a cheap alternative source of water?} \todo[color=red]{A any alternative explanations?
for example, what if this is just substitution? it's not that borrowing money is expensive, it's just that i won't pay and get water from my neighbors or a vendor. }  Using this strategy, I estimate high interest rates for borrowing, consistent with poor observed access to formal credit in Manila with only 3.9\% of households having credit cards and 18.7\% having bank accounts.\footnote{Statistics are from the Philippines 2014 Consumer Finance Survey.}  \todo[color=green]{Compare to actual interest rates here}

These estimates also allow me to simulate several counterfactual policies.  First, I find that in a counterfactual where households are required to pay their bills on time (holding all else equal), consumer welfare drops by \input{tables/U_nl_abs}PhP (or \input{tables/U_nl_abs_usd}USD) per household-month, which is equal to \input{tables/U_nl_per}\unskip\% of an average water bill.\todo[color=orange]{D How much of household consumption?}  Usage also declines since households no longer have an incentive to overconsume water to fund their borrowing from delaying bill payments.  

Second, I simulate a policy that (1) eliminates delayed payments by charging a high interest rate on unpaid bills, (2) eliminates default risk by preventing households from leaving large outstanding balances when they permanently disconnect, and (3) adjusts prices to ensure that the water utility remains revenue-neutral (consistent with regulation in Manila).\footnote{\cite{laffont2005regulation} discusses economic reasons for these regulatory structures which are common in both developed and developing country settings.}  Without default risk, the utility has fewer costs to recover with prices. Yet without billing flexibility, households use less water, generating less revenue for the utility.  On net, prices remain the same and the policy reduces social welfare by \input{tables/U_ppe_abs}PhP (or \input{tables/U_ppe_abs_usd}USD) per household-month.  

Finally, I use this framework to evaluate implementing prepaid meters in this context.  While ensuring that households pay upfront, prepaid meters require steep increases in prices to cover their installation and maintenance costs.  As a result, prepaid meters produce large reductions in social welfare on the order of \input{tables/U_pp_abs}PhP (or \input{tables/U_pp_abs_usd}USD) per household-month.  Taken together, these results suggest that popular policies to reduce nonpayment may be welfare reducing primarily by limiting consumption smoothing.\todo[color=green]{ Decompose loss due to cost and due to consumption smoothing? }  Given around 2.4 million piped water using households in Metro Manila, these policies would imply welfare costs on the order of \input{tables/U_ppe_abs_usd_per}to \input{tables/U_pp_abs_usd_per}million US dollars per year. \todo[color=orange]{D how many are affected by law? }

This paper contributes to three main strands of economic literature.  First, this paper brings payment and billing policy questions into a growing literature on optimal policy for utilities in developing countries (\cite{mcrae2015infrastructure}; \cite{szabo2015value}; \cite{jack2016charging}; \cite{jack2015pay}; \cite{szabo2015reducing}).  Building on previous static models, this paper also provides a framework for considering how dynamic incentives affect household demand for public utilities.  Second, this paper draws on a large body of research analyzing household credit constraints, particularly through microfinance interventions (\cite{morduch1999microfinance}; \cite{morduch1995income}; \cite{cull2009microfinance}; \cite{dupas2013savings}; \cite{jack2016borrowing}).  \cite{karlan2009expanding} and \cite{gine2014group} specifically focus on microfinance in the Philippines. Third, \cite{deaton1991saving} provides the foundational model of dynamic decision-making that serves as the starting point for the model in this paper, and the estimation of this model adapts methods developed by \cite{gourinchas2002consumption} and \cite{laibson2007estimating}.

This paper proceeds with Section~\ref{section:data} describing the water billing data from Manila.  Billing practices, disconnection policies, and institutional details are discussed in Section~\ref{section:descriptives} alongside descriptive evidence.  Section~\ref{section:model} develops a model of household consumption and savings decisions while Section~\ref{section:estimation} discusses the estimation and results.  Counterfactual exercises and welfare impacts are provided in Section~\ref{section:counterfactuals}.  Section~\ref{section:conclusion} concludes. \todo[color=orange]{ non-pay correlated with other non-pay? possibility of bribery?}

\section{Data}\label{section:data}

% Describe notice of disconnection in more detail here??? and  NO DO IT LATER~! PLEASE!!!!

Measuring creditab wt through delayed water bill payments requires information on water consumption and bill payments at the household-level as well as features of the credit environment in Manila.  % As well as information on the credit environment facing households

Water consumption and bill payments come from water utility records collected as part of a research partnership with one of the two regulated utilities in Metro Manila, Philippines where each utility is responsible for serving their assigned geographic half of Metro Manila.  This utility provided access to monthly billing records for each connection as well as detailed information covering the regulatory structure and costs of production.  Monthly billing records include meter readings, billing amount, outstanding balances, and payments spanning January, 2008 to May, 2015.  Over this period, the total number of connections increased from 900,000 to 1,500,000 as the water utility expanded service access.  Water connections are split into four categories: residential (90\%), semi-business (4\%), commercial (5\%), and industrial (1\%).  

To examine household demographics, connections are linked to a water connection survey conducted independently to monitor the quality of the utility's service.  The survey randomly interviewed households with water connections covering 15,000 households in 2008, 24,000 households in 2010, and 23,000 households in 2012.  Since the survey followed a similar sampling design across rounds, 13\% of households were interviewed in two waves and 1.4\% of households were interviewed in three waves while remaining households were interviewed in one wave.  To focus on household decisions, the analysis includes only residential connections that are identified in this survey as serving a single household (31\% of all connections).  The connection survey also includes demographics for households that own their water connections.  Households using connections alone tend to be larger and wealthier than households that share connections with other households according to previous research (\cite{wjv}).  Table~\ref{table:sampleconstruction} in Appendix~\ref{appendix:sampleconstruction} includes more details on how the final sample is constructed for the analysis.
% This sample restriction ensures that connection-level is equal to household-level.

DESCRIBE CBMS!

Since the analysis focuses on the borrowing and savings behavior of a representative household, additional data sources help characterize credit access in Manila.  Average household income is calculated using the  2015 Family Income and Expenditure Survey, which covers 4,130 households in Metro Manila.  Interest rates for borrowing and saving come from the 2014 Consumer Finance Survey in Metro Manila as well as the World Bank Databank (2010-2015) for the Philippines.  



% To exclude connection-level survey data, which measure the number of households using each connection as well as demographics for the household that owns the connection.  Conducted every two years between 2008 and 2012, the survey covers a cross-section of nearly 50,000 water connections.  Documentation indicates that surveyors used stratified random sampling where the number of connections surveyed was proportional to the census population in the smallest administrative census region, making this survey roughly representative of the population of piped water users.  To examine the consumption and savings behavior of individual households, the analysis is limited to surveyed connections that serve only one household (31\% of surveyed connections), which tend to serve larger, wealthier households according to previous research (\cite{wjv}).  This sample restriction also ensures that connection-level is equal to household-level.

% Measuring household income for water users requires two-steps: in the first step, billing data are merged with connection-level survey data on household demographics.  In the second step, household income is imputed with household location and demographics using additional survey data on household income.

% The first step uses connection-level survey data, which measure the number of households using each connection as well as demographics for the household that owns the connection.  Conducted every two years between 2008 and 2012, the survey covers a cross-section of nearly 50,000 water connections.  Documentation indicates that surveyors used stratified random sampling where the number of connections surveyed was proportional to the census population in the smallest administrative census region, making this survey roughly representative of the population of piped water users.  To examine the consumption and savings behavior of individual households, the analysis is limited to surveyed connections that serve only one household (31\% of surveyed connections), which tend to serve larger, wealthier households according to previous research (\cite{wjv}).  This sample restriction also ensures that connection-level is equal to household-level.




% Since multiple households often share the same connections in Manila, 



% The second step leverages the 2015 Family Income and Expenditure Survey, which covers 2,528 households across 216 wards that overlap with the connection survey.  This cross-sectional survey collects detailed information on household income, demographics, and measures of credit usage.  The analysis is further limited to data from these 216 geographic wards (18\% of wards covered by the connection-level survey).\footnote{Wards in the Philippines are known as ``Barangays,'' and have average populations of 7,600 people in Metro Manila.}  Income is imputed with a linear regression of indicators for each household size, each number of employed household members, each age of household head, each ward, and each dwelling type (duplex, single-house, and apartment).  Log-income is used for imputation both to prevent negative values for imputed incomes and to match the right-skewed distribution of household income.\footnote{The imputation procedure results in an $\text{R}^{2}$ of 0.48.}  



% Since the distribution of household income  and ensure that imputed incomes are positive.
% To impute household income, 

% Since connection survey data do not include income measures, household income is imputed from household survey data according to demographics and location.  The household survey data 


% Household income is imputed using household demographics and location from connection-level survey data and separate household survey data.  

% The billing data is merged at the water connection level to survey data that measures the number of households using each connection as well as demographics for the household that owns the connection.  Conducted every two years between 2008 and 2012, the survey data cover a total of nearly 50,000 water connections.  Connection survey documentation indicates that surveyors used stratified random sampling where the number of connections surveyed was proportional to the census population in the smallest administrative census region, making this survey roughly representative of the population of piped water users.  

% To analyze consumption and savings decisions at the household level,  
% The analysis focuses on 

% the analysis is further limited to connections that each serve a single household in order to ensure that billing data for each connection maps directly into the consumption and savings decisions of a single household.\todo[color=green]{Its not clear from identification strategy that I need to drop them, too much detail, just do generalizability sentence at the end.}  

% Previous research from \cite{wjv} finds that sharing a connection with multiple households increases the costs of consuming water for each household, which affects each household's water demand.  \cite{wjv} also finds that households using a connection individually tend to have higher water demand, larger household sizes, and greater likelihood of living in single houses (as opposed to duplexes or apartments) in comparison to households sharing water connections.  These differences may limit the generalizability of the findings outside of this subpopulation in Manila.  

% This sample restriction also ensures that household-level data and connection-level data\todo[color=green]{Do you know if its the same HH over time? (no, but make that clear!)} can be referred to interchangeably.  Table~\ref{table:sampleconstruction} in Appendix~\ref{appendix:sampleconstruction} includes more details on how the sample is constructed for the analysis. 

% Because the connection survey data does not include income measures, average monthly income for households in Manila is taken from the 2015 Family Income and Expenditure Survey.  

% % Mention later: average interest rate, and other calibrations 
% % The average interest rate for the Philippines is calibrated from the World Bank Databank (2010-2015).


\section{Credit through Delayed Water Bill Payments}\label{section:descriptives}

% \footnote{The company upgraded t in the first year of the sample.  Previously, the company mailed the bill to each household at the end of the month.} 
% Each month, households consume \input{tables/usage_all}m3 of piped water on average, which totals an average monthly bill of \input{tables/bill_all}PhP as shown in Table~\ref{table:descriptives_all}.  Given an average monthly income in Manila of \input{tables/y_avg}PhP and savings rate of \input{tables/save_avg}PhP,  monthly water bills reach a modest average of \input{tables/bill_inc_all}\unskip\% of income and \input{tables/bill_sav_all}\unskip\% of savings.  

%%% NOTES FROM TOWNSEND PAPER ! %%%
% For the purposes of this partial equilibrium analysis, this borrowing constraint is exogenous.
% It is not a natural borrowing constraint as in Aiyagari (1994) and therefore somewhat ad
% hoc, but such a constraint can arise endogenously in models with limited commitment (see
% Wright (2002)) or where lenders have rights to garnish a fraction of future wages (e.g.,
% Lochner and Monge-Naranjo (2008))

In Manila, Households have poor access to credit: 1\% have credit cards, 3.8\% have auto-loans, 0.5\% have mortgages, and \input{tables/wloan}\unskip\% have all-purpose loans (of which \input{tables/money_lender}\unskip\% are from money lenders and \input{tables/micro_finance}\unskip\% are from micro-finance groups).\footnote{Author's calculations from the 2014 Consumer Finance Survey.}  All-purpose loans charge high monthly interest rates of \input{tables/irate}\unskip\% on average.  Households also have 

Delaying water bill payments may provide a reliable source of low-cost credit to households because the water utility often tolerates high rates of delinquency before disconnecting water service and the utility is prohibited from charging any interest on outstanding balances.  Consistent with this theory, households are \input{tables/days_delinquent_all}days behind in their payments on average.  Households also make large, infrequent payments.  Table~\ref{table:descriptives_all} provides summary statistics on household consumption, billing, and payments.  While the average bill is \input{tables/bill_all}PhP per month, payment sizes average \input{tables/pay_size_all}PhP and households make payments in only \input{tables/pay_freq}\unskip\% of months.  These payment patterns leave an average total outstanding balance of \input{tables/balance_all}PhP per month.  With average monthly incomes of \input{tables/y_avg}PhP, households spend around 2\% of their income on water while unpaid water bills total around 3.5\% of their income.\footnote{Author's calculations with the 2015 Family Income and Expenditure Survey.}  In comparison, all-purpose loans average \input{tables/meanloansize}PhP with an average period of \input{tables/meanloanlength}months.\footnote{For reference, 1 USD is worth around 45 PhP over this period.}  

  

The utility enforces payment by visiting households that are at least one month delinquent and disconnecting their water service if these households do not immediately pay their bills.\footnote{Disconnection typically involves placing a metal lock on the water meter that stops any flow.}  Likely due to time and travel costs, these collection attempts are relatively rare, occurring in only \input{tables/prob_caught}\unskip\% of months where households are at least a month delinquent.\footnote{This rate ranges from \input{tables/prob_caught_min}\unskip\% to \input{tables/prob_caught_max}\unskip\% across the 12 business areas and is positively correlated with outstanding balances.  \input{tables/tcd_max}\unskip\% of households receive a collection attempt at some point during the sample.}  \input{tables/tcd_share_pay}\unskip\% of collection attempts result in immediate payment while 28\% result in disconnection within two months.  While households often try to negotiate for additional time to pay their bills, \input{tables/days_pay_under_30}\unskip\% of households report having to pay within 30 days and the average grace period is \input{tables/days_pay_average}days.  As a result, only \input{tables/enough_time}\unskip\% of households report having ``enough time'' to make their payments, which may be in part due to poor credit access in Manila.  While disconnected, households likely substitute to alternative water sources including sharing with neighbors, using from deepwells, or purchasing from local water vendors.  

% (\input{tables/tcd_share_rec}\unskip\% of total attempts)
% The remaining \input{tables/tcd_share_dc}\unskip\% result
Among collection attempts that lead to disconnection, 64\% result in households reconnecting by paying their outstanding bills as well as a 200 PhP fee.  These households choose to reconnect an average of \input{tables/t_rec}months after disconnection.  The remaining 36\% result in households remaining permanently disconnected throughout the sample period. These households may have already planned to change dwellings, switch to a different water source, or leave Manila at the time that they receive a collection attempt.  Households that permanently disconnect often leave large outstanding balances that are never repaid averaging \input{tables/out_bal}PhP, which is over 10 times the average water bill.  Since 0.13\% of households permanently disconnect each month on average,\footnote{The disconnection rate is calculated (1) for households that connect between 2007 and 2011, which covers the connection survey and (2) for disconnections occurring between Jan, 2012 and May, 2014 to ensure that any disconnections last for at least a year before the sample ends in June, 2015.} the average household remains connected for around 32 years assuming a constant rate over time.\footnote{This constant rate is consistent with a weak observed correlation between the permanent disconnection rate and calendar months of -0.0033.}  Therefore, unpaid outstanding balances average 18.5 PhP per household-month or 2.8\% of the average water bill.


\begin{table}[h!] % PUT IN DEMOGRAPHICS PLEASE ?!?!?
\centering
\caption{Mean Characteristics}\label{table:descriptives_all}
\vspace{-2mm}
\begin{threeparttable}
\begin{tabular}{l*{1}{cccccc}}
\toprule
 & Mean & Std. Dev. & Min &  25th & 75th & Max \\
\midrule
 Usage (m3)  & 24.3  & 15.4  & 0.0  & 14.0  & 31.0  & 200.0  \\ 
 Bill  & 671  & 734  & -4,640  & 265  & 843  & 19,999  \\ 
 Unpaid Balance  & 1,206  & 3,316  & -9,965  & 0  & 1,044  & 79,904  \\ 
 Share of Months with Payment  & 0.71  & 0.45  & 0.00  & 0.00  & 1.00  & 1.00  \\ 
 Payment Size  & 901  & 1,067  & 0  & 313  & 1,070  & 49,688  \\ 
 Days Delinquent  & 56.2  & 113.7  & 0.0  & 0.0  & 61.0  & 750.0  \\ 
 Delinquency Visits per HH  & 0.40  & 0.70  & 0.00  & 0.00  & 1.00  & 6.00  \\ 
 Share of Months Disconnected  & 0.04  & 0.18  & 0.00  & 0.00  & 0.00  & 1.00  \\ 

%  Usage (m3)  & 23.1  & 20.5  & 23.1  & 25.8  \\ 
 Bill (PhP)  & 613  & 517  & 609  & 711  \\ 
 Unpaid Balance (PhP)  & 1,028  & 906  & 1,040  & 1,140  \\ 
 Share of Months with Payment  & 0.71  & 0.70  & 0.71  & 0.72  \\ 
 Payment Size  & 816  & 697  & 809  & 938  \\ 
 Days Delinquent  & 52.9  & 54.8  & 53.2  & 50.7  \\ 
 Delinquency Visits per HH  & 0.41  & 0.40  & 0.41  & 0.41  \\ 
 Share of Months Disconnected  & 0.04  & 0.03  & 0.04  & 0.03  \\ 

% OLD \\
%  Usage (m3)  & 23.5  & 20.9  & 23.6  & 26.1  \\ 
 Bill  & 626  & 529  & 625  & 722  \\ 
 Unpaid Balance  & 1,034  & 909  & 1,074  & 1,119  \\ 
 Share of Months with Payment  & 0.71  & 0.70  & 0.72  & 0.72  \\ 
 Payment Size  & 829  & 709  & 826  & 948  \\ 
 Days Delinquent  & 33.5  & 35.5  & 34.3  & 30.6  \\ 
 Delinquency Visits per HH  & 0.38  & 0.38  & 0.39  & 0.38  \\ 
 Share of Months Disconnected  & 0.04  & 0.04  & 0.04  & 0.04  \\ 
 Monthly Income (PhP)  & 28,614  & 16,957  & 25,147  & 43,665  \\ 
 HH Size  & 5.3  & 4.7  & 5.4  & 5.9  \\ 
 Age of HoH  & 46.1  & 43.9  & 46.7  & 47.7  \\ 
 Low Skilled HoH  & 0.16  & 0.19  & 0.15  & 0.13  \\ 
 &  &  &  &  \\ 
 Total Households   & 7,917  & 2,631  & 2,646  & 2,640  \\ 
 Total Observations  & 471,609  & 156,486  & 157,617  & 157,506  \\ 

\bottomrule \\ [-.8em]
% \multicolumn{5}{l}{Total Households: \input{tables/total_hhs_stayers}  Obs. per Household: \input{tables/obs_per_hh_all} Total Obs.: 2,844,481} \\
% \multicolumn{5}{l}{\scriptsize Bill, Unpaid Balance, and Payment Size are in PhP per household-month.  45 PhP $\sim$ 1 USD} \\[-.5em]
% \multicolumn{5}{l}{\scriptsize Negative bills and balances generally arise from refunds for billing or payment errors.}  \\[-.5em]
% \multicolumn{5}{l}{\scriptsize See Appendix~\ref{appendix:sampleconstruction} for more details on the sample construction.} 
\end{tabular}
\begin{tablenotes}
\footnotesize
\item  45 PhP $\sim$ 1 USD \,\, Std. Dev. refers to standard deviation.  25th and 75th refer to percentiles.  Bills, Unpaid Balances, and Payment Sizes are in PhP per household-month.  Negative bills and balances generally arise from refunds for billing or payment errors.  See Appendix~\ref{appendix:sampleconstruction} for more details on the sample construction.
\end{tablenotes}
\end{threeparttable}
\end{table}

% , households do not pay and are disconnected within two months.  

% Among delinquency visits that occur when households are over three months overdue on their bills,  \input{tables/tcd_share_rec_3}\unskip\% result in temporary disconnections and these households wait \input{tables/t_rec_3}months on average before reconnecting.  By contrast, when households are under three months overdue, only \input{tables/tcd_share_rec_3l}\unskip\% result in temporary disconnections and these households wait \input{tables/t_rec_3l}months on average before reconnecting.  This pattern is consistent with households facing credit constraints.  When households are confronted with small debts to keep water service, they quickly pay to remain connected to service.  Yet, when households must pay large debts to keep water service, many households delay payment for many months while they save income.  This pattern suggests that households do not have access to low-cost credit, which would allow them to maintain water service.  


% Households are expected to pay the bill by the end of the month.  


Households may also pay their water bills infrequently for other reasons aside from accessing low-cost credit.  First, households may not be aware of the size or due date for each water bill.  The utility addresses this problem by sending meter readers to record monthly consumption, deliver each bill in person, and educate households about their payment options.  Second, households may experience time or hassle costs in making each payment, which may naturally lead households to pay infrequently.  The utility reduces these costs by offering many different payment options.\footnote{\input{tables/center}\unskip\% of households use small payment centers (mall kiosks, gas stations, convenience stores, etc.), \input{tables/maynilad}\unskip\% of households pay at local utility offices, and \input{tables/atm}\unskip\% of households pay over the phone, online, or via ATM kiosks according to the connection survey.}  Third, previous research has documented how negative opinions toward public utilities or local governments can increase delinquency \citep{szabo2015reducing}.  In Manila, the utility enjoys largely positive public opinion because the water utility in Manila represents a public-private partnership that has dramatically improved service since taking over for the previous government utility.\footnote{The connection survey conducts an independent assessment of people's satisfaction with the utility finding that over 95\% of households rate the utility as ``good'' or ``very good'' as opposed to ``fair,'' ``poor,'' or ``very poor'' in terms of overall service quality.}


% One reason may be households still experience time or hassle costs per payment, which lead households to pay infrequently.  Another reason may be that households treat delayed bill payments as a micro-loan 

% The borrowing rate from standard assets is estimated to be \input{tables/est_irate_per}\unskip\% per month, which implies an annual interest rate of \input{tables/est_irate_annual_per}\unskip\%.  This estimate is substantially lower than the 20\% per month interest rate that \cite{karlan2009expanding} document as being commonly charged by moneylenders in Manila.  Despite this high interest rate, \cite{karlan2009expanding} document that at least 30\% of their sample of microentrepreneurs report taking credit from moneylenders.  The estimated borrowing rate of \input{tables/est_irate_per}\unskip\% is more similar to microloans of 1,000 PhP at 2.5\% monthly interest offered to rural Filipino households as part of a microfinance experiment conducted by \cite{gine2014group}.\footnote{The annual interest rate of \input{tables/est_irate_annual_per}\unskip\% well exceeds the average 13 to 25\% range offered by microfinance providers worldwide surveyed by \cite{cull2009microfinance}.  Two possible reasons for this discrepancy are that (1) institutional reasons unique to Manila may limit lenders' abilities to offer low rates and (2) subsidies may allow many microfinance providers to offer below-market interest rates.}

% \todo[color=red]{ A what about procrastination? 
% in the us when i paid utilities via direct pay online, i would always pay on time. here in mex, you can't do that so emilio and i are always delinquent on our bills but only cause we lazy a.f. to go to the 7-eleven to pay. deal with this later?}
% \todo[color=red]{isn't this huge?}
% \todo[color=red  ]{A so there's a 60 day grace period? any chance hhs know this and pay in 60 day intervals?}   Check!
% \todo[color=green]{Do you not see visits in the larger dataset? Table 2 has visits for the full sample}

% Delinquency visits often surprise households. 
% --- low awareness, hassle costs, and negative public opinion

Unlike these other reasons for delinquency, the credit mechanism predicts that households with volatile incomes may time their bill payments to smooth their consumption over time (\cite{deaton1991saving}).  The following equation tests whether income is more correlated with bill payments than with consumption by using changes in the number of working household members as a proxy for income shocks
\begin{align*}
\Delta Y_{ht} =\beta_0\,+\, \beta_1\, \Delta \textsc{Working Members}_{ht} \, + \beta_2\, \Delta \textsc{Total Members}_{ht}   \,+\varepsilon_{ht}
\end{align*}
where $ \Delta \textsc{Working Members}_{ht} $ is the change in working household members and \\ $\Delta \textsc{Total Members}_{ht}$ is the change in total household members between each survey round, $t$ and for each household, $h$.  $\Delta Y_{bq}$ includes four outcomes measured as changes: water consumption (measured by the water bill in PhP), water bill payments (in PhP), disconnection status, as well as monthly household income, which is measured in the household survey from Pasay City.  Standard errors are clustered at the household-level.

\begin{table}[!ht]
\small
\centering
\begin{threeparttable}
\caption{Correlations of Household Working Members and Total Members with Consumption, Payments, and Income}\label{table:panelanalysis}
\vspace{-2mm}
% \resizebox{.8\linewidth}{!}{
\begin{tabular}{lccc}
\toprule
 & \small (1) & \small (2) & \small (3)  \\
 & \small $\Delta$ Consumption & \small $\Delta$ Payment & \small $\Delta$ Income \\[.5em]
 \toprule
$\Delta$ Working Members&        6.68                   &       40.44\textsuperscript{a}&     4920.00\textsuperscript{a}\\
                    &      (8.33)                   &     (15.35)                   &    (132.41)                   \\[0.5em]
$\Delta$ Total Members&       23.05\textsuperscript{a}&        1.76                   &     1480.07\textsuperscript{a}\\
                    &      (5.84)                   &     (10.43)                   &     (72.07)                   \\[0.5em]
Mean                &      565.02                   &      553.05                   &   21,663.21                   \\
$\text{R}^{2}$      &       0.010                   &       0.003                   &       0.143                   \\
N                   &       4,788                   &       4,808                   &      21,374                   \\

\bottomrule
\end{tabular}
\begin{tablenotes}
\footnotesize
\item Outcomes are in PhP per month and household members are reported in the 2008-2012 connection survey.  Standard errors are clustered by household.  Income changes exclude top and bottom 1\% outliers. \textsuperscript{c} p$<$0.10,\textsuperscript{b} p$<$0.05,\textsuperscript{a} p$<$0.01 
\end{tablenotes}
\end{threeparttable}
\end{table}

Column (1) of Table~\ref{table:panelanalysis} finds a large statistically significant correlation between total household members and consumption as well as a very small, statistically insignificant correlation between employed members and consumption.  This finding suggests both that the number of household members is a key driver of water demand and that changes in household income have little impact on water consumption.  Column (2) finds the opposite pattern for payments where employed members drive large increases in payments while total household members exhibit almost zero correlation.  Consistent with Column (2), disconnection patterns in Column (3) suggest that adding an additional employed household member reduces the probability that a household is disconnected by 2\% which is economically significant given an average disconnection rate of 13\%.  Column (4) finds that working household members correlated much more strongly with household income than the total number of household members.  Comparing results from Column (2) and (4), estimates suggest that in response to income shocks, households spend almost one out of every 100 PhP on outstanding bills.  These results provide suggestive evidence of consumption smoothing where households flexibly adjust their payment patterns in response to income shocks without changing their water consumption.  

Appendix~\ref{appendix:alternativeincome} confirms similar results by linking the age, education, and location of connection owners to average changes in earnings for people with the same demographics from quarterly Labor Force Surveys.  This exercise suggests that consumption smoothing patterns may be robust to using household employment changes as proxies for income shocks.


% a positive but statistically insignificant correlation between income and consumption.  Due to measurement error in income, the small point estimate of 0.0007 likely underestimates this correlation.  Measurement error in income is likely to be substantial both because only around 55 incomes are observed per quarter-bin in the Labor Force Survey and because bins are constructed from coarse demographic categories that may only loosely reflect household incomes.  Despite this potential measurement error, Column (2) finds a statistically significant correlation between bill payments and income that is six times larger than the correlation between income and consumption.  


\section{A Model of Borrowing, Saving, and Water Use}\label{section:model}

% The real interest rate $r_a$ is takes a value of $r_l$ when households are saving ($A_{t+1} \leq 0$) and a value of $r_h$ when households are borrowing ($A_{t+1} > 0$).  This wedge in interest rates is intended to capture an institutional context with underdeveloped financial markets where households face high borrowing costs from moneylenders as well as high transaction costs in loaning out their own savings.  This assumption is consistent with a recent literature in development economics that emphasizes how households often face savings and borrowing constraints.\footnote{In Kenya, \cite{dupas2013savings} and \cite{dupas2013don} estimate large willingness-to-pay for improved savings technologies.  In Manila, \cite{karlan2009expanding} find high  interest rates for loans from moneylenders.}  Households are assumed to be unconstrained in their amount of borrowing or saving with this asset.  % possibly cite (\cite{karlan2009expanding}) here?

% At the beginning of each period, households can choose whether to be temporarily disconnected ($D_{t+1}=1$) or connected ($D_{t+1}=0$) to water service.  If households disconnect (or remain disconnected), they pay a fixed cost $f$.  Since temporarily disconnected households are likely to share with neighbors or purchase from local vendors who resell piped water, they are assumed to face the same price per unit, $p(w_t)$, as connected households.  Temporary disconnection allows households to delay paying their outstanding water balances, especially after receiving a delinquency visit, as described in more detail below.  % .\footnote{For example, Manila has a steeply increasing price schedule with monthly usage, which is described in more detail in Appendix~\ref{appendix:tariff}.} 


%\subsection{Setting up the Household's Problem}
%Time horizon etc.  Ignore disconnecting households.. 

To quantify the implications of different billing policies for household welfare, I construct a simple model of household borrowing, saving, and water consumption decisions over time.

In this model, households first move to Manila and begin consuming piped water.  As households receive positive income shocks, they accumulate precautionary savings to insure against future negative income shocks.  In these months, households simply consume water according to their preferences and pay their bills.

When households receive enough negative income shocks, they spend their savings and begin to borrow.  First, households borrow by not paying their current water bills because they face zero interest rates on unpaid bills.  Each successive month, households can increase this debt by continuing to consume water without paying.

Further negative income shocks may induce households to borrow beyond their current water bill at which point they face a trade-off: borrowing from a standard asset at a high interest rate, or further increasing their water consumption to expand the size of their unpaid bills.  For example, instead of taking out a costly loan from a moneylender, households may choose to wash clothes at home instead of at a laundromat or drink water instead of other beverages while leaving their bills unpaid.  As substitution toward piped water becomes increasingly costly, households borrow from the standard asset.

With some probability, the utility conducts collection attempts, which confront indebted households with a choice: pay their debts or disconnect and use an alternative water source.  Since the alternative water source may involve an additional fixed cost each month, households with small debts simply prefer to pay their debts and stay connected either by taking out a costly loan or by lowering their consumption this month.  Households with large debts decide to disconnect in order to avoid taking out a large loan or experiencing a sharp drop in consumption this month.  Once disconnected, households begin accumulating savings to eventually pay their debts and reconnect to service.

Just before changing dwellings, households have an incentive to accumulate large unpaid bills since the utility is unable to collect on these bills if households leave Manila.  In these months, households may also have an incentive to disconnect in response to collection attempts.

To formally model household borrowing and savings decisions, I start from a standard intertemporal utility maximization problem as developed by \cite{deaton1991saving}.  This approach assumes an infinite time horizon where households choose water usage, borrowing, and savings in each month to maximize current and expected future utility.  

Households first move to Manila in month 1 and then choose their consumption of water and other goods as well as how much to borrow and save each month until they leave the city in month $T$.  Household expected utility in each month $t$ is given by the following equation
\begin{align}\label{eq:u}
\begin{split}
E_t \Big[ \sum_{\tau = t}^{T} (1+\delta)^{t-\tau} \,\frac{u(w_{\tau},x_{\tau})^{1-\gamma}}{1-\gamma}   \Big]\\
u(w_{\tau},x_{\tau}) =  x_{\tau} -  \frac{1}{2} (w_{\tau} - \alpha)^2
\end{split} 
\end{align}
where households have a rate of time preference $\delta \in (0,1)$.  Households have quadratic preferences over water consumption, $w_{t}$, and a bundle of other goods, $x_{t}$.  $\alpha$ captures household's water satiation point each month, capturing the intuition that households are likely to consume a finite amount of water even if the price is zero.  Quadratic preferences also assume that income has no effect on consumption consistent with descriptive evidence that fluctuations in household employment are uncorrelated with water spending.  Households are assumed to have constant relative risk aversion for consumption across months given by $\gamma$.

% \footnote{\cite{wjv} finds an average price elasticity of 0.84 in this setting while \cite{szabo2015value} finds an average price elasticity of 0.98 in South Africa.  Other studies find elasticities ranging from 0.01 to 0.81 (\cite{diakite2009proposal}, \cite{strand2005water}).} 

In each month, households face a budget constraint as follows
\begin{align}\label{eq:bc}
\begin{split}
x_t \, + \, p w_t \, &= \, y_t \, - (1-C_{t+1})f  \, + \,  A_t  - \dfrac{A_{t+1}}{1+r}   \, + \,  B_t - B_{t+1} 
\end{split}
\end{align}
where $p$ captures the price of water, which is assumed to depend linearly on water consumption to approximate the tariff in Manila, $p=p_1+p_2w_t$.   The price of the bundle of other goods, $x_t$, is normalized to one.  $y_t$ represents household income each month which takes a value of $(1+\theta)\bar{y}$ with probability $\pi$ and a value of $(1-\theta)\bar{y}$ with probability $(1-\pi)$ where $\pi \in (0,1)$ and $\theta  \in (0,1)$.  

Households choose whether to remain connected in the next month given by $C_{t+1}$.  When households are disconnected ($C_{t+1}=0$), they are assumed to face fixed cost $f$ each month.  Since temporarily disconnected households are likely to share with neighbors or purchase from local vendors who resell piped water, they are assumed to face the same marginal prices as connected households.

% talk about savings earlier!
Households can use asset, $A$, for borrowing and saving.  $A_t$ captures debt (when $A_t<0$) or savings (when $A_t>0$) inherited from the previous month.  $A_{t+1}$ captures how much debt or savings to pass onto the following month.  Households are assumed to earn zero interest from saving consistent with low access to interest-bearing bank accounts in Manila.\footnote{Among households in the 2014 Consumer Finance Survey, 21\% have bank accounts, 14\% have bank accounts that pay interest, and 6\% have accounts that pay interest and balances over 10,000 PhP, which is the minimum balance need to earn interest.  The Philippines National Bank lists an interest rate of 0.1\% on savings accounts above 10,000 PhP (accessed Jan. 28th, 2020).} Households face interest rate $r_h$ when borrowing so that  $r=r_h$ if $A_{t+1}<0$ (and $r = 0$ if $A_{t+1}\geq 0$).  Households are assumed to pay all debts or enjoy all savings from this asset before leaving Manila so that $A_{T}=0$.  Households are also assumed to face a borrowing constraint such that $A_{t+1}\geq\overline{A}$, which prevents households from infinitely borrowing.

Households can also borrow by delaying payments to their water bills.  $B_t$ ($\leq0$) captures unpaid bills inherited from the previous month.  $B_{t+1}$ ($\leq0$) describes how much water debt to pass onto the following month (by delaying payment of the current bill).  Households must satisfy the following borrowing constraint each month, which ensures that water debt must be less than this month's bill and unpaid bills from previous months
\begin{align}\label{eq:borrowingconstraint}
B_{t+1} &\geq (1-v_t)  C_{t} C_{t+1} (B_t -pw_t) + (1-C_{t+1})B_t 
\end{align}


The choice of $B_{t+1}$ depends on whether households receive a collection attempt, $v_t$, which occurs with probability $\lambda \in (0,1)$ and is assumed to be independent of the two income states.  The choice of $B_{t+1}$ also depends on whether households are connected to piped water, $C_t$, and choose to remain connected in the next month, $C_{t+1}$.  


The borrowing constraint creates the following four cases:

\textbf{Case 1:} If there is no collection attempt ($v_t=0$) and the household is connected ($C_{t}=1$) and chooses to stay connected ($C_{t+1}=1$), then the household can borrow up to their existing water debt plus their bill this month so that the borrowing constraint equals $B_{t+1}\geq B_t -  p w_t $.

\textbf{Case 2:} If there is a collection attempt ($v_t=1$) and the household is connected ($C_t=1$) and chooses to stay connected ($C_{t+1}=1$), then the household needs to pay off their existing debt and cannot borrow from their water bill this month so that $B_{t+1} = 0$.

\textbf{Case 3:} If the household chooses to be disconnect in the next month ($C_{t+1}=0$) likely in response to a collection attempt, then the household avoids paying existing debt but faces an additional fixed cost of $f$ per month from using from a temporary water source.  In this case, the borrowing constraint reduces to $ B_{t+1} \geq B_t$ ensuring that households can only borrow up to their existing water debt.  

\textbf{Case 4:} If the household is disconnected ($C_t=0$) and chooses to reconnect ($C_{t+1}=1$), then the household needs to pay off their existing debt and cannot borrow from their water bill this month so that $B_{t+1} = 0$.  
% given their optimal choices of water, $w_t$, other goods, $x_t$, standard assets, $A_{t+1}$, water borrowing, $B_{t+1}$, and connection decision, $C_{t+1}$

Households also face an additional borrowing constraint, $\overline{B} \leq B_{t+1}$, which prevents households from borrowing indefinitely.  In the last month $T$, households learn that they will move to either a new home in Manila (with probability $\chi$) in which case the need to pay their outstanding bills ($B_T=0$) or leave Manila (with probability $(1-\chi)$) in which case they can leave bills unpaid $\overline{B}\leq B_T \leq0$

The household's maximization problem can then be written recursively where households solve for a value function, $V_t(.)$, over four possible states given by $s_{t}$: (1) high income and collection attempt with probability $\pi\lambda$, (2) high income and no collection attempt with probability $\pi(1-\lambda)$, (3) low income and collection attempt with probability $(1-\pi)\lambda$, and (4) low income and no collection attempt with probability $(1-\pi)(1-\lambda)$.  Given these states, the recursive problem can be written as follows
\begin{align}
\label{eq:valmax}
\begin{split}
V_t(X_{t},s_t) &= max_{w_t,x_{t}} \,\,\, \frac{u(w_{\tau},x_{\tau})^{1-\gamma}}{1-\gamma}   + (1+\delta)^{-1}  E \Big[\, V(X_{t+1},s_{t+1})\Big] \\
X_t &= [w_t,x_t,A_{t+1},B_{t+1},C_{t+1}] 
\end{split}
\end{align}
with equations (\ref{eq:bc}) and (\ref{eq:borrowingconstraint}) holding each month.  This problem can be solved in two steps: 

% (1) choosing $w_t$ and $x_t$ to maximize utility given each possible choice of assets, and (2) given optimal choices of $w_t$ and $x_t$, choosing assets to maximize $V_t(.)$ by iterating backwards from month $T$.


\textbf{Step 1:} For each possible choice of assets, $A_{t+1}$ and $B_{t+1}$, and connection status, $C_{t+1}$, households maximize utility over water, $w_t$, and other goods, $x_t$, subject to their budget constraint and borrowing constraint.  Within each month, households maximize utility given by $ x_t -  \frac{1}{2} (w_t - \alpha)^2 $ such that the budget constraint and borrowing constraint hold.  When the borrowing constraint does not bind, then households reach an interior solution under quasi-linear preferences and prices, $p=p_1+p_2w_t$, given by $w_t^{*} =  \frac{ \alpha - p_1}{2p_2 + 1}$.  The borrowing constraint does not bind when connected households have small unpaid bills (ie. $B_{t+1}$ is small).  Households also consume $w_t^{*}$ in months where $w_t$ does not affect the borrowing constraint such as when households are disconnected or when households receive collection attempts.

When connected households want to use large unpaid bills as a source of credit (ie. $B_{t+1}$ is very negative), then the  interior solution, $w_t^{*}$, may be too small to satisfy the borrowing constraint.  Let $\overline{w}_t$ denote water consumption that just satisfies the borrowing constraint so that the bill this month, $-(p_1 +p_2\overline{w}_t)\overline{w}_t$ equals demand for additional borrowing, $B_{t+1}-B_{t}$ which gives $\overline{w}_t = \frac{(p_1^2 - 4p_2 (B_{t+1}-B_{t}))^{.5} - p_1}{2 p_2}$.  When $w_t^{*}=\overline{w}_t$, the interior solution just satisfies the borrowing constraint.  When $w_t^{*}<\overline{w}_t$, the interior solution is unable to satisfy the borrowing constraint so households consume $\overline{w}_t$.  When $w_t^{*}>\overline{w}_t$, the  interior solution more than satisfies the constraint so households consume $w_t^{*}$.

\textbf{Step 2:}  Given optimal consumption choices of $w_t$ and $x_t$, households then choose their assets and whether to remain connected each month to maximize $V_t(.)$.  $V_t(.)$ can be solved by working backwards from month $T$ where $A_T = 0$.  Since households are assumed to have positive time preferences ($\delta>0$) and zero interest on their savings, households have an incentive to consume their initial assets until they reach their borrowing constraints (\cite{deaton1991saving}).  






\section{Estimation and Results}\label{section:estimation}    % https://www.cgdev.org/blog/compartamos-context


% -- Representative household: Where is heterogeneity coming from then?
% -- Assuming log utility  (justify that please!)

Quantifying the welfare consequences of different billing policies requires estimates of household preferences, the relative cost of an alternative water sources, and the rate at which households receive collection attempts.  The empirical strategy matches simulated consumption and payment for a representative household to average characteristics in the billing data.  This strategy estimates the preference for water $\alpha$ from observed water consumption.  The rate of collection attempts $\lambda$are estimated from the level of observed unpaid water bills.  Frequent collection attempts reduce the incentive for households to accumulate unpaid water bills.  The fixed cost of using an alternative water source $f$ from the observed share of households that temporarily disconnect in response to collection attempts.  Higher fixed costs discourage households from disconnecting when they receive a collection attempt.

The empirical strategy requires calibrating prices, income, and interest rates to match economic conditions for the average household in Manila.  The strategy uses a fixed price per $\text{m}^{3}$ of $p_1=17.5$ PhP and a marginal increase in price per $\text{m}^{3}$ of $p_2=0.29$ PhP, which together reflect the linear function that best fits the increasing tariff in Manila as shown in Appendix Table~\ref{table:tarifftrue}.\footnote{Appendix~\ref{appendix:tariff} includes more details on this calculation. }  This approximation captures how prices increase with usage while also allowing for computational tractability within the dynamic model.\footnote{\cite{wjv} and \cite{szabo2015value} carefully capture non-linear pricing incentives with static models, which are computationally expensive.  The linear approximation also parallels average pricing models of consumer demand for utilities as suggested by \cite{ito2014consumers}.}  

Mean monthly income, $\overline{y}$, is calibrated to average income in the household survey of Pasay City, taking a value of \input{tables/y_avg}PhP.  Using panel structure between 2008 and 2011 in the same survey, monthly variation in income, $\theta$, is calculated by dividing the standard deviation of household income net of household fixed effects by mean income for a value of \input{tables/y_cv}\unskip.  For $\theta$ to capture monthly income variation within households, this calibration requires the assumption that reported income changes between 2008 and 2011 are representative of monthly income changes for households.  This estimate for $\theta$ falls within range of previous estimates in the literature.\footnote{$\theta$ can also be interpreted as measuring the coefficient of variation (CV), which measures the standard deviation of monthly household income divided by average household income (\cite{hannagan2015income}).  \cite{hannagan2015income} use monthly financial diaries in the US to calculate CVs of 0.39 for average households and 0.55 for households below the poverty line.  Using household surveys from Mexico, \cite{amuedo2011remittances} calculate CVs between 0.29 and 0.46.}  Households are also assumed to face even probabilities of high or low income realizations each month, which implies that $\pi$ equals 0.5.  The monthly interest rate on borrowing from standard assets is calibrated to equal \input{tables/irate}\unskip\% according to reported rates for ``all-purpose'' loans from the Consumer Finance Survey for Metro Manila.  The majority of these loans are issued from money lenders.

The estimation also assumes parameters for household time-preferences.  First, the monthly discount rate $\delta$ is assumed to equal 0.005, which implies an annual discount rate of 6\%. This discount rate is in range of estimates from similar structural dynamic models.\footnote{In US contexts,\cite{gourinchas2002consumption} use a similar structural approach finding a lower discount rate of around 5\% while \cite{laibson2007estimating} recover a discount rate of around 15\%.}  Second, the coefficient of relative risk aversion $\gamma$ is set equal to 1, which is approximated by the natural logarithm and falls within a wide range of estimates from the literature.\footnote{The literature provides a range of estimates for $\gamma$ which are above, below, and around one.  \cite{barseghyan2013nature} use insurance choices in the US to estimate a $\gamma$ between 0.21 and 0.37 while  \cite{beetsma2001measuring} use a natural experiment from a game show to estimate $\gamma$ ranging from 0.42 to 6.99.  \cite{carvalho2016effect} leverage an experimental setting in Nepal to estimate $\rho$ equal to 0.63.  Given these estimates, assuming $\rho$ equal to one implies a moderate curvature of the utility function and is relatively close to a comparable estimate from a development economics setting.}

Households are assumed to remain in Manila for 384 months before disconnecting, consistent with a 0.13\% rate of permanent disconnection rate for households in the sample.  Upon disconnection, \input{tables/bal_0_end_per}\unskip\% of households pay all of their remaining balances, which calibrates the share $\chi$ of households that are assumed to move to a new dwelling within Manila.

The estimation routine solves the household's problem in equation (\ref{eq:valmax}) through backwards induction.  In the last month $T=384$, the routine finds the combination of water and standard assets that maximize utility under the constraint that households must pay their standard debt and also may have to pay their water debt depending on whether they plan to stay in Manila (which occurs with probability $\chi$).  Taking these choices as given, the routine then steps back to month $T-1$ and finds the combination of assets (and connection status) that maximizes current utility plus discounted future utility.  Repeating this process brings households to a steady state mapping of current assets/connection status to future assets/connection status.  

The routine allows households to choose across 40 standard and 40 water asset values.\footnote{Estimates are similar using denser grids with results available upon request.}  Standard assets range from borrowing twice average household income to saving three times average household income.  In simulations, households are never observed borrowing at the maximum.  Households are observed to save the maximum amount in less than 5\% of months.  Water assets are bounded below by the 99th percentile of observed water debt, which is equal to \input{tables/Bb}PhP.  In practice, households rarely  .  

Many households choose to borrow the maximum .

Households can choose across 40 values of standard assets ranging from borrowing twice their average income to saving three times their average income.


 by iterating over a grid of standard asset and water borrowing asset values and choosing the set of assets that maximize the household's value function.  Households can choose from \input{tables/par_nA}values of the standard asset, $A_{t+1}$, 

evenly spaced across a mean zero normal distribution with a standard deviation of \input{tables/par_sigA}\unskip, a minimum of \input{tables/par_Amin}\unskip, and a maximum of \input{tables/par_Amax}\unskip.  Households can also choose how much to borrow from unpaid water bills, $B_{t+1}$, over \input{tables/par_nB}values evenly spaced across a mean zero normal distribution truncated above at zero with a standard deviation of \input{tables/par_sigB}and a minimum of \input{tables/par_Bmin}\unskip.  The additional choice of whether to stay connected each period, $D_{t+1}$, brings the total possible number of asset combinations to \input{tables/par_totalsize}\unskip.\todo[color=red]{i'm completely ignorant here. how do you choose these distributions for the assets?}

The estimation strategy uses a simulated method of moments approach, which chooses parameters to minimize the sum of squared distances between simulated and true moments, weighted by their average values in the data.  To generate simulated moments, the estimator creates a random \input{tables/par_n_iter}month chain of states according to the transition matrix (equation (\ref{eq:tmatrix})) and calculates the household's predicted asset and consumption choices across these states (assuming asset levels of zero to start).\footnote{See \cite{laibson2007estimating} and \cite{gourinchas2002consumption} for similar approaches to estimation.}






The estimation simulates the consumption paths for 26 households composing a total of nearly 10,000 household-months.  




mention 10,000 draws! and grid size!  


Grid!
% Visit risk

Grid size (results available upon request)


\begin{table}[H]
\centering
\caption{Calibrated and Assumed Parameters}\label{table:calibratedparam}
\vspace{-2mm}
\resizebox{\columnwidth}{!}{%
\begin{tabular}{l*{1}{ccl}}
\toprule
%Parameter  &   &  Value & Source \\
%\midrule
Savings Interest Rate & $r_l$ & \input{tables/irate}\unskip\%  & {\footnotesize Philippines data from World Bank Databank (2010-2015) } \\
Water Interest Rate & $r_w$ & 0\% & {\footnotesize The regulator prevents charging interest on unpaid bills} \\
Discount Rate & $\delta$ & 2\% & {\footnotesize Mean of structural estimates from 
$\text{literature}^{\dagger}$} \\
Tariff    & $(p_1 + p_2 w)$ & $(\input{tables/p_int} + \input{tables/p_slope}w)$ & {\footnotesize Estimated price by water usage  (See Appendix~\ref{appendix:tariff} for details) } \\
Mean Inc. (PhP) & $\bar{y}$ & \input{tables/y_avg} & {\footnotesize Family Income Expenditure Survey (2015) for Manila} \\
High Inc. Risk& $\pi$ & 50\% & {\footnotesize Assumed to ensure symmetric income shocks} \\
\bottomrule
\multicolumn{4}{l}{\scriptsize All measures are monthly.  Annual rates are converted to monthly rates as follows: Monthly Rate = $(1+\text{Annual Rate})^{1/12}-1$} \\[-.5em]
\multicolumn{4}{l}{\scriptsize $\text{}^{\dagger}$ See \cite{andreoni2012estimating}, \cite{laibson2007estimating}, and \cite{gourinchas2002consumption} for structural $\delta$ estimates.}
\end{tabular}
}
\end{table}

  % M = (1 + A)^(1/12) - 1 % A = (1 + M)^12 - 1


Table~\ref{table:estimates} provides the estimation results.  The Cobb-Douglas preference for water consumption is estimated to be \input{tables/est_alpha}\unskip, which is consistent with households' observed budget share dedicated to water in Manila.  The estimated income shock of \input{tables/est_theta}implies that household income either increases or decreases by \input{tables/est_theta_per}\unskip\% of its average with 50\% probability each month.  This estimate can also be interpreted as measuring the coefficient of variation (CV) of income and falls on the lower end of previous estimates in the literature.\footnote{The coefficient of variation (CV) measures the standard deviation of monthly household income divided by average household income (\cite{hannagan2015income}).}  \cite{hannagan2015income} use monthly financial diaries in the US to calculate CVs of 0.39 for average households and 0.55 for households below the poverty line.  Using household surveys from Mexico, \cite{amuedo2011remittances} calculate CVs between 0.29 and 0.46. %, which more closely resemble the estimate in Table~\ref{table:estimates}

The estimation recovers a fixed cost of being disconnected of \input{tables/est_fc}PhP/month.  Previous research uses a static, structural approach to estimate a long-term monthly fixed cost from using alternative water sources of 130 PhP/month (\cite{wjv}).  While these estimates fall in a similar range, this paper produces a larger estimate of the fixed-cost likely because sudden disconnections leave little time for households to invest in finding low-cost alternative sources for water.  This estimate is also likely to capture the one-time 200 PhP fee charged to households for reconnection.

The borrowing rate from standard assets is estimated to be \input{tables/est_irate_per}\unskip\% per month, which implies an annual interest rate of \input{tables/est_irate_annual_per}\unskip\%.  This estimate is substantially lower than the 20\% per month interest rate that \cite{karlan2009expanding} document as being commonly charged by moneylenders in Manila.  Despite this high interest rate, \cite{karlan2009expanding} document that at least 30\% of their sample of microentrepreneurs report taking credit from moneylenders.  The estimated borrowing rate of \input{tables/est_irate_per}\unskip\% is more similar to microloans of 1,000 PhP at 2.5\% monthly interest offered to rural Filipino households as part of a microfinance experiment conducted by \cite{gine2014group}.\footnote{The annual interest rate of \input{tables/est_irate_annual_per}\unskip\% well exceeds the average 13 to 25\% range offered by microfinance providers worldwide surveyed by \cite{cull2009microfinance}.  Two possible reasons for this discrepancy are that (1) institutional reasons unique to Manila may limit lenders' abilities to offer low rates and (2) subsidies may allow many microfinance providers to offer below-market interest rates.}

\begin{table}[h!]
\centering
\caption{Estimates}\label{table:estimates}
\vspace{-2mm}
%\resizebox{\columnwidth}{!}{%
\begin{tabular}{l*{1}{cc}}
\toprule
Parameters  &   & Estimates \\
\midrule
Water Preference & $\alpha$ & \input{tables/est_alpha} \\
 &  & (\input{tables/est_sd_alpha}\unskip) \\[.4em]
% Income Shock Magnitude & $\theta$ & \input{tables/est_theta} \\
 % &  & (\input{tables/est_sd_theta}\unskip) \\[.4em]
Fixed Cost of being Disconnected (PhP) & $f$ &  \input{tables/est_fc} \\
 &  &  (\input{tables/est_sd_fc}\unskip) \\[.4em]
Collection Rate & $\lambda$ & \input{tables/est_irate} \\
 &  & (\input{tables/est_sd_irate}\unskip) \\[.8em]
Households & & \input{tables/est_hhs} \\
Household-Months & & 2,118,861 \\
\bottomrule
\multicolumn{3}{l}{\scriptsize Standard errors in parentheses are bootstrapped at the household-level.} % with \input{tables/breps}repetitions 
\end{tabular}
%}
\end{table}

Table~\ref{table:fit} provides both moments in the data used for estimation as well as other moments to help understand the fit of the model.  While the model is able to almost exactly match average usage and outstanding balance, the model has more difficulty matching the slow decline in disconnection rates observed after delinquency visits.  The model instead predicts that households who disconnect in response to a delinquency visit will quickly reconnect over the following four months.  One explanation for this discrepancy may be that the distribution of income shocks does not allow for serial correlation so that households are assumed to quickly recover from negative shocks in the model.  In reality, households may be temporarily disconnecting in response to longer-term negative income shocks like job-loss or illness.  A similar pattern exists for disconnection conditional on being at least 90 days overdue when visited.\todo[color=red]{how impossible is it to include this serial correlation in the model/computations?}

\begin{table}[H]
\centering
\caption{Model Fit}\label{table:fit}

\begin{tabular}{l*{1}{cc}}
\toprule
 Moments & Data  & Predicted \\[.5em]
Used in Estimation & & \\
Consumption (m3) &\input{tables/c_avg} &\input{tables/fit_usage} \\
Balance (PhP) &\input{tables/bal_avg} &\input{tables/fit_bal} \\
Disconnected &\input{tables/dc_shr}& \input{tables/fit_dc} \\[.4em]
Unused in Estimation & & \\
% \midrule


\bottomrule
\multicolumn{3}{l}{\scriptsize ``Disconnection'' is measured before permanent disconnection. }  \\[-.5em]
\multicolumn{3}{l}{\scriptsize ``Corr.'' stands for correlation.   }  \\[-.5em]
% \multicolumn{3}{l}{\scriptsize are calculated with variation within households.}
\end{tabular}
%\begin{tabular}{lcc}
& Data & Estimated \\
Mean Usage (m3) &24.9&24.6\\
SD Usage &11.1&2.3\\
Mean Water Debt (PhP) &1205&1412\\
SD Water Debt (PhP) &1258&1955\\
Corr. Usage and Water Debt &0.35&0.05\\
Mean Disc. for 1 month &0.061&0.073\\
Mean Disc. for 2 months &0.061&0.047\\
Mean Disc. for 3 months &0.048&0.030\\
Mean Disc. for 4 months &0.040&0.017\\
\end{tabular} 

\end{table}

The model has difficultly matching moments that were not used in the estimation.  Since log-utility encourages households to smooth their consumption over time, this model predicts very little variation in water usage over time.  By contrast, high observed variation in usage is likely driven by the fact that households face idiosyncratic shocks to their water demand each month as household members travel for work, other families come to visit, or Manila experiences a heat wave.  A more complete model may include a term for indiosyncratic water shocks although it is unclear whether these shocks would substantively affect the model's predictions for income smoothing across time.  While poorly matching variation in usage, the model is able to generate over half of the observed variation in outstanding balances.  

\todo[color=green]{This is your hook for the intro, but also this could be normal substitution from soda, is water a normal good?}The data find a positive correlation between usage and outstanding balances, which suggests that households may take on water debt to fund extra consumption in months where they face large, positive shocks to their water demand.  In the model, positive income shocks reduce demand for water debt and increase demand for water. At the same time, negative income shocks reduce demand for water while increasing demand for water debt.  On net, the model finds zero average correlation between outstanding balances and usage. % Therefore, in some cases, households use more water during negative income shocks in order to increase their borrowing limits.     

% \todo[color=green]{Could you recenter this around zero as the starting position and/or label zero?}
% \begin{figure}[H]
% \centering
% \caption{100 Months of Simulated Data around \\ a Period of Disconnection}\label{figure:deaton}
% \includegraphics[scale=1.1]{tables/new_deaton_graph.pdf} \\
% {\scriptsize  
% Note: 100 months are chosen to center around the first disconnection event in the \input{tables/par_n_iter}month random sequence of income and delinquency visit states used in the estimation.  ``Visit'' indicates months with a delinquency visit with a diamond.  ``Disconnect'' indicates months disconnected with a thick line.  Cum. $\Delta y_t$ measures cumulative shocks to income in PhP, $A_{t+1}$ indicates the optimal position for standard assets in PhP, $B_{t+1}$ indicates the optimal amount of water borrowing in PhP, and $\text{Usage}_t$ indicates water consumption in m3.
% \par}
% \end{figure}


To build intuition, Figure~\ref{figure:deaton} provides 100 time periods of simulated data from the model.  These 100 time periods are chosen to center around the first disconnection occurrence in the \input{tables/par_n_iter}month random sequence of states used in the estimation.  The first panel in Figure~\ref{figure:deaton} indicates the cumulative, exogenous income shocks faced by the household.  This sample begins with a long period where negative income shocks occur more frequently than positive income shocks.  Positive shocks only begin to outweigh negative shocks at around 50 months.  Indicators for when the household receives delinquency visits as well as whether it chooses to remain disconnected are also nested in this first panel.  Over the course of 100 months, the household experiences three visits, the second of which leads the household to disconnect for around 12 months.  This disconnection corresponds to a period where the household has accumulated a long string of negative income shocks.

The second panel indicates the household's choice of asset position, $A_{t+1}$, in each month.  Asset position closely tracks income realizations as the household increasingly borrows (moving into very negative positions) following the long series of negative income shocks.  At around the time of disconnection, the household chooses to borrow the maximum allowed by the grid of assets chosen for the simulation.  Positive income shocks then allow the household to borrow less and begin saving at around 60 months.

The third panel indicates household borrowing through unpaid water bills, $B_{t+1}$.  In the beginning, the household increases its borrowing more slowly than with standard assets since each month's increase in borrowing is limited by the size of the household's current water bill.  Matching the downturn in income, the household continues borrowing before reaching the maximum borrowing allowed by the grid of assets chosen in this simulation.  With few positive income shocks, the household remains at this maximum borrowing level for at least 24 months.  When the second delinquency visit occurs at around 40 months, the household is still borrowing the maximum from unpaid water bills and therefore, instead of choosing to pay its full outstanding balance to stay connected, the household chooses to disconnect until it accumulates enough positive income shocks to pay its full water bill around month 55.  During the third delinquency visit, the household's outstanding balance happens to be relatively small so it is able to immediately pay in full to remain connected.

\todo[color=green]{Why lumpy/spikey? Why doesn't usage just go up smoothly?}The fourth panel indicates household water usage patterns over the same 100 months.  Usage begins to spike as the household increases its usage to fund borrowing through unpaid water bills.  The largest spikes in usage occur when the household moves to the maximum level of borrowing.  Because of the step-size chosen for the asset grid, moving to the largest borrowing level requires a jump in unpaid bills of around 1,000 PhP.  Since the average bill is around 600 PhP, the household needs to almost double its usage to fund this jump in borrowing from unpaid bills.  These spikes in usage measure the extent to which borrowing from unpaid water bills may distort consumption choices, adding an additional friction associated with borrowing from water bills.  After maximizing water borrowing at around 24 months, usage begins to stabilize at lower levels, mirroring the long string of negative income shocks faced by the household.  Usage recovers to a higher level after the second delinquency visit before spiking again as the household maximizes its water debt in the last several months.\todo[color=green]{If this is just a grid-size quirk, don't highlight it like its!  Borrowing limit is awkward}

\section{Counterfactual Policies}\label{section:counterfactuals}
%In practice, this counterfactual setting may correspond to a policy of strict enforcement by the utility (ie. monthly delinquency visits). 
To measure how much households value credit from unpaid water bills, I examine how household welfare changes in a counterfactual setting where households are unable to borrow from their water bills.  Table~\ref{table:counter} includes outcomes for the current setting in Manila in Column (1) and for a counterfactual setting without water borrowing in Column (2), which is captured by raising the interest rate on unpaid water bills to 100\% and holding all else equal.  The first row calculates compensating variation equal to \input{tables/U_nl_abs}PhP (or \input{tables/U_nl_abs_usd}USD) per household-month associated with losing access to water credit.  This estimate suggests that households are indifferent between their current situation and a counterfactual with \input{tables/U_nl_abs}PhP per month in additional income and without access to water credit.  Given an average water bill of \input{tables/bill_all}PhP/month, this estimate would translate into households paying around \input{tables/U_nl_per}\unskip\% smaller bills each month.  

Eliminating credit access also decreases mean usage by \input{tables/c_h_nl_drop}\unskip\% as shown by columns (1) and (2) in the second row of Table~\ref{table:counter}.  Eliminating credit access lowers the extent to which households can smooth their consumption over time while also removing the incentive for households to overconsume water in order to finance water borrowing.  Given that households spend around \input{tables/bill_inc_all}\unskip\% of their income on water, this estimate is roughly proportional to similar evidence from South Africa where restricting credit access with prepaid electricity meters produced a 13\% reduction in usage and where households spend around 8-10\% of their income on electricity (\cite{jack2016charging}).\footnote{\cite{jack2016charging} also propose other mechanisms that may account for reductions in usage such as transaction costs and intra-household bargaining constraints.}

\begin{table}[H]
\centering
\caption{Counterfactual Policies}\label{table:counter}
\resizebox{\columnwidth}{!}{%
\begin{tabular}{l*{1}{CCCC}}
\toprule
 & (1)     & (2)        & (3)                         & (4) \\
 & Current & Prepaid Metering   &  Optimal Collection Visits  &  Utility as a Bank  \\
\midrule
Compensating Variation  &   & \input{tables/cv_comp_nlpp}  & \input{tables/cv_comp_opt}  & \input{tables/cv_comp_ut} \\[.2em]
Consumption 			 &   \input{tables/cons_val_reg} & \input{tables/cons_val_nlpp}    &  \input{tables/cons_val_opt} &  \input{tables/cons_val_ut} \\[.2em]
Outstanding Balance  & \input{tables/debt_reg} & \input{tables/debt_nlpp}    &  \input{tables/debt_opt} &  \input{tables/debt_ut} \\[.2em]
Fixed Price  & 0 & \input{tables/fee_nlpp}    &  \input{tables/fee_opt} &  \input{tables/fee_ut} \\[.2em]
Collection Risk & \input{tables/vrate_reg}& \input{tables/vrate_nlpp}  & \input{tables/vrate_opt} & \input{tables/vrate_ut} \\
\bottomrule
\multicolumn{5}{l}{ \scriptsize  All values are at the household-month level. }
\end{tabular}
}
\end{table}



%\subsection{Prepaid Metering}

This setting also provides a useful opportunity to simulate the social welfare impacts of popular policies to reduce delinquency.  I consider a policy that (1) eliminates borrowing by raising the interest rate on unpaid bills to 100\% and (2) adjusts prices to ensure that the utility remains revenue-neutral.  The regulatory structure in Manila as well as many other developing cities ensures that prices for water are regulated to exactly cover all production costs (\cite{hoque2013state}).

Eliminating borrowing affects the costs of the utility in four main ways

\begin{enumerate}
\item \textit{Opportunity Cost of Lending}: Currently, the utility faces an opportunity cost for the loans extended to households in the form of unpaid bills.  Assuming that the utility would have been able to invest this money at an average annual interest rate of \input{tables/irate}\unskip\%, the opportunity cost of lending averages \input{tables/coste}PhP per household-month, which would be recouped by the utility in a counterfactual without delayed payments.\footnote{Interest rate reflects the average between 2010 and 2015 as reported by the World Bank Databank.}

\item \textit{Delinquency Visit Cost}: Without water borrowing, the utility would no longer need to conduct delinquency visits.  Conversations with the utility suggest that travel costs make up the majority of the costs for any service performed on a water meter.  Since the utility requires a 200 PhP fee to reconnect disconnected households, I assume that delinquency visits cost the same amount to the utility.  Conditional on being delinquent, households receive visits in \input{tables/prob_caught}\unskip\% of household-month observations, which implies an average delinquency visit cost to the utility of \input{tables/visit_cost}PhP per household-month. 

\item \textit{Marginal Costs}: The utility reports a marginal cost per cubic meter of consumption equal to 5 PhP.

\item \textit{Disconnection Rebate:}  Currently, the utility is exposed to default risk where households that permanently disconnect from the utility often leave large outstanding balances that are never paid.  On average, \input{tables/dc_per_month_account}households permanently disconnect per household-month.  These households that permanently disconnect leave average outstanding balances of \input{tables/out_bal}PhP.  These estimates imply household savings equal to an average of \input{tables/delinquency_cost}PhP per household-month.  In practice, households enjoy all of these savings in the final few months that they remain connected.  However, since households use water indefinitely in the model, the counterfactual exercise captures these savings by assuming that households receive a monthly fixed disconnection rebate of \input{tables/delinquency_cost}PhP.  By evenly spreading these savings over time, this approach is likely to overstate the true benefits to households from leaving unpaid bills.  This approach relies on the following additional assumptions \todo[color=green]{How is this a cost? I'm confused by this section.  Is this real or model assumption or counterfactual? Definitely need to rewrite this better!!}

\begin{itemize}
    \item By assuming that households receive fixed rebates, this approach ignores any incentives that households may face to overconsume in their final months connected (since households may behave as if they face an effective price of zero in these months).  Appendix~\ref{appendix:permanentdc} finds some evidence of overconsumption before permanent disconnection although the short duration and small magnitude of overconsumption suggest that excluding this margin will have little impact on total welfare estimates.

    \item Household decisions over when and whether to permanently disconnect from service are assumed to be unaffected by changes in billing flexibility.  Since permanent disconnections are likely to be driven by households changing residences, this assumption is consistent with quality of water access having little effect on household location decisions.

    \item This approach assumes that under a counterfactual setting with a high borrowing interest rate, households always pay their bills on time even when they are about to permanently disconnect.  This assumption may be reasonable since households would only have an incentive to delay their payments when they are close to permanently disconnecting.  Since permanent disconnections are rare, the utility can credibly threaten to disconnect households as soon as they stop paying their bills.  This threat would likely ensure that households always pay their bills on time.
\end{itemize}    
\end{enumerate}  

To determine price increases necessary to cover these costs, the following expression first calculates the revenue raised through prices,  $p_1$ and $p_2$, per household-month given income, $Y$, net of marginal cost, $MC$
\begin{align*}
REV \big(p_1,r_b,Y \big) &= \Big ( \,  p_1 - MC +  p_2 \, w^{*}(p_1,p_2,r_b,Y)  \,  \Big ) \, \, w^{*}(p_1,p_2,r_b,Y)
\end{align*}
%With prepaid meters, the  utility would also no longer need to conduct delinquency visits.  

%Appendix~\ref{appendix:visitcosts} provides evidence that these cost savings are likely to be negligible.

I then solve for the new price intercept, $p^{\prime}_1$, such that the current revenue is equal to revenue under a counterfactual where the borrowing rate is equal to 100\% and the utility enjoys cost savings.   This exercise assumes that the government regulator is able to perfectly forecast water demand among all households.  Adjusting only the price intercept, $p^{\prime}_1$, instead of both price terms provides a way of preserving the slope of the tariff structure, which likely reflects equity concerns among policymakers in Manila.  

$p^{\prime}_1$ is chosen to solve the following expression
\begin{align*}
\sum_t REV_t \big(p_1,0,Y_t \big) =& \sum_t REV_t \big(p^{\prime}_1,1, Y_t - \text{Disc. Rebate} \big) \\
& - (\text{Opp. Cost of Lending + Visit Cost + Disc. Rebate})
\end{align*}
In the counterfactual, cost savings reduce the amount of revenue needed to be raised to stay revenue neutral.  At the same time, eliminating borrowing in the counterfactual lowers water consumption, which reduces revenue since prices are well above marginal costs.  Table~\ref{table:counter} finds that these two effects almost exactly offset each other, producing almost identical prices in the current setting in Column (1) and in the counterfactual without water borrowing in Column (3).  Removing water borrowing while keeping similar prices results in a drop in average usage between Columns (1) and (3) that is nearly identical to the drop in usage between Columns (1) and (2).

According to the first row of Table~\ref{table:counter}, households would require a compensating monthly payment of at least \input{tables/U_ppe_abs}PhP in order to move from the current setting in  Column (1) to a revenue-neutral counterfactual without borrowing in Column (3).  Although prices remain almost identical in Column (3), revenue neutrality ensures that households no longer receive a monthly disconnection rebate of \input{tables/delinquency_cost}PhP, which almost exactly accounts for the difference in compensating variation between columns (2) and (3).  


I then simulate the effects of introducing prepaid metering technologies and adjusting water prices to similarly account for their costs.  By requiring households to pay upfront for their water usage, these meters provide an alternative strategy for eliminating unpaid water bills.\todo[color=red]{does this constrain my water use? i can only use what i paid for? oh or is it like prepaid cell phones and you just pay more when your balance is running low?}  These technologies have become increasingly popular for both electricity and water utilities throughout the developing world.\footnote{See \cite{jack2016charging} and \cite{northeast2014} for electricity utilities and \cite{heymans2014limits} for water utilities.}  These technologies may be especially useful in contexts where other factors drive delinquency instead of consumption smoothing.   For example, \cite{szabo2015reducing} suggest that low levels of trust in local government and perceptions of fairness may drive some nonpayment behavior.  While these factors are not explicitly modeled in this context, this exercise may still provide a useful lower bound for evaluating the welfare effects of prepaid metering programs.

% as in column (2); however, I also approximate the impact of  as well as the welfare effects of passing these costs onto consumers through higher prices consistent with regulatory structure in Manila.
%The welfare effects of installing prepaid meters also depend on the costs of

Prepaid meters introduce an additional cost for the utility in terms of purchasing and installing this new technology.  \cite{heymans2014limits} surveyed eight large water providers that implemented prepaid meters in developing countries and found that each prepaid meter costs about four times as much as a standard meter and requires replacement every 7 years.  In the context of Manila, each standard meter costs around 1,500 PhP and is replaced around every 6 years and 3 months, bringing the monthly cost to 20 PhP/month.\footnote{The utility provided additional documentation of their costs and frequency of meter replacement for residential households.}  Assuming that a prepaid meter costs 4 times as much as a standard meter with a replacement rate of 7 years, the estimated monthly cost of a prepaid meter would be 71 PhP/month.  Therefore, prepaid meters imply an additional cost of 51 PhP/month per household.\footnote{\cite{heymans2014limits} also report that the fixed administrative costs of installing and monitoring new meters account for less than 4\% of the total costs of switching to prepaid meters while the bulk of the expenses come from purchasing new meters.  By focusing on meter replacement, this exercise is likely to capture the majority of switching costs associated with prepaid metering.}  

Column (4) of Table~\ref{table:counter} includes the results of the prepaid metering counterfactual.  To cover the much higher costs of prepaid meters, the price intercept, $p_1$, increases by around \input{tables/p_increase_per}\unskip\% as indicated by the fourth row.  Households also no longer receive a disconnection rebate of \input{tables/delinquency_cost}PhP per month.  With higher prices and without water credit, households lower their consumption by \input{tables/c_h_pp_drop}\unskip\% under prepaid metering compared to the current setting in Column (1).  In order to be indifferent between the current setting and a counterfactual with prepaid metering, households would need to receive \input{tables/U_pp_abs}PhP/month in compensation.  Taken together, these results provide suggestive evidence that prepaid metering would reduce welfare substantially in this context.

%%% also include counterfactual where we increase the interest rate and see substitution? may have to rewrite some of the identification section....

\section{Conclusion}\label{section:conclusion}

Prepaid meters for electricity already compose over 27.5\% of residential meters in Sub-Saharan Africa and are predicted to grow to 52.8\% by 2024 (\cite{northeast2014}).  Similarly, by planning to install over 300,000 prepaid meters, the Botswana Water Utilities Corporation provides an example of the growing use of this technology in the water sector (\cite{heymans2014limits}).  Policy proposals for prepaid meters often emphasize how this technology ensures cost-recovery for utility providers.  At the same time, households stress how ``water is a need, but money is not always available'' and  how ``postpaid gives you more time to find the money'' in qualitative evidence documented by \cite{heymans2014limits}.  These anecdotes suggest a potentially important role for billing flexibility in allowing households to smooth consumption.  

This paper builds a dynamic model of household consumption smoothing to measure the extent to which households value billing flexibility.  Estimates imply that households' valuation of flexibility is on the order of \input{tables/U_nl_per}\unskip\% of their monthly water bills.  Counterfactual exercises further find that policies to eliminate nonpayment --- raising interest rates on unpaid bills and prepaid metering --- do not produce enough cost savings to justify their negative impacts on household consumption smoothing.\todo[color=red]{i'm still not sure i get this... what are the cost savings?}  On net, these policies are predicted to reduce social welfare by between \input{tables/U_ppe_abs_usd}and \input{tables/U_pp_abs_usd}USD per household-month.  \todo[color=red]{i would add more oomph to this. i think this is really interesting but the punchline needs to be bigger!}

In focusing on the role of consumption smoothing, this approach abstracts away from other channels by which nonpayment may affect welfare.  First, high rates of nonpayment may weaken incentives for utilities to invest in and extend access to high quality infrastructure.  While regulators in Manila successfully ensure universal access and service quality, \cite{mcrae2015infrastructure} documents how electricity providers in Colombia often shirk on infrastructure investments in areas where they face high levels of nonpayment, which are also often underprivileged areas.  Second, policing nonpayment through unexpected disconnections may also have unintended health consequences (\cite{franklin2017}).  The US Department of \cite{liheap} lists a series of state-level policies restricting disconnections, especially in months with extreme temperatures, for public health reasons.  Finally, nonpayment may provide a way for households to voice their dissatisfaction with a utility as well as local government.  \cite{szabo2015reducing} find evidence that reaching out to consumers on behalf of a water utility increased payments from households motivated by a sense of reciprocity.  

%Third, \cite{jack2015pay} also argue that prepaid meters may help savings constrained consumers by allowing low-income households to pay in small increments each month.

% The counterfactual exercises constructs a framework for incorporating households' demand for consumption smoothing into evaluations of prepaid meters and other policies to reduce nonpayment.  

% By focusing on consumption smoothing, this paper abstracts away from several other consequences of utility billing policy.  First, many governments require utilities to tolerate nonpayment due to health concerns.  For example, the NAACP cites stuff ....  cite some more of Jack!! what mechanisms? savings constraints?  intrahousehold bargaining; go look up her new paper!!!!



% https://liheapch.acf.hhs.gov/Disconnect/disconnect.htm

%% compared to prepaid electricity
% • Prepaid water is often seen as controversial. Payment for the supply of electricity is accepted more widely than
% payment for water, and access to electricity is not regarded a basic human right. T
% • The electricity sector is much less fragmented than the water sector, and has far greater clout to direct what
% manufacturers supply. 
% • Prepaid water meters face physical stresses that do not apply to electricity. There are more moving parts,
% most subjected to fluctuating pressures and flows, and wear, fatigue, and abrasion increase the likelihood of
% malfunction. 


% When discussing costs with service providers, the review
% team was told on several occasions that “prepaid meters cost
% about four times more than a conventional meter.” This is
% based on the typical cost of a prepaid metering device for an
% individual domestic connection, about US$210, compared
% to typical conventional mechanical meter, about US$50.   96.2\% of costs for large systems.  average of 7 year life span (assume the same for both)


% \cite{northeast2014} predict rapid increases in prepaid electricity meters throughout Sub-Saharan Africa.  

% \cite{heymans2014limits}
%  prepayment water systems are in use in more than
% 20 African countries, and in locations such as Turkey, parts
% of the Balkans and Azerbaijan, and Colombia. Their scale
% of use is ramping up rapidly. The Botswana Water Utilities
% Corporation, for example, is reported to be planning to install
% 300,000 prepaid water meters in the near future, beyond
% the existing small-scale installations. 

%  don't like : “Postpaid gives you more time to find the money.”  “Water is a need, but money is not always available.”






\section{Appendix}


% \subsection{Transition Matrix}\label{appendix:transitionmatrix}
% \resizebox{.85\linewidth}{!}{
%   \begin{minipage}{\linewidth}
% \begin{align}
% \begin{split}
% \label{eq:tmatrix}
% T_{t,t+1} &= \bordermatrix{\text{} & (1+\theta)\bar{y}\,,\,\text{no visit} & (1-\theta)\bar{y}\,,\,\text{no visit} & (1+\theta)\bar{y}\,,\,\text{visit} & (1-\theta)\bar{y}\,,\,\text{visit} \cr
%                 (1+\theta)\bar{y}\,,\,\text{no visit} & \pi (1-\lambda) & (1-\pi) (1-\lambda) & \pi \lambda & (1-\pi) \lambda  \cr
%                 (1-\theta)\bar{y}\,,\,\text{no visit} & \pi (1-\lambda) & (1-\pi) (1-\lambda) & \pi \lambda & (1-\pi) \lambda  \cr
%                 (1+\theta)\bar{y}\,,\,\text{visit}    & \pi (1-\lambda) & (1-\pi) (1-\lambda) & \pi \lambda & (1-\pi) \lambda  \cr
%                 (1-\theta)\bar{y}\,,\,\text{visit}  & \pi (1-\lambda) & (1-\pi) (1-\lambda) & \pi \lambda &  (1-\pi) \lambda  }
% \end{split}
% \end{align}
%   \end{minipage}
% }

% \subsection{Households Receiving a Delinquency Visit}
% \begin{figure}
% \centering
% \caption{Share of Households that Receive a Delinquency Visit \\ depending on Days Delinquent in the Previous Month}\label{figure:dc_hazard}
% \includegraphics[scale=.7]{tables/connected_visit_hazard_all.pdf}
% \end{figure}


% \subsection{Predicting Delinquency Visits}

% \begin{table}[H]
% \small
% \centering
% \caption{ Linear Probability of Receiving a Delinquency Visit }\label{table:tcd_predict}
% \vspace{-2mm}
% \begin{tabular}{lCCC}
% \toprule
% & \small (1) & \small (2) & \small (3)  \\
% \midrule 
% Usage t-1           &  -0.0000150\textsuperscript{a}&  -0.0000186\textsuperscript{a}&  -0.0000225\textsuperscript{a}\\
                    & (0.0000052)                   & (0.0000052)                   & (0.0000079)                   \\[0.5em]
Days Delinquent t-1 &   0.0000524\textsuperscript{a}&   0.0000581\textsuperscript{a}&   0.0000499\textsuperscript{a}\\
                    & (0.0000016)                   & (0.0000016)                   & (0.0000020)                   \\[0.5em]
Unpaid Balance t-1  &   0.0000009\textsuperscript{a}&   0.0000009\textsuperscript{a}&   0.0000011\textsuperscript{a}\\
                    & (0.0000001)                   & (0.0000001)                   & (0.0000001)                   \\[0.5em]
Single House        &  -0.0001480                   &  -0.0001468                   &                               \\
                    & (0.0002032)                   & (0.0002050)                   &                               \\[0.5em]
Apartment           &  -0.0004173\textsuperscript{b}&  -0.0005555\textsuperscript{a}&                               \\
                    & (0.0001861)                   & (0.0001869)                   &                               \\[0.5em]
Age of HoH          &  -0.0000398\textsuperscript{a}&  -0.0000373\textsuperscript{a}&                               \\
                    & (0.0000040)                   & (0.0000040)                   &                               \\[0.5em]
HoH Low Skill Empl. &   0.0005415\textsuperscript{a}&   0.0005333\textsuperscript{a}&                               \\
                    & (0.0001827)                   & (0.0001830)                   &                               \\[0.5em]
HH Size             &   0.0003462\textsuperscript{a}&   0.0003482\textsuperscript{a}&                               \\
                    & (0.0000363)                   & (0.0000364)                   &                               \\[0.5em]
Employed HH Members &  -0.0002883\textsuperscript{a}&  -0.0002957\textsuperscript{a}&                               \\
                    & (0.0000573)                   & (0.0000574)                   &                               \\[0.5em]
Location            &                               &  \checkmark                   &                               \\
Year \tim Month \textsc{FE}&                               &  \checkmark                   &  \checkmark                   \\
Household \textsc{FE}&                               &                               &  \checkmark                   \\
N                   &   1,951,543                   &   1,948,783                   &   1,951,543                   \\
Mean Visits Per Month&      0.0072                   &      0.0072                   &      0.0072                   \\

% \bottomrule
% \multicolumn{4}{l}{\footnotesize Std. errors clustered at the HH-level. \textsuperscript{c} p$<$0.10,\textsuperscript{b} p$<$0.05,\textsuperscript{a} p$<$0.01 }
% \end{tabular}
% \end{table}


\subsection{Sample Construction}\label{appendix:sampleconstruction}

Merging the full sample from the connection survey to the billing data yields an initial population of 3,343,644connection-months as described in Table~\ref{table:sampleconstruction}.  Non-residential accounts are first removed to ensure that results apply to household-level decisions.  Due to some data inconsistencies, payment records are missing for some connections, which are excluded.  Due to leaks, meter replacements, and meter reading errors, connections occasionally experience extremely high meter readings and bills.  Consumption records above 200 m3 as well as bills and payments above 80,000 are censored to address these issues.  Large negative payments and outstanding balances (due to reimbursements of billing errors) are also excluded due to likely measurement error.  Remaining negative payments and balances likely represent refunds to households (possibly due to system errors), which are included to accurately reflect each household's asset position.  Households that connect during the sample or have large stretches of missing records are excluded by including only connections with over 30 months of data.  Keeping only connections serving single households brings the final sample size to 2,073,884household-months.


\begin{table}[H]
\centering
\caption{Sample Construction}\label{table:sampleconstruction}
\vspace{-2mm}
\resizebox{\columnwidth}{!}{%
\begin{tabular}{l*{1}{cc}}
\toprule
 & Observations & Observations Removed  \\
\midrule
Initial sample   &         3,343,644  &     \\
Keep residential connections (excluding commercial)   &            &    \input{tables/class_drop}  \\
Keep connections with payment records &  &   \input{tables/pay_drop}   \\
Keep months with usage under 200 m3    &            &  \input{tables/chigh_drop}     \\
Keep bills $>$ -5,000 PhP and $<$ 80,000 PhP  &  & \input{tables/amount_drop}  \\
Keep unpaid bills $>$ -5,000 PhP and $<$ 80,000 PhP &             &     \input{tables/bal_drop}  \\
Keep payments $>$ -80,000 PhP and $<$ 80,000 PhP    &            &    \input{tables/paylh_drop}   \\
Keep connections with over 30 months of records   &            &   \input{tables/month_drop}    \\
% Keep connections serving a single household   &            &   \input{tables/SHH_drop}    \\
%Drop connections that are disconnected for final yr.  &            &   \input{tables/dc_drop}    \\
Final sample & 2,073,884 & \\
\bottomrule
\end{tabular}
}
\end{table}



% \subsection{Shape of the Utility Function}\label{appendix:utilityshape}

% Log-utility is a special case of Constant Relative Risk Aversion (CRRA) utility given by $u(c) = \frac{c^{1-\rho}}{1-\rho}$ when $\rho=1$.  CRRA is one of the most popular functions for risk aversion in the economics literature (\cite{wakker2008explaining}).  The literature provides a range of estimates for $\rho$ which are above, below, and around one.  \cite{barseghyan2013nature} use insurance choices in the US to estimate a $\rho$ between 0.21 and 0.37.  \cite{beetsma2001measuring} use a natural experiment from a Dutch game show to estimate $\rho$ ranging from 0.42 to 6.99.  \cite{carvalho2016effect} leverage an experimental setting in Nepal to estimate $\rho$ equal to 0.63.  Given these estimates, assuming $\rho$ equal to one implies a moderate curvature of the utility function and is relatively close to a comparable estimate from a development economics setting.  

% \subsection{Indirect Utility Function}\label{appendix:indirectutil}

% \begin{align}\label{eq:vstar}
% \begin{split}
% v^{*} &= 
% \begin{cases}
% \alpha \,\ln(\frac{p_{1}-\sqrt{{p_{1}}^2-8\,L\,\alpha \,p_{2}+8\,Y\,\alpha \,p_{2}+4\,L\,\alpha ^2\,p_{2}-4\,Y\,\alpha ^2\,p_{2}}}{2\,p_{2}\,(\alpha -2)})- \\ \ln(\frac{(\alpha -1)\,(8\,L-8\,Y-4\,L\,\alpha +4\,Y\,\alpha )}{2\,{(\alpha -2)}^2}+ \\
% \frac{(p_{1}\,\sqrt{{p_{1}}^2-8\,L\,\alpha \,p_{2}+8\,Y\,\alpha \,p_{2}+4\,L\,\alpha ^2\,p_{2}-4\,Y\,\alpha ^2\,p_{2}}-{p_{1}}^2)\,(\alpha -1)}{2\,p_{2}\,{(\alpha -2)}^2})\,(\alpha -1)\\
%  &\text{ if } L \geq \widehat{L} \\
%                                                                                                                                                                                                                                                                                                                                                                                              \alpha \,\ln\left(-\frac{p_{1}-\sqrt{{p_{1}}^2-4\,L\,p_{2}}}{2\,p_{2}}\right)-\ln\left(Y\right)\,\left(\alpha -1\right)
 &\text{ if } L < \widehat{L}
% \end{cases} \\
% &\widehat{L} =                                                                                                                                                                                                                                                                                                                                 \frac{Y}{2\,\left(\alpha -1\right)}+\frac{\frac{Y\,p_{2}}{2}+\frac{{p_{1}}^2}{8}-\frac{p_{1}\,\sqrt{{p_{1}}^2-\alpha \,{p_{1}}^2+8\,Y\,\alpha \,p_{2}}}{8\,\sqrt{1-\alpha }}}{p_{2}}

% \end{split}
% \end{align}



\subsection{Tariff Structure and Approximation}\label{appendix:tariff}

% \begin{figure}
% \caption{Example Residential Tariff Presented to Consumers}\label{figure:tarifftrue}
% \begin{center}
% \includegraphics[scale=.8]{tables/tariff_pic_png.png} \\
% \footnotesize{``conn.'' refers to connection, ``cu.m.'' refers to cubic meters of usage \\ per month, and \textbf{P} refers to Philippine Pesos.  Includes Tariff as of February, 2019.}
% \end{center}
% \end{figure}


\begin{table}[H]
\centering
\caption{Example Residential Tariff As Presented to Consumers}\label{table:tarifftrue}
\vspace{-2mm}
\begin{tabular}{l*{1}{r}}
\toprule
Usage (m3) & Price (PhP) \\
\midrule
Under   10    &   \input{tables/f_10}/conn. \\
Over    10    &   \input{tables/f_11}/conn. \\
Next    10    &   \input{tables/c_12}/cu.m. \\
Next    20    &   \input{tables/c_21}/cu.m. \\
Next    20    &   \input{tables/c_41}/cu.m. \\
Next    20    &   \input{tables/c_61}/cu.m. \\
Next    20    &   \input{tables/c_81}/cu.m. \\
Next    50    &   \input{tables/c_101}/cu.m. \\
Next    50    &   \input{tables/c_151}/cu.m. \\
Over   200    &   \input{tables/c_200}/cu.m. \\
\bottomrule
\multicolumn{2}{l}{\scriptsize Mean tariff 2010-2015 with value added tax. }\\[-.5em]
\multicolumn{2}{l}{\scriptsize ``conn.'' refers to connection. }\\[-.5em]
\multicolumn{2}{l}{\scriptsize ``cu.m.'' refers to m3/month.  50 PhP$\sim$1 USD }
\end{tabular}
\end{table}

Table~\ref{table:tarifftrue} provides the monthly tariff structure as it is presented to consumers.  Consumers face a fixed price as well as marginal prices for any usage above 10 m3.  The regulator gradually adjusts prices at roughly yearly intervals in order to ensure that the utility is able to exactly cover its costs.  The marginal price is highly non-linear, accelerating quickly at low usage levels before slowly increasing at high usage levels.  To achieve a tractable approximation of this price schedule, Table~\ref{table:tcd_predict} fits a simple regression model predicting average price as a function of an intercept, $p_1$, and monthly usage levels, $p_2$.  This model predicts that a increase in monthly usage of 10 m3 results in an increase in average price of 2.2 PhP/m3.  

\begin{table}[H]
\small
\centering
\caption{Average Price and Monthly Usage}\label{table:tcd_predict}
\vspace{-2mm}
\begin{tabular}{lc}
\toprule
& \small Avg. Price: $\frac{\text{Bill (PhP)}}{\text{Usage (m3)}}$    \\
\midrule 
Usage (m3)          &        0.29\textsuperscript{a}\\
                    &      (0.00)                   \\[0.5em]
Intercept           &       17.57\textsuperscript{a}\\
                    &      (0.05)                   \\[0.5em]
Household-Months    &   1,861,200                   \\

\bottomrule
\multicolumn{2}{l}{\scriptsize \textsuperscript{c} p$<$0.10,\textsuperscript{b} p$<$0.05,\textsuperscript{a} p$<$0.01 }
\end{tabular}
\end{table}




% \subsection{Disconnection when Visited and Delinquent}\label{appendix:delinquent}

% \begin{table}[H]
% \centering
% \caption{Effect of days delinquent when visited on disconnection in 1 or 2 months following a visit}\label{table:delinquent}
% \vspace{-2mm}
% \begin{threeparttable}
% \begin{tabular}{l*{1}{cc}}
% \toprule
%  & (1) & (2) \\
% \midrule
% Months Delinquent   &       0.036\textsuperscript{a}&       0.022\textsuperscript{a}\\
                    &     (0.001)                   &     (0.005)                   \\[0.5em]
Household FE        &                               &  \checkmark                   \\
Year-Month FE       &                               &  \checkmark                   \\
Mean                &        .201                   &        .201                   \\
R2                  &        0.09                   &        0.80                   \\
N                   &      10,685                   &      10,685                   \\

% \bottomrule
% \end{tabular}
% \begin{tablenotes}
% \item 
% \end{tablenotes}
% \end{threeparttable}
% \end{table}



% \begin{figure}[H]
% \centering
% \caption{Stayers Share of Households that Receive a Delinquency Visit \\ depending on Days Delinquent in the Previous Month}\label{figure:dc_hazardstayers}
% \includegraphics[scale=.7]{tables/connected_visit_hazard.pdf} \\
% { \footnotesize Only stayer households. }
% \end{figure}




% \subsection{Stayer Descriptives}\label{appendix:stayerdescriptives}

% \begin{table}[H]
% \centering
% \caption{Descriptives for Stayers}\label{table:descriptives_stayers}
% \vspace{-2mm}
% \begin{tabular}{l*{1}{cccccc}}
% \toprule
%  & Mean & SD & Min & 25th & 75th & Max  \\
% \midrule
%  Usage (m3)  & 26.2  & 17.5  & 0.0  & 15.0  & 33.0  & 200.0  \\ 
 Bill  & 761  & 1,124  & -4,640  & 287  & 920  & 78,409  \\ 
 Unpaid Balance  & 2,416  & 5,070  & -4,995  & 261  & 2,346  & 79,904  \\ 
 Share of Months with Payment  & 0.60  & 0.49  & 0.00  & 0.00  & 1.00  & 1.00  \\ 
 Payment Size  & 1,214  & 1,498  & 0  & 426  & 1,482  & 61,298  \\ 
 Days Delinquent  & 84.9  & 155.4  & 0.0  & 0.0  & 91.0  & 720.0  \\ 
 Delinquency Visits per HH  & 1.32  & 0.61  & 1.00  & 1.00  & 2.00  & 6.00  \\ 
 Share of Months Disconnected  & 0.03  & 0.17  & 0.00  & 0.00  & 0.00  & 1.00  \\ 

% \bottomrule
% \multicolumn{7}{c}{Total Households: \input{tables/total_hhs_stayers}  Obs. per Household: \input{tables/obs_per_hh_stayers} Total Obs.: \input{tables/total_obs_stayers}}
% \end{tabular}
% \end{table}

% \begin{figure}[H]
% \centering
% \caption{Stayers Share of Households that Receive a Delinquency Visit \\ depending on Days Delinquent in the Previous Month}\label{figure:dc_hazardstayers}
% \includegraphics[scale=.7]{tables/connected_visit_hazard.pdf} \\
% { \footnotesize Only stayer households. }
% \end{figure}

% \subsection{Discussion of the Discount Rate}


\subsection{Alternative Income Test}\label{appendix:alternativeincome}

The following equation empirically tests whether income is more correlated with bill payments than with consumption
\begin{align*}
\Delta Y_{bq} = \beta\, \Delta \textsc{Income}_{bq} \,+\, \gamma_q \,+\varepsilon_{bq}
\end{align*}
where $\textsc{Income}_{bq}$ is proxied by the quarterly change in average labor earnings from Quarterly Labor Force Surveys where $q$ indexes 28 quarters from 2009 to 2015.  Average earnings are computed for 80 unique bins, $b$, defined by combinations of 4 sub-municipalities, age in 5 year intervals, and whether college educated in order to merge individuals from the Labor Force Survey to individuals water connection owners from the water connection survey.  $\Delta Y_{bq}$ is the quarterly change in either average water consumption (measured by the water bill in PhP) or average water bill payments (in PhP) by bin-quarter.  Standard errors are clustered at the bin-level.

\begin{table}[!ht]
\small
\centering
\begin{threeparttable}
\caption{Correlations of Income with Consumption and Payment Patterns}\label{table:lfsanalysis}
\vspace{-2mm}
% \resizebox{.8\linewidth}{!}{
\begin{tabular}{lcc}
\toprule
 & \small (1) & \small (2)  \\
 & \small $\Delta$ Consumption & \small $\Delta$ Payment \\[.5em]
 \toprule
$\Delta$ Employed   &     -0.0005                   &      0.0074\textsuperscript{b}&     -0.0076\textsuperscript{c}\\
                    &    (0.0071)                   &    (0.0034)                   &    (0.0044)                   \\[0.1em]
$\Delta$ Hours worked&     -0.0012                   &      0.0018                   &     -0.0017                   \\
                    &    (0.0028)                   &    (0.0016)                   &    (0.0025)                   \\[0.1em]
N                   &     512,978                   &     554,228                   &     554,228                   \\

\bottomrule
\end{tabular}
\begin{tablenotes}
\footnotesize
\item All units are in PhP.  Quarterly fixed effects are included.  Standard errors are clustered by age-group, sub-municipality, and whether college educated (80 units).  \textsuperscript{c} p$<$0.10,\textsuperscript{b} p$<$0.05,\textsuperscript{a} p$<$0.01 
\end{tablenotes}
\end{threeparttable}
\end{table}

Column (1) of Table~\ref{table:lfsanalysis} finds a positive but statistically insignificant correlation between income and consumption.  Due to measurement error in income, the small point estimate of 0.0007 likely underestimates this correlation.  Measurement error in income is likely to be substantial both because only around 55 incomes are observed per quarter-bin in the Labor Force Survey and because bins are constructed from coarse demographic categories that may only loosely reflect household incomes.  Despite this potential measurement error, Column (2) finds a statistically significant correlation between bill payments and income that is six times larger than the correlation between income and consumption.  These results provide suggestive evidence of consumption smoothing where households flexibly adjust their payment patterns in response to income shocks without changing their water consumption.  



\subsection{Usage Before Permanent Disconnection}\label{appendix:permanentdc}

%%% DISCUSS AVERAGE LENGTH OF ACCOUNTS IN THE SAMPLE!!!

% Figure~\ref{figure:dc_permanent} plots average usage across households according to the number of months before these households permanently disconnect. Average consumption increases in the months leading up to permanent disconnection.  This increase is likely due to some households using water as if they faced a zero marginal price since they know that they will never pay their bills.  With an average bill of \input{tables/bill_all}PhP/month, leaving an outstanding balance of \input{tables/out_bal}PhP at permanent disconnection is equal to enjoying an average of around 18 months of free water consumption.

\begin{figure}[H]
\centering
\caption{Average Usage in Months Before Permanent Disconnection}\label{figure:dc_permanent}
\includegraphics[scale=.7]{tables/line_disc_graph.pdf} \\
{ \footnotesize Permanent disconnection is defined as households disconnecting and remaining disconnected for the at least the last two years of the sample.  Negative months indicate months leading up to permanent disconnection.  Months with zero usage are dropped because households may leave months with zero consumption before the company disconnects them. }
\end{figure}


\subsection{Income Coefficient of Variation}\label{appendix:cvcalc}

\begin{table}[H]
\centering
\caption{Income Coefficient of Variation Estimates}\label{table:cvcalc}
\vspace{-2mm}
\begin{tabular}{l*{1}{cc}}
\toprule
 & (1) & (2)  \\
  & Raw & Adjusted  \\ \midrule
All & \input{tables/cv} & \input{tables/cv_adj} \\
\\
By Income Tercile \\
T1 & \input{tables/cv_t1} & \input{tables/cv_adj_t1} \\ 
T2 & \input{tables/cv_t2} & \input{tables/cv_adj_t2} \\ 
T3 & \input{tables/cv_t3} & \input{tables/cv_adj_t3} \\ 
Demographic/Occupation Controls  & No & Yes \\
Households & \input{tables/cbms_hhs} & \input{tables/cbms_hhs} \\
Years & 2008, 2011 & 2008, 2011 \\
\bottomrule
\multicolumn{3}{l}{\footnotesize The Coefficient of Variation (CV) for each household (HH) is $ \frac{ 2 \mid y_{2011} \,\, - \,\, y_{2008} \mid  }{y_{2011} \,\,+\,\, y_{2008}}  $    } \\[-.3em]
\multicolumn{3}{l}{\footnotesize where $y$ is HH income. Estimates take the mean CV across HHs. } \\[-.5em]
\multicolumn{3}{l}{\footnotesize   Income terciles are computed by mean HH income.} \\[-.2em]
\multicolumn{3}{l}{\footnotesize   Adjusted CV is $ \frac{ 2 \mid \overline{y}_{2011} \,\, - \,\, \overline{y}_{2008} \mid  }{y_{2011} \,\,+\,\, y_{2008}}  $  where $\overline{\text{y}}$ is  residual income controlling for} \\[-.3em]
\multicolumn{3}{l}{\footnotesize  HH employment by skill-level, oldest HH age, HH size by education,  } \\[-.5em]
\multicolumn{3}{l}{\footnotesize year, and HH fixed effects.  } \\
\end{tabular}
\end{table}


% \subsection{Sampling of the Connection Survey}\label{appendix:pawssampling}

% Connection survey documentation indicates that surveyors used stratified sampling where the number of connections surveyed was proportional to the census population in each small administrative district (Barangay).  Surveyors randomly interviewed owners of connections until reaching these targets.  


% Although this approach ensures that connections are randomly sampled within areas, it may also lead to oversampling of areas with few connections.  To measure the extent of possible oversampling, the following regression expresses the share of surveyed connections in each Meter Reading Unit (MRU) --- the water company's smallest unit of geography including around 200 households per unit on average --- as a function of the demographics of households owning connections, $Z_{MRU,i}$, in each of these MRUs:

% To test this hypothesis, the following equation predicts the share of connections surveyed in each Meter Reading Unit (MRU) --- the smallest geographical unit used by the water provider including a couple hundred water connections on average --- according to demographics of households owning connections, $Z_{MRU,i}$, in each of these MRUs:
% \begin{align*}
% ShareSurveyed_{MRU} = \beta Z_{MRU,i} + \epsilon_{MRU,i}
% \end{align*}
% Table~\ref{table:pawssampling} provides the results for this regression.  Although many coefficients are statistically significant consistent with large sample sizes for this estimation, the magnitudes of the effects are small in economic terms across most measures.  For example, increasing the average number of households per connection by a standard deviation (0.65) results in a 0.0021 percentage point decrease in the probability of being surveyed; given a mean survey probability of 6 percentage points, this change represents only a 3.3 percent decrease.  Dwelling types are the strongest predictor of coverage, indicating that surveyors may have oversampled apartments possibly due to easier availability to survey.  Taken together, these results suggest that to the extent that non-random sampling may affect results, it will lead to an underestimation of the importance of sharing networks in Manila.

% \begin{table}[H]
% \centering
% \caption{ Predicting Share of Connections Surveyed with \\ Connection Owner Demographics }\label{table:pawssampling}
% \begin{tabular}{lc} \hline
 & (1) \\
VARIABLES & Percent Surveyed \\ \hline
 &  \\
Mean Consumption (m3) & 0.000286*** \\
 & (3.94e-05) \\
HHs per Connection & -0.00334*** \\
 & (0.000775) \\
HH Size & -0.000285 \\
 & (0.000190) \\
Age HoH & 0.000138*** \\
 & (2.27e-05) \\
Total Empl. & 4.04e-05 \\
 & (0.000326) \\
Low Skill Emp. & 0.00158* \\
 & (0.000909) \\
Apartment & 0.00704*** \\
 & (0.00157) \\
Single House & -0.0114*** \\
 & (0.00117) \\
Constant & 0.117*** \\
 & (0.00250) \\
 &  \\
Observations & 45,907 \\
R-squared & 0.018 \\
Mean Coverage & .06 \\
Std. Dev. Coverage & .06 \\
 Cluster MRU & Yes \\ \hline
\multicolumn{2}{c}{ Robust standard errors in parentheses} \\
\multicolumn{2}{c}{ *** p$<$0.01, ** p$<$0.05, * p$<$0.1} \\
\multicolumn{2}{c}{ 2,947 Meter Reading Units (MRUs)} \\
\end{tabular}

% \end{table}

\nocite{*}
\singlespacing
\setlength{\bibsep}{7pt}
\bibliographystyle{abbrvnat}
\bibliography{ref}



\end{document}


