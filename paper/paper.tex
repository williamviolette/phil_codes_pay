\documentclass[12pt]{article}

% TEMPLATE DEFAULT PACKAGES
\usepackage{amssymb,amsmath,amsfonts,eurosym,geometry,ulem,graphicx,color,setspace,sectsty,comment,natbib,pdflscape,array,adjustbox}

% ADDED PACKAGES FOR THIS MANUSCRIPT
\usepackage{palatino,newtxmath,multirow,titlesec,threeparttable,tabu,booktabs,titlesec,threeparttable,mathtools,bm,bbm,subcaption,pdflscape,tcolorbox,mathrsfs,float}
% endfloat,

%\usepackage{kbordermatrix}% http://www.hss.caltech.edu/~kcb/TeX/kbordermatrix.sty
%\usepackage{amsmath}% http://ctan.org/pkg/amsmath

\usepackage{afterpage}
\usepackage[hyphens]{url}
\usepackage[margin=1cm]{caption}

\usepackage[draft]{hyperref}
\newcommand{\tim}{$\,\times\,$}
% FIGURES & TABLES CAPTION STYLING
\captionsetup[figure]{labelfont={bf},name={Figure},labelsep=period}
\captionsetup[table]{labelfont={bf},name={Table},labelsep=period}

% SECTION TITLE SETTINGS
\titlelabel{\thetitle.\enskip}
\titleformat*{\section}{\large\bfseries}
\titleformat*{\subsection}{\normalsize\bfseries}

% COLUMN TYPES
\newcolumntype{L}[1]{>{\raggedright\let\newline\\\arraybackslash\hspace{0pt}}m{#1}}
\newcolumntype{C}{>{\centering\arraybackslash}p{5.2em}}
\newcolumntype{D}{>{\centering\arraybackslash}p{5em}}
\newcolumntype{R}[1]{>{\raggedleft\let\newline\\\arraybackslash\hspace{0pt}}m{#1}}


% MARGINS AND SPACING
\normalem
\geometry{left=1.1in,right=1.1in,top=1.0in,bottom=1.0in}
\setlength{\parskip}{2.5pt}

% SPECIAL CELL 
\newcommand{\specialcell}[2][c]{%
	\begin{tabular}[#1]{@{}l@{}}#2\end{tabular}}

% NO INDENT ON FOOTNOTES
\usepackage[hang,flushmargin]{footmisc}

\begin{document}

\begin{titlepage} 
\title{{Paying bills}\thanks{We are grateful to Andrew Foster}}
\author{\\[3em]
  William Violette\thanks{Federal Trade Commission, Washington, DC. E-mail: william.j.violette@gmail.com} \\
 \\ 
  }
\vspace{30mm}
\date{\vspace{5mm}This Version: \today}
\maketitle
\begin{abstract}

	paying bills

 %\\

%\vspace{0in}\\
%\textbf{Keywords:} traffic externalities; street livability; urban policy; housing market.\\
%\vspace{0in}\\
%\textbf{JEL Codes:} O18; H4; R2; R4.\\
\bigskip
\end{abstract}
\setcounter{page}{0}
\thispagestyle{empty}
\end{titlepage}
\pagebreak \newpage

\spacing{1}


\section{Pitch}

Why structure?
\begin{itemize}
    \item cool counterfactuals
        \begin{enumerate}
            \item full welfare of pre-paid meters (with revenue equaling counterfactual)
            \item how do interest rate changes affect water demand and delinquency? spillover effects!?
        \end{enumerate}
    \item measure consumption distortions from water loans [ shows up with first one.. ]
    \item estimate a revealed credit constraint parameter using utility data
        \begin{enumerate}
            \item show that they are substitutes!
        \end{enumerate}
\end{itemize}

This paper looks at whether utilities in developing countries provide an important source of credit to households by letting them not pay their bill?  And how do these benefits compare to the costs of delinquency?

Motivation : (1) around the world, there are restrictions on when utilities can disconnect people (health, low-income, elderly), motivated by providing a sense of insurance;  (2) Also, in developing countries, pre-paid metering where utilities essentially put fancy meters that only distribute water when people pay first.  these shut off any credit channel but eliminate delinquency

Context : In Manila, water bills make up about 3\% of income on average, jumps to 5 to 10\% for low-income folks.  I have data from a large water provider serving half the city of Manila.  On average, people make a payment in three out of four months and are about a month behind in payments.  For example, if you don't pay for three months, that's like taking a three-month loan of around 9\% of your income.  

People might be paying infrequently because they are credit-constrained and consumption smoothing; or it might just be a pain to pay their bills so they just avoid the hassle by paying infrequently (and they have plenty of other opportunities to smooth).  

I use disconnection threats to better isolate the role of credit constraints.  Workers will occasionally visit delinquent customers and say if you don't pay your bill soon (within 12 days on average), we'll disconnect you.  Only 23\% of people say that's ``enough time'' to pay the bill.  After the threats, payments spike to double the average bill, delinquency drops to zero, and consumption drops by around 25\%.  If it were simply a hassle, people would pay their bill and consumption wouldn't change (they could get easy credit from other places and pay the bill); but if they are credit constrained, they have to deviate from consumption smoothing to make the payments.  The next step is to use theory to see what this decrease in consumption implies about the short-term interest rate that households face.


\section{Introduction}

%% disconnection 

% https://liheapch.acf.hhs.gov/Disconnect/disconnect.htm
\subsection{Question} 
Are utilities providing an important source of credit to households by letting them not pay their bill?

\subsection{Motivation}
\begin{itemize}
\item Disconnection policies as insurance (in the US) 
	\begin{itemize}
		\item weather, health, low-income
		\item mandate that utilities amortize arrears
		\item Utilities even tolerate greater non-payment
	\end{itemize}
\item At the same time, pre-paid metering is growing like crazy in the developing world (Sources)
\end{itemize}

\subsection{Descriptives}
% How much credit could households get from their water bill?
\begin{itemize}
\item Avg Income: 22,000 PhP (488 USD)   [ Bill 3\% ]
\item Avg Savings: 4,300 PhP (96 USD) 
\item 20\% Income: 8,300 PhP (184 USD)  [ Bill 7.6\% ]
\item 20\% Savings: 330 PhP (7 USD)  
\item Avg Bill: 630 PhP (14 USD)
\item Make payments 75\% of months
\item Avg delinquency: 30 days
\item Avg Payment Amount: 830 PhP (18 USD)
\end{itemize}

But : might just be inconvenient to pay every month (but its really easy to pay bills in this context)

How can I ballpark this against the consumption smoothing literature?

\subsection{Approach}

Disconnection : Don't pay bill; come and threaten disconnection
\begin{itemize}
\item avg days to pay : 12 days (only 23\% say that's enough time)
\item if you (agree to?) pay, you are reconnected after 2 days
\item 30\% of connections are threatened with disconnection
\item (small percent actually disconnect)
\item Pay (+1) 800 PhP (+2) 300 PhP = total 1100 PhP (24 USD) [ about two water bills on average ]
\item Consumption drops by about 20\% for two months (there is a pre-trend which can be interpreted as positive demand shocks)
\end{itemize}

\noindent Theory :
\begin{itemize}
\item suddenly this source of credit is cut off (loan with uncertain payback date)
\item concave utility predicts that households would want to smooth consumption (could get another loan, then fund consumption in that period)
\item 
\end{itemize}


\section{Data}

To measure water consumption and payments, I partnered with a large water provider in Manila who shared administrative data on over 1.5 million water connections.  These data include water consumption and payment data as well as disconnection status.  

Describe notice of disconnection in more detail here???

used to measure water consumption and payment data comes from monthly administrative data from a large water provider in Metro Manila

The 2015 Family Income and Expenditure Survey provides household income measures, which help to calibrate the structural model.  

STICK WITH ONE HOUSEHOLD PER METER IN THIS PAPER!  EXPLAIN THIS IN THE DATA SECTION!  CONNECT TO FIES DATA AS WELL!!


\section{Credit through Unpaid Water Bills}

% \footnote{The company upgraded t in the first year of the sample.  Previously, the company mailed the bill to each household at the end of the month.} 
% Each month, households consume \input{tables/usage_all}m3 of piped water on average, which totals an average monthly bill of \input{tables/bill_all}PhP as shown in Table~\ref{table:descriptives_all}.  Given an average monthly income in Manila of \input{tables/y_avg}PhP and savings rate of \input{tables/save_avg}PhP,  monthly water bills reach a modest average of \input{tables/bill_inc_all}\unskip\% of income and \input{tables/bill_sav_all}\unskip\% of savings.  


\begin{table}[h!] % PUT IN DEMOGRAPHICS PLEASE ?!?!?
\centering
\caption{Mean Characteristics}\label{table:descriptives_all}
\vspace{-2mm}
\begin{tabular}{l*{1}{cccccc}}
\toprule
 & Mean & SD & Min & 25th & 75th & Max  \\
\midrule
 Usage (m3)  & 24.3  & 15.4  & 0.0  & 14.0  & 31.0  & 200.0  \\ 
 Bill  & 671  & 734  & -4,640  & 265  & 843  & 19,999  \\ 
 Unpaid Balance  & 1,206  & 3,316  & -9,965  & 0  & 1,044  & 79,904  \\ 
 Share of Months with Payment  & 0.71  & 0.45  & 0.00  & 0.00  & 1.00  & 1.00  \\ 
 Payment Size  & 901  & 1,067  & 0  & 313  & 1,070  & 49,688  \\ 
 Days Delinquent  & 56.2  & 113.7  & 0.0  & 0.0  & 61.0  & 750.0  \\ 
 Delinquency Visits per HH  & 0.40  & 0.70  & 0.00  & 0.00  & 1.00  & 6.00  \\ 
 Share of Months Disconnected  & 0.04  & 0.18  & 0.00  & 0.00  & 0.00  & 1.00  \\ 

\bottomrule \\ [-.8em]
\multicolumn{7}{c}{Total Households: \input{tables/total_hhs_all}  Obs. per Household: \input{tables/obs_per_hh_all} Total Obs.: 2,844,481}
\end{tabular}
\end{table}

Unpaid water bills may provide a reliable source of low-cost credit to households because (1) the company tolerates high rates of delinquency before disconnecting them from service and (2) the water company is prohibited from charging any interest on outstanding balances. 

At the end of each month, the water company sends meter readers who record monthly consumption for each connection and then use a mobile device to print and deliver the bill to the household in person.  The household is then expected to pay the bill by the end of the month.  Households have many options to pay their bills with \input{tables/center}\unskip\% using small payment centers (mall kiosks, gas stations, convenient stores, etc.), \input{tables/maynilad}\unskip\% paying at local water company offices, and \input{tables/atm}\unskip\% paying over phone, online, or via ATM kiosks.\footnote{Figures are tabulated from the connection survey sample.}  Despite easy payment mechanisms, households rarely pay their bills on time.  Table~\ref{table:descriptives_all} provides summary statistics on the usage, billing, and payment patterns of households.  On average, households are \input{tables/days_delinquent_all}days behind in their payments.  Households also make large, infrequent payments.  While the average bill is \input{tables/bill_all}PhP per month, payment sizes average \input{tables/pay_size_all}PhP and households make payments in only \input{tables/pay_freq}\unskip\% of months.  These payment patterns leave an average total outstanding balance of \input{tables/balance_all}PhP per month.  

Given average monthly household incomes of \input{tables/y_avg}PhP and savings rates of \input{tables/save_avg}PhP in Manila, unpaid water bills reach \input{tables/balance_inc}\unskip\% of income and \input{tables/balance_save}\unskip\% of savings on average each month.  For households at the 20th income percentile, unpaid water bills jump to \input{tables/balance_inc_20_all}\unskip\% of monthly income.  As a benchmark, \cite{cull2009microfinance} survey microfinance institutions throughout the developing world and find median yearly loan sizes expressed as a share of the 20th percentile of household income at 48\% for nongovernmental organizations (NGOs), 160\% for nonbank financial institutions, and 224\% for banks.  These descriptives suggest that a yearly loan from an NGO could be reached with around 5 months of unpaid water bills for households at the 20th income percentile.\footnote{Deal with income effects on water consumption!}  Moreover, microfinance loans charge high yearly interest ranging from 25\% for NGOs to 13\% for banks.  In comparison, unpaid water bills are interest-free, but households face some risk of service disconnection for delinquency.

After bills have remained unpaid for at least 60 days, the government regulator permits the water company to visit delinquent households to disconnect their water service.\footnote{Regulations also require the water company to issue written statements to households, notifying them that their connections will be disconnected in 7 days if their outstanding balances remain unpaid.  In practice, only \input{tables/disc_notice}\unskip\% of disconnected households report receiving advanced notice.  All percentages are calculated from \input{tables/disc_count}respondents who report being visited for disconnection in the past year out of \input{tables/paws_accounts}total respondents in the connection survey.}  Likely due to time and travel costs, delinquency visits are relatively rare in practice, occurring in only \input{tables/tcd_id_mean}\unskip\% of household-month observations.  This probability increases to \input{tables/tcd_id_ar_cond}\unskip\% for household-months over 60 days overdue.  Figure~\ref{figure:dc_hazard} plots the share of months with delinquency visits according to days overdue.  The risk of delinquency visits spikes at 61 days overdue before settling to around 2.5\% at higher levels of delinquency.  Appendix Table~\ref{table:tcd_predict} predicts delinquency visits with days overdue, outstanding balance, and household characteristics.  This exercise finds that days overdue and unpaid balances are strongly and independently correlated with visits while demographic indicators have weaker associations. % more to say here about magnitudes?!

When company staff conduct delinquency visits, households often negotiate for additional time to pay their outstanding balances.  \input{tables/days_pay_under_30}\unskip\% of households report agreeing to pay within 30 days and the average grace period is \input{tables/days_pay_average}days; however, only \input{tables/enough_time}\unskip\% of connections report having ``enough time'' to make their payments.  For households who fail to pay, disconnection typically involves workers from the water company placing a metal lock on the water meter stopping the flow.  In order to reconnect, households must pay a small, one-time fee of 200 PhP on top of settling any outstanding balances.  The water company is then required to restore service within 48 hours of receiving full payment for reconnection, which is confirmed in survey data.\footnote{Households report being reconnected \input{tables/days_rec_average}days after payment.}  


\begin{figure}
\centering
\caption{Share of Households that Receive a Delinquency Visit \\ depending on Days Delinquent in the Previous Month}\label{figure:dc_hazard}
\includegraphics[scale=.7]{tables/connected_visit_hazard_all.pdf}
\end{figure}




\begin{table}[h!]
\centering
\caption{Mean Characteristics by Delinquency Visit Status}\label{table:descriptives_3g}
\vspace{-2mm}
\begin{tabular}{l*{1}{ccc}}
\toprule
 & Never Visited & Stayers & Leavers  \\
% &   &         &        \\
\midrule
 Usage (m3)  & 27.1  & 30.0  & 31.5  \\ 
 Bill  & 782  & 931  & 1,062  \\ 
 Unpaid Balance  & 917  & 3,009  & 8,308  \\ 
 Share of Months with Payment  & 0.78  & 0.62  & 0.38  \\ 
 Payment Size  & 978  & 1,448  & 1,553  \\ 
 Days Delinquent  & 20.7  & 90.9  & 259.1  \\ 
 Delinquency Visits per HH  & 0.00  & 1.35  & 1.45  \\ 
 Months Disconnected  & 0.01  & 0.04  & 0.33  \\ 
 HH Size  & 5.0  & 5.4  & 5.4  \\ 
 Age of HoH  & 47.2  & 44.7  & 45.6  \\ 
 Low Skilled HoH  & 0.15  & 0.19  & 0.21  \\ 
 &  &  &  \\ 
 Total Households   & 30,393  & 11,856  & 3,849  \\ 
 Total Observations  & 1,876,658  & 731,935  & 235,888  \\ 

\bottomrule
\multicolumn{4}{l}{\scriptsize ``Never Visited'' includes HHs that never receive a delinquency visit.}\\  [-.5em]
\multicolumn{4}{l}{\scriptsize ``Stayers'' includes HHs with $\geq$1 visit and are connected for the last 6 months.}\\ [-.5em]
\multicolumn{4}{l}{\scriptsize ``Leavers'' includes HHs with $\geq$1 visit and are disconnected for $\geq$1 of the last 6 months.}\\ [-.5em]
\multicolumn{4}{l}{\scriptsize Bill, Unpaid Balance, and Payment Size are in PhP/month.} 
\end{tabular}
\end{table}


Table~\ref{table:descriptives_3g} provides mean characteristics according to whether and how households respond to delinquency visits over the course of the sample.  The first column, ``Never Visited,'' includes households that never receive a delinquency visit.  Since delinquency visits are relatively rare, the majority of observations fall into this category.  The second and third columns include households that receive at least one delinquency visit.   The second column, ``Stayers,'' also requires that households are connected for the final 6 months of the sample while the third column, ``Leavers,'' includes households that are disconnected for at least one of the final 6 months of the sample.  

Leavers are predominantly composed of households that permanently disconnect over the sample period, which likely occurs when households move out of their current residences.  These households often leave large outstanding balances that are almost never repaid due to difficulties tracking households across locations.  By incentivizing households to pay, frequent delinquency visits provide a strategy for the company to minimize this lost revenue.

Stayers include households that remain connected at the end of the sample despite receiving at least one delinquency visit over the duration.\footnote{Stayers also excludes \input{tables/pcd_hh}households that the company has flagged as ``permanently disconnected.''}  Compared to households that are never visited, stayers have much higher outstanding balances and days delinquent.  Their payments also occur less frequently but have larger average sizes.  Stayers spend 3\% of the sample period disconnected from service.  Until they are able to pay for reconnection, these households likely substitute to alternative water sources including sharing with neighbors, using from deepwells, or purchasing from local water vendors.\footnote{Table~\ref{table:descriptives_3g} indicates that even for households that are never visited, they are disconnected during 1\% of months.  These disconnections include (1) households that received a delinquency visit before the start of the sample (but later reconnected) and (2) households that notify the water company about their moving plans in advance and therefore, do not need a delinquency visit.}  Stayers also have slightly larger household sizes, younger heads of household, and greater incidence of low-skilled employment than never visited households.  These demographic patterns are consistent with lower-income households having greater difficulty paying their bills promptly.




\begin{figure*}[t!]
    \centering
    \vspace{2mm}
    \begin{subfigure}[b]{0.49\textwidth}
        \centering
        \caption[]{\small Share Disconnected}  
        \vspace{-1mm}
        \includegraphics[width=\textwidth,trim={.2cm .2cm .2cm 0cm}, clip=true]{tables/line_disconnection}
        \label{fig:line_disc}
    \end{subfigure}
    \hfill
    \begin{subfigure}[b]{0.49\textwidth}  
        \centering 
        \caption[]{\small Unpaid Balance}
        \vspace{-1mm}
        \includegraphics[width=\textwidth,trim={.2cm .2cm .2cm 0cm}, clip=true]{tables/line_bal}
        \label{fig:line_bal}
    \end{subfigure}
    \vskip 1mm \vskip 0pt
    \begin{subfigure}[b]{0.49\textwidth}
        \centering
        \caption[]{\small Days Overdue}
        \vspace{-1mm}
        \includegraphics[width=\textwidth,trim={.2cm .2cm .2cm 0cm}, clip=true]{tables/line_ar}
        \label{fig:line_ar}
    \end{subfigure}
    \hfill
    \begin{subfigure}[b]{0.49\textwidth}  
        \centering
        \caption[]{\small Monthly Payment}  
        \vspace{-1mm}
        \includegraphics[width=\textwidth,trim={.2cm .2cm .2cm 0cm}, clip=true]{tables/line_pay}
        \label{fig:line_pay}
    \end{subfigure}
    \vskip 1mm \vskip 0pt
    \begin{subfigure}[b]{.49\textwidth}  
        \centering
        \caption[]{\small Average Consumption} 
        \vspace{-1mm}
        \includegraphics[width=\textwidth,trim={.2cm .2cm .2cm 0cm}, clip=true]{tables/line_c}  
        \label{fig:line_c}
    \end{subfigure}
    \hfill \hspace{.02\textwidth}
    \begin{minipage}{0.47\textwidth}   
    \vspace{-6cm}
    \caption[]
    {\small Mean Outcomes around First Delinquency Visit  \\  \\  Figures include all households with delinquency visits.  Stayers include only households who are connected for the last 6 months of the sample.  Mean monthly payments include zeros when no payment is made. } \label{fig:line_graphs}
    \end{minipage}
\end{figure*} 


To investigate the timing of household responses to delinquency visits, Figure~\ref{fig:line_graphs} plots monthly mean outcomes in months relative to the first delinquency visit for each household.  Outcomes are plotted separately for all households that experience delinquency visits as well as stayer households who experience visits and also remain connected for the last 6 months of the sample.  In the months just preceding a visit, monthly payments (Figure~\ref{fig:line_pay}) decrease suddenly, leading to corresponding increases in unpaid balances (Figure~\ref{fig:line_bal}) and days overdue (Figure~\ref{fig:line_ar}).  Average consumption (Figure~\ref{fig:line_c}) increases slightly as well.  Prior to disconnection, stayers follow similar patterns but with lower levels of delinquency than the population average.

Immediately following the first delinquency visit, monthly payments spike as many households pay their full outstanding balances to prevent any disconnection.  Despite these payments, the average share of households disconnected (Figure~\ref{fig:line_disc}) also spikes to around \input{tables/am_2}\unskip\% before decreasing, as some households pay for reconnection, and stabilizing at \input{tables/am_20}\unskip\%, which is likely composed of households who have permanently disconnected.  Stayer households pay more and disconnect less than the population average following a visit.  Disconnection rates for stayers spike to \input{tables/am6_2}\unskip\% before declining to around \input{tables/am6_20}\unskip\% two years later.\footnote{Although stayers are connected for the last 6 months of the sample by definition, disconnection rates do not decline to zero since some stayers experience multiple disconnection spells.}  The decline among stayers accounts for much of the average decline in disconnection rates across the sample.  Although many stayers quickly reconnect within 6 months, reconnections continue accumulating up to one year after a visit.  

This descriptive evidence indicates that many households choose to disconnect for long periods before finally paying for reconnection.  This behavior is consistent with households facing credit-constraints, which prevent them from taking a low-cost loan to fund immediate reconnection.  Instead, households substitute to lower-quality alternative sources of water until they can save enough income for reconnection.  Credit constraints would additionally predict that the most delinquent households at the time of the visit may be the most likely to disconnect following a visit.  The data support this hypothesis, finding that the share disconnected two months after a visit is \input{tables/amar_2}\unskip\% for stayers that are over 90 days delinquent at the time of the visit.  The corresponding share for stayers under 90 days delinquent shrinks to \input{tables/amar_2a}\unskip\%.

Another hypothesis may be that households choose to stop paying when they leave their homes for vacation or overseas work.  This mechanism would likely predict a large decrease in water consumption in the months preceding delinquency visits.  Instead, consumption rises consistently between 24 and 2 months before a visit.  Consumption then drops one month before, during, and after a visit, before quickly returning to an average level.  This dip in consumption likely corresponds to connections that are disconnected for less than a month before being reconnected and having their consumption recorded for that month.  Overall, this variation is small given a mean consumption for stayers of \input{tables/c_avg}m3 with a standard deviation of \input{tables/c_std}\unskip.  Moreover, delinquency visits are relatively rare, which may make their exact timing difficult for households to predict.  











%The administrative data indicate that \input{tables/share_pay_3_months}\unskip\% resolve their outstanding balances following a notice.

% At this time, households report paying 

%In practice, the water company often tolerates delinquency well over 60 days, which is consistent with (1) costs of sending and following up on notifications and (2) household demand for credit from the water company.  Households receive notice in \input{tables/disc_prob}\unskip\% of months.  

% This rate increases to \input{tables/disc_prob_60}\unskip\% for households with at least 60 days of delinquency.  

% Figure~\ref{figure:dc_hazard} graphs the probability of receiving a delinquency notice conditional on the level of delinquency for each account.  The water company appears to carefully follow government regulations preventing any disconnection for delinquency below 60 days.  Once households reach 61 days of delinquency, they face some probability of receiving a notice, which triples moving above 90 days unpaid balance.  Conditional on having greater than 90 days of delinquency 

%  In practice, the water company tolerates \input{tables/ar_dc}days of delinquency on average before issuing a notice.  The level of delinquency tolerated before issuing notices also appears uncorrelated with (list measures/demographics...).  ( SHOW REGRESSION IN APPENDIX )  

% Upon receiving a notice of disconnection, connection owners then negotiate for time to pay their outstanding balances.  In \input{tables/days_pay_under_30}\unskip\% of notices, owners are agree to pay within 30 days, making an average grace period of \input{tables/days_pay_average}days according to the water connection survey.  \input{tables/enough_time}\unskip\% of connections report having ``enough time'' to make their payments.  

% %The administrative data indicate that \input{tables/share_pay_3_months}\unskip\% resolve their outstanding balances following a notice.

% Disconnection typically involves workers from the water company placing a metal lock on the water meter stopping the flow.  Once disconnected, connection owners are charged a small, one-time fee of 200 PhP to restore their water service on top of settling any outstanding balances.  The water company is then required to restore service within 48 hours of receiving full payment for reconnection.

% \begin{itemize}
% \item No notice: great access to credit, or just not enough time to observe them hitting a
% \item Average notice rate!?!
% \end{itemize}
%\footnote{To account for potential mismeasurement in the administrative data, this measure allows connection owners to resolve their outstanding balances within 60 days of receiving notice.}
% In practice, \input{tables/dc_note}\unskip\% of owners report receiving advanced notice.  

\pagebreak

\section{Model}

Intro:

\subsection{Setting up the Household's Problem}

Time horizon etc.  Ignore disconnecting households.. 


Households maximize their expected utility given by the following equation
\begin{align}\label{eq:u}
E_t \Big[ \sum_{\tau = t}^{t-\tau} (1+\delta)^{t-\tau} u(w_{\tau},x_{\tau})   \Big]
\end{align}
where utility in each period $t$ is expressed as an increasing and concave function of water consumption, $w_{\tau}$, and consumption of all other goods, which are collapsed into a single term, $x_{\tau}$.  Households have a rate of time preference $\delta>0$.  In each period, households face a budget constraint as follows
\begin{align}\label{eq:bc}
x_t \, + \, p(w_t) w_t \, &= \, y_t \, - D_{t+1} f  \, + \, S_t \, - \, S_{t+1}
% y_t &=
% \begin{cases}
% y_H \text{ if } s = H \\
% y_L \text{ if } s = L
% \end{cases} 
\end{align}
where $p(w_t)$ captures the price per unit of water and may change with water use according to the tariff structure.  The price of other goods, $x_t$, is normalized to one.  $y_t$ represents household income each period which takes a value of $(1+\theta)\bar{y}$ with probability $\pi \in (0,1)$ and a value of $(1-\theta)\bar{y}$ with probability $(1-\pi)$ where $\theta  \in (0,1)$.  At the beginning of each period, households can choose whether to temporarily disconnect from water service $D_{t+1}=1$ or remain connected $D_{t+1}=0$.  If households disconnect, they pay a fixed cost $f$.  Since disconnected households are likely to share with neighbors or purchase from vendors who resell piped water, these households are assumed to face the same price schedule as connected households.  Disconnecting also affects each household's ability to use an unpaid water bill as an asset.  $S_t$ captures assets chosen in the previous period to be consumed in period $t$ while $S_{t+1}$ captures assets set aside in period $t$ for consumption in $t+1$.  Total assets in each period can be expressed in terms of two types, $A_t$ and $B_t$ given by the following expression
\begin{align}
S_t &= A_t + B_t \\
S_{t+1} &=  \dfrac{A_{t+1}}{1+r^{a}_{t}}  + I_t \dfrac{B_{t+1}}{1+r^{b}_{t}} 
\end{align}
$A_t$ captures standard assets for borrowing and saving.  The real interest rate $r_a$ is assumed to take a value of $r_l$ when households are saving ($A_{t+1} \leq 0$) and a value of $r_h$ when households are borrowing ($A_{t+1} > 0$).  This wedge in interest rates is intended to capture an institutional context with underdeveloped financial markets where households face high borrowing costs from moneylenders as well as high transaction costs in loaning out their own savings.  Households are unconstrained in their amount of borrowing or saving with this asset.  % possibly cite (\cite{karlan2009expanding}) here?

$B_t$ captures borrowing through unpaid water bills.  Households face a real interest rate $r_b$ for borrowing from water bills.  To prevent arbitrage cases where households borrow infinitely from water bills to invest in standard assets, $r_b$ is assumed to equal $r_h$ when households are saving through standard assets ($A_t>0$), and is equal to $r_w$ otherwise.  Since the water company is not allowed to charge interest on outstanding balances, $r_w$ is equal to zero in this setting.%This assumption is also consistent with some transaction costs in switching between asset types.

$I_t$ determines when households are able to borrow from their unpaid water bills and takes the following form
\begin{align}
I_t = D_{t+1} + (1-c_t) (1-D_t) (1-D_{t+1})
\end{align} %I_t = (D_{t+1} + c_t (1-D_{t+1})(1-D_t)) 
where $c_t$ indicates whether a household receives a delinquency visit, which occurs with probability $\lambda \in (0,1)$ in each period and $D_{t+1}$ indicates the household's disconnection choice given their disconnection status $D_t$.  If households receive a delinquency visit ($c_t=1$) and they choose to remain connected ($D_{t+1}=0$), then they must pay their full outstanding balance to remain connected, which means that households cannot continue borrowing against their unpaid water bill this period.  Similarly, if disconnected households ($D_t=1$) want to reconnect  ($D_{t+1}=0$), then they also need to pay their outstanding balance.  Choosing to disconnect ($D_{t+1}=1$) allows households to avoid paying their unpaid balances until they choose to reconnect.  

Given that households are able to borrow, the amount that they can borrow through unpaid bills is constrained as follows
\begin{align}\label{eq:borrowconstraint}
&B_t -  p(w_t) w_t (1-D_{t+1})(1+r_b) \leq B_{t+1} \leq 0 
\end{align}
By leaving bills unpaid, households are able to borrow up to their existing outstanding balance, $B_t$, plus their total bill from water consumption, $p(w_t) w_t$, given that they stay connected to service, $D_{t+1}=0$.  If households choose to disconnect, then they can maintain their existing outstanding balance, but cannot add to this balance through unpaid water use.  To capture the fact that water bills are almost never overpaid, households are assumed to be unable to save with this asset.

This framework creates four possible states depending on whether income, $y_t$, is high or low and whether households receive a delinquency visit given by the indicator variable, $c_t$.  Assuming that the probabilities of reaching each state are uncorrelated with each other yields the following transition matrix between states:
\begin{align*}
T_{t,t+1} = \bordermatrix{\text{} & (1+\theta)\bar{y}\,,\,\text{no visit} & (1-\theta)\bar{y}\,,\,\text{no visit} & (1+\theta)\bar{y}\,,\,\text{visit} & (1-\theta)\bar{y}\,,\,\text{visit} \cr
                (1+\theta)\bar{y}\,,\,\text{no visit} & \pi (1-\lambda) & (1-\pi) (1-\lambda) & \pi \lambda & (1-\pi) \lambda  \cr
                (1-\theta)\bar{y}\,,\,\text{no visit} & \pi (1-\lambda) & (1-\pi) (1-\lambda) & \pi \lambda & (1-\pi) \lambda  \cr
                (1+\theta)\bar{y}\,,\,\text{visit}    & \pi (1-\lambda) & (1-\pi) (1-\lambda) & \pi \lambda & (1-\pi) \lambda  \cr
                (1-\theta)\bar{y}\,,\,\text{visit}  & \pi (1-\lambda) & (1-\pi) (1-\lambda) & \pi \lambda &  (1-\pi) \lambda  }
\end{align*}


\subsection{Solving the Model}

The problem can be reformulated using a recursive value function approach where the household solves for a function, $V(X_t,z_t)$, that maximizes both current utility and the expected continuation value given current asset-levels and disconnection status captured by $X_t$ as well as current states given by $z_t$ below
\begin{align}\label{eq:valmax}
\begin{split}
V(X_t,z_t) \,=\, &max_{x_t,w_t}  \,\,\, u(x_t,w_t) \,+\, (1+\delta)^{-1} E \Big[\, V(X_{t+1}|z_{t})\,\Big| z_{t+1}, T_{t,t+1} \Big]
\\
s.t.& \\
x_t & \, + \, p(w_t) w_t \, = \, y_t \, - D_{t+1} f  \, + \, S_t \, - \, S_{t+1} \\
-p&(w_t) w_t (1-D_{t+1}) \leq \frac{B_{t+1}-B_t}{(1+r_b)} \leq 0  \\
X_t &= [A_{t},B_{t},D_{t}] \\
z_t &= [y_t,c_t]
\end{split} 
\end{align}
This problem can be divided into two steps: (1) maximizing utility within a given period by choosing $w_t$ and $x_t$ holding asset levels fixed, and (2) choosing asset values to maximize utility over time.

Each period, households face a standard utility maximization problem except in cases where they need to overconsume water to raise enough revenue through unpaid water bills to meet their borrowing goals.  Formally, this situation occurs when equation (\ref{eq:borrowconstraint}) is binding.  To solve this problem, let $L = \frac{B_{t+1} - B_t }{1+r^{b}_{t}}$ indicate the amount of revenue needed to be raised through water consumption to fund borrowing levels.  Likewise, let $Y = y_t  - D_{t+1} f   +  A_t + B_t  -  \frac{A_{t+1}}{1+r^{a}_{t}} - I_t \frac{B_t}{1+r^{b}_{t}}$ equal all other income each period.  Also, let utility take a Cobb-Douglas shape in water and all other goods with preference parameter, $\alpha \in (0,1)$.  Cobb-Douglas preferences assume that households spend a constant share of their income on water and that the price-elasticity of demand for water is equal to one, which is close to recent estimates in the literature.\footnote{\cite{wjv} finds an average price elasticity of 0.84 in this setting while \cite{szabo2015value} finds an average price elasticity of 0.98 in South Africa.}  The Cobb-Douglas utility function is also assumed to take a log-log form, which determines how households smooth income over time.\footnote{CITE SOME LITERATURE HERE?!}

The price of water is parameterized as a linear function of water use, $p(w) = p_1 + p_2w$, to approximate the increasing block tariff present in Manila.  Suppressing most time subscripts for ease of exposition, the household maximization problem takes the following form within each period
\begin{align}\label{eq:hhmax}
\begin{split}
&max_{w_t,x_t} \,\,\,\, \alpha\, log(w) \, + \, (1-\alpha)\,log(x) \\
s.t.& \\
\,\,\,\, &(p_1 + p_2 w)\,w + x  =  Y - L \\
&(p_1 + p_2 w)\,w (1-D_{t+1}) \leq L  
\end{split}
\end{align}
Optimal consumption takes a piecewise form depending on whether households have to use more water to satisfy demand for borrowing each period
\begin{align}\label{eq:ws}
\begin{split}
w^{*} &= 
\begin{cases}
                                                                                                                                                                                                                                                                                                                                                             \frac{p_{1}-\sqrt{{p_{1}}^2-8\,L\,\alpha \,p_{2}+8\,Y\,\alpha \,p_{2}+4\,L\,\alpha ^2\,p_{2}-4\,Y\,\alpha ^2\,p_{2}}}{2\,p_{2}\,\left(\alpha -2\right)}
 &\text{ if } L \geq \widehat{L} \\
                                                                                                                                                                                                                                                                                                                                                                                                                                                                -\frac{p_{1}-\sqrt{{p_{1}}^2-4\,L\,p_{2}}}{2\,p_{2}}
 &\text{ if } L < \widehat{L}
\end{cases} \\
&\widehat{L} =                                                                                                                                                                                                                                                                                                                                 \frac{Y}{2\,\left(\alpha -1\right)}+\frac{\frac{Y\,p_{2}}{2}+\frac{{p_{1}}^2}{8}-\frac{p_{1}\,\sqrt{{p_{1}}^2-\alpha \,{p_{1}}^2+8\,Y\,\alpha \,p_{2}}}{8\,\sqrt{1-\alpha }}}{p_{2}}

\end{split}
\end{align}
where $\widehat{L}$ captures the point at which revenue demanded for borrowing is exactly generated by the household's optimal consumption.  When borrowing demand outpaces revenue from optimal consumption (ie. $L<\widehat{L}$), then households must deviate from their optimal consumption choice and instead consume enough water to exactly satisfy borrowing demand.  This overconsumption captures a possibly important inefficiency associated with using unpaid water bills as a source of credit.  Combining optimal consumption in equation (\ref{eq:ws}) with the budget constraint in equation (\ref{eq:bc}) and the utility function in equation (\ref{eq:hhmax}) provides an indirect utility function as a function of prices, income, and assets.  The full indirect utility function is given by equation (\ref{eq:vstar}) in Appendix~\ref{appendix:indirectutil}.

Given indirect utility in each period and a discount rate that is greater than the interest rates, $\delta>max \{r_a,r_b\}$, households solve for a value function in (\ref{eq:valmax}) that maximizes utility by mapping any combination of asset levels into period $t$ into future asset levels in period $t+1$.



% Above $\widehat{L}$, the household must deviate from its optimal consumption choice to meet its demand for borrowing. 


% \begin{align*}
% &max_{w_t,x_t} \,\,\,\, w_t^{\alpha} x_t^{1-\alpha} \\
% s.t.& \\
% \,\,\,\, &(p_1 + p_2 w_t)\,w_t + x_t  =  Y_t - L_t \\
% \text{and}& \\[-.5em] 
% &(p_1 + p_2 w_t)\,w_t (1-D_{t+1}) \leq \dfrac{B_{t+1} - B_t}{1+r_b} \\
% \text{where}& \\[-.5em] 
% &Y_t = y_t  - D_{t+1} f   +  A_t + B_t  -  \dfrac{A_{t+1}}{1+r^{a}_{t}}  + I_t \dfrac{B_t}{1+r^{b}_{t}}    \\
% &L_t = \dfrac{B_{t+1} - B_t }{1+r^{b}_{t}} 
% \end{align*}




% $D_{t}$ indicates the disconnection status inherited from the previous period and .

% \begin{align}
% I_t = (D_{t+1} + (1-c_t) (1-D_{t+1})(1-D_t)) 
% \end{align}


% \begin{align*}
% &B_t - c_t (1-D_t) (1-D_{t+1}) p(w_t) w_t \leq B_{t+1} \leq 0 \\
% &D_t \in \{0,1\} \\
% &A_t \in \mathbb{R} 
% \end{align*}

% &A_{t} = A_{t+1} \\
% &B_{t} = B_{t+1} \\





\section{Model Primitives}    % https://www.cgdev.org/blog/compartamos-context


The estimation allows households to choose over \input{tables/par_nA}values of the standard asset, $A_t$, evenly spaced across a normal distribution with a standard deviation of \input{tables/par_sigA}\unskip, a minimum of \input{tables/par_Amin}\unskip, and a maximum of \input{tables/par_Amax}\unskip.  Households can also choose how much to borrow from unpaid water bills, $B_t$, over \input{tables/par_nB}values evenly spaced across a normal distribution truncated above at 0 with a standard deviation of \input{tables/par_sigB} and a minimum of \input{tables/par_Bmin}\unskip.  The additional choice of whether to stay connected each period given by $D_t$ brings the total possible number of asset combinations to \input{tables/par_totalsize}\unskip.

To generate simulated moments, the estimator creates a random \input{tables/par_n_iter}month chain of states and calculates the household's predicted asset and consumption choices (assuming asset levels of zero to start).


\begin{table}[H]
\centering
\caption{Calibrated and Assumed Parameters}\label{table:calibratedparam}
\vspace{-2mm}
\resizebox{\columnwidth}{!}{%
\begin{tabular}{l*{1}{ccl}}
\toprule
%Parameter  &   &  Value & Source \\
%\midrule
Savings Rate & $r_l$ & \input{tables/irate}\unskip\%  & {\footnotesize Philippines data from World Bank Databank (2010-2015) } \\
Water Rate & $r_w$ & 0\% & {\footnotesize Regulators prevent charging interest } \\
Discount Rate & $\delta$ & 2\% & {\footnotesize Mean of structural estimates from 
$\text{literature}^{\dagger}$} \\
Tariff    & $(p_1 + p_2 w)$ & $(\input{tables/p_int} + \input{tables/p_slope}w)$ & {\footnotesize Estimated price by water usage  (See Appendix~\ref{appendix:tariff} for details) } \\
Mean Inc. (PhP) & $\bar{y}$ & \input{tables/y_avg} & {\footnotesize Family Income Expenditure Survey (2015) for Manila} \\
High Inc. Risk& $\pi$ & 50\% & {\footnotesize Assumed to ensure symmetric income shocks} \\
Visit Risk & $\lambda$ & \input{tables/prob_caught}\unskip\% & {\footnotesize \% of months with a visit among stayers with $>$31 days overdue} \\
\bottomrule
\multicolumn{4}{l}{\scriptsize All measures are monthly.  Annual rates are converted to monthly rates as follows: Monthly Rate = $(1+\text{Annual Rate})^{1/12}-1$} \\[-.5em]
\multicolumn{4}{l}{\scriptsize $\text{}^{\dagger}$ See \cite{andreoni2012estimating}, \cite{laibson2007estimating}, and \cite{gourinchas2002consumption} for structural $\delta$ estimates.}
\end{tabular}
}
\end{table}




\begin{table}[H]
\centering
\caption{Parameters to be Estimated}\label{table:calibratedparam}
\vspace{-2mm}
%\resizebox{\columnwidth}{!}{%
\begin{tabular}{l*{1}{cl}}
\toprule
Parameters  &   & Main Identifying Moments  \\
\midrule
Water Preference (Cobb-Douglas) & $\alpha$ & Mean Usage \\[.1em]
Income Shock Magnitude & $\theta$ & Mean Outstanding Balance  \\[.1em]
Fixed Cost of being Disconnected  & $f$ &  \% Disconnected 1-4 months post visit \\[.1em]
Borrowing Rate from Standard Assets & $r_h$ & \% Disconnected 1-4 months post visit \\
 & & given 90+ days overdue when visited \\
\bottomrule
%\multicolumn{3}{l}{\scriptsize All rates are monthly.  Annual rates are converted to monthly as follows: Monthly Rate = $(1+\text{Annual Rate})^{1/12}-1$} 
\end{tabular}
%}
\end{table}





%%% Karlan and Zinman

% Informal credit markets
% and serial borrowing from moneylenders charging 20% per month or more is common (e.g., more than
% 30% of our sample reported borrowing from moneylenders during the past year). Trade credit is quite
% uncommon. There are several microlenders operating in Metro Manila, but most MFIs operate on a
% small scale (as noted above) and charge high rates (see below).


\section{Results}

\begin{table}[h!]
\centering
\caption{Estimates}\label{table:estimates}
\vspace{-2mm}
%\resizebox{\columnwidth}{!}{%
\begin{tabular}{l*{1}{cc}}
\toprule
Parameters  &   & Estimates \\
\midrule
Water Preference (Cobb-Douglas) & $\alpha$ & \input{tables/est_alpha} \\
 &  & (\input{tables/est_sd_alpha}\unskip) \\[.4em]
Income Shock Magnitude & $\theta$ & \input{tables/est_theta} \\
 &  & (\input{tables/est_sd_theta}\unskip) \\[.4em]
Fixed Cost of being Disconnected (PhP) & $f$ &  \input{tables/est_fc} \\
 &  &  (\input{tables/est_sd_fc}\unskip) \\[.4em]
Borrowing Rate from Standard Assets & $r_h$ & \input{tables/est_irate} \\
 &  & (\input{tables/est_sd_irate}\unskip) \\[.8em]
Households & & \input{tables/est_hhs} \\
Household-Months & & 2,118,861 \\
\bottomrule
\multicolumn{3}{l}{\scriptsize Standard errors in parentheses are bootstrapped with \input{tables/breps}repetitions at the household-level.} 
\end{tabular}
%}
\end{table}



\cite{karlan2009expanding} find money lenders regularly charge at least 20\% per month for credit.  \cite{gine2014group} offer small monthly loans of 1,000 PhP at 2.5\% monthly interest.





\begin{table}[H]
\centering
\caption{Model Fit}\label{table:fit}

\begin{tabular}{l*{1}{cc}}
\toprule
 Moments & Data  & Predicted \\[.5em]
Used in Estimation & & \\
\midrule
Mean Usage (m3) &26.20&26.58\\
Mean Outstanding Balance (PhP) &2415.8&2450.0\\
[.5em]
\% Disconnected Post-Visit  & & \\[.2em]
1 month &0.13&0.12\\
2 months &0.14&0.10\\
3 months  &0.12&0.06\\
4 months &0.10&0.05\\[.8em]
\% Disconnected Post-Visit & & \\
given 90+ days overdue when visited & & \\[.2em]
1 month &0.30&0.28\\
2 months &0.32&0.23\\
3 months &0.26&0.13\\
4 months &0.23&0.11\\
& & \\
Out of Sample & & \\
\midrule
SD Usage &12.3&2.4\\
SD Outstanding Balance  &3588.7&2634.7\\
Corr. Usage and Out. Bal. &0.31&-0.02\\

\bottomrule
%\multicolumn{3}{l}{\scriptsize Standard errors in parentheses are bootstrapped with \input{tables/breps}repetitions at the household-level.} 
\end{tabular}

%\begin{tabular}{lcc}
& Data & Estimated \\
Mean Usage (m3) &24.9&24.6\\
SD Usage &11.1&2.3\\
Mean Water Debt (PhP) &1205&1412\\
SD Water Debt (PhP) &1258&1955\\
Corr. Usage and Water Debt &0.35&0.05\\
Mean Disc. for 1 month &0.061&0.073\\
Mean Disc. for 2 months &0.061&0.047\\
Mean Disc. for 3 months &0.048&0.030\\
Mean Disc. for 4 months &0.040&0.017\\
\end{tabular} 

\end{table}

\begin{figure}[H]
\centering
\caption{100 Months of Simulated Data}\label{figure:deaton}
\includegraphics[scale=1.1]{tables/new_deaton_graph.pdf}
\end{figure}


\begin{table}[H]
\centering
\caption{Counterfactuals}\label{table:counter}
\begin{tabular}{lccc}
& Current & No Water Credit & No Water Credit and \\
&         &                  & Revenue Neutral \\
Water Credit Interest Rate &0.0&1.0&1.0\\
Mean Usage (m3) &24.6&23.2&25.1\\
Compensating Variation  & &-42.7&8.0\\
Delinquency Savings  &20.8&20.8&0.0\\
Price Intercept  &16.2&16.2&13.3\\
\end{tabular} 

\end{table}









\pagebreak

\section{Appendix}


\subsection{Predicting Delinquency Visits}

\begin{table}[H]
\small
\centering
\caption{ Linear Probability of Receiving a Delinquency Visit }\label{table:tcd_predict}
\vspace{-2mm}
\begin{tabular}{lCCC}
\toprule
& \small (1) & \small (2) & \small (3)  \\
\midrule 
Usage t-1           &  -0.0000150\textsuperscript{a}&  -0.0000186\textsuperscript{a}&  -0.0000225\textsuperscript{a}\\
                    & (0.0000052)                   & (0.0000052)                   & (0.0000079)                   \\[0.5em]
Days Delinquent t-1 &   0.0000524\textsuperscript{a}&   0.0000581\textsuperscript{a}&   0.0000499\textsuperscript{a}\\
                    & (0.0000016)                   & (0.0000016)                   & (0.0000020)                   \\[0.5em]
Unpaid Balance t-1  &   0.0000009\textsuperscript{a}&   0.0000009\textsuperscript{a}&   0.0000011\textsuperscript{a}\\
                    & (0.0000001)                   & (0.0000001)                   & (0.0000001)                   \\[0.5em]
Single House        &  -0.0001480                   &  -0.0001468                   &                               \\
                    & (0.0002032)                   & (0.0002050)                   &                               \\[0.5em]
Apartment           &  -0.0004173\textsuperscript{b}&  -0.0005555\textsuperscript{a}&                               \\
                    & (0.0001861)                   & (0.0001869)                   &                               \\[0.5em]
Age of HoH          &  -0.0000398\textsuperscript{a}&  -0.0000373\textsuperscript{a}&                               \\
                    & (0.0000040)                   & (0.0000040)                   &                               \\[0.5em]
HoH Low Skill Empl. &   0.0005415\textsuperscript{a}&   0.0005333\textsuperscript{a}&                               \\
                    & (0.0001827)                   & (0.0001830)                   &                               \\[0.5em]
HH Size             &   0.0003462\textsuperscript{a}&   0.0003482\textsuperscript{a}&                               \\
                    & (0.0000363)                   & (0.0000364)                   &                               \\[0.5em]
Employed HH Members &  -0.0002883\textsuperscript{a}&  -0.0002957\textsuperscript{a}&                               \\
                    & (0.0000573)                   & (0.0000574)                   &                               \\[0.5em]
Location            &                               &  \checkmark                   &                               \\
Year \tim Month \textsc{FE}&                               &  \checkmark                   &  \checkmark                   \\
Household \textsc{FE}&                               &                               &  \checkmark                   \\
N                   &   1,951,543                   &   1,948,783                   &   1,951,543                   \\
Mean Visits Per Month&      0.0072                   &      0.0072                   &      0.0072                   \\

\bottomrule
\multicolumn{4}{l}{\footnotesize Std. errors clustered at the HH-level. \textsuperscript{c} p$<$0.10,\textsuperscript{b} p$<$0.05,\textsuperscript{a} p$<$0.01 }
\end{tabular}
\end{table}


\subsection{Sample Construction}



\begin{table}[H]
\centering
\caption{Sample Construction}\label{table:sampleconstruction}
\vspace{-2mm}
\begin{tabular}{l*{1}{cc}}
\toprule
 & Observations & Observations Removed  \\
\midrule
Initial sample   &         3,343,644  &     \\
Keep residential connections (excluding commercial)   &            &    \input{tables/class_drop}  \\
Keep connections with payment records &  &   \input{tables/pay_drop}   \\
Keep months with consumption under 200 m3    &            &  \input{tables/chigh_drop}     \\
Keep bills $>$ -5,000 PhP and $<$ 80,000 PhP  &  & \input{tables/amount_drop}  \\
Keep unpaid bills $>$ -5,000 PhP and $<$ 80,000 PhP &             &     \input{tables/bal_drop}  \\
Keep payments $>$ -80,000 PhP and $<$ 80,000 PhP    &            &    \input{tables/paylh_drop}   \\
Keep connections with over 30 months of records   &            &   \input{tables/month_drop}    \\
Keep connections serving a single household   &            &   \input{tables/SHH_drop}    \\
Drop connections that are disconnected for final yr.  &            &   \input{tables/dc_drop}    \\
Final sample & 2,073,884 & \\
\bottomrule
\end{tabular}
\end{table}

\subsection{Indirect Utility Function}\label{appendix:indirectutil}

\begin{align}\label{eq:vstar}
\begin{split}
v^{*} &= 
\begin{cases}
\alpha \,\ln(\frac{p_{1}-\sqrt{{p_{1}}^2-8\,L\,\alpha \,p_{2}+8\,Y\,\alpha \,p_{2}+4\,L\,\alpha ^2\,p_{2}-4\,Y\,\alpha ^2\,p_{2}}}{2\,p_{2}\,(\alpha -2)})- \\ \ln(\frac{(\alpha -1)\,(8\,L-8\,Y-4\,L\,\alpha +4\,Y\,\alpha )}{2\,{(\alpha -2)}^2}+ \\
\frac{(p_{1}\,\sqrt{{p_{1}}^2-8\,L\,\alpha \,p_{2}+8\,Y\,\alpha \,p_{2}+4\,L\,\alpha ^2\,p_{2}-4\,Y\,\alpha ^2\,p_{2}}-{p_{1}}^2)\,(\alpha -1)}{2\,p_{2}\,{(\alpha -2)}^2})\,(\alpha -1)\\
 &\text{ if } L \geq \widehat{L} \\
                                                                                                                                                                                                                                                                                                                                                                                             \alpha \,\ln\left(-\frac{p_{1}-\sqrt{{p_{1}}^2-4\,L\,p_{2}}}{2\,p_{2}}\right)-\ln\left(Y\right)\,\left(\alpha -1\right)
 &\text{ if } L < \widehat{L}
\end{cases} \\
&\widehat{L} =                                                                                                                                                                                                                                                                                                                                 \frac{Y}{2\,\left(\alpha -1\right)}+\frac{\frac{Y\,p_{2}}{2}+\frac{{p_{1}}^2}{8}-\frac{p_{1}\,\sqrt{{p_{1}}^2-\alpha \,{p_{1}}^2+8\,Y\,\alpha \,p_{2}}}{8\,\sqrt{1-\alpha }}}{p_{2}}

\end{split}
\end{align}



\subsection{Tariff Structure and Approximation}\label{appendix:tariff}

% \begin{figure}
% \caption{Example Residential Tariff Presented to Consumers}\label{figure:tarifftrue}
% \begin{center}
% \includegraphics[scale=.8]{tables/tariff_pic_png.png} \\
% \footnotesize{``conn.'' refers to connection, ``cu.m.'' refers to cubic meters of usage \\ per month, and \textbf{P} refers to Philippine Pesos.  Includes Tariff as of February, 2019.}
% \end{center}
% \end{figure}


\begin{table}[H]
\centering
\caption{Example Residential Tariff As Presented to Consumers}\label{table:tarifftrue}
\vspace{-2mm}
\begin{tabular}{l*{1}{r}}
\toprule
Usage (m3) & Price (PhP) \\
\midrule
Under  10    &   \input{tables/f_10}/conn. \\
Over  10     &   \input{tables/f_11}/conn. \\
Next   10    &   \input{tables/c_12}/cu.m. \\
Next    20   &   \input{tables/c_21}/cu.m. \\
Next   20    &   \input{tables/c_41}/cu.m. \\
Next    20   &   \input{tables/c_61}/cu.m. \\
Next    20   &   \input{tables/c_81}/cu.m. \\
Next    50   &   \input{tables/c_101}/cu.m. \\
Next   50    &   \input{tables/c_151}/cu.m. \\
Over  200    &   \input{tables/c_200}/cu.m. \\
\bottomrule
\multicolumn{2}{l}{\scriptsize Mean tariff 2010-2015 with value added tax. }\\[-.5em]
\multicolumn{2}{l}{\scriptsize ``conn.'' refers to connection. }\\[-.5em]
\multicolumn{2}{l}{\scriptsize ``cu.m.'' refers to m3/month.  50 PhP$\sim$1 USD }
\end{tabular}
\end{table}

Table~\ref{table:tarifftrue} provides the monthly tariff structure as it is presented to consumers.  Consumers face a fixed price as well as marginal prices for any usage above 10 m3.  The marginal price is highly non-linear, accelerating quickly at low usage levels before slowly increasing at high usage levels.  To achieve a tractable approximation of this price schedule, Table~\ref{table:tcd_predict} fits a simple regression model predicting average price as a function of an intercept, $p_1$, and monthly usage levels, $p_2$.  This model predicts that a increase in monthly usage of 10 m3 results in an increase in average price of 2.2 PhP/m3.  

\begin{table}[H]
\small
\centering
\caption{Average Price and Monthly Usage}\label{table:tcd_predict}
\vspace{-2mm}
\begin{tabular}{lc}
\toprule
& \small Avg. Price: $\frac{\text{Bill (PhP)}}{\text{Usage (m3)}}$    \\
\midrule 
Usage (m3)          &        0.29\textsuperscript{a}\\
                    &      (0.00)                   \\[0.5em]
Intercept           &       17.57\textsuperscript{a}\\
                    &      (0.05)                   \\[0.5em]
Household-Months    &   1,861,200                   \\

\bottomrule
\multicolumn{2}{l}{\scriptsize \textsuperscript{c} p$<$0.10,\textsuperscript{b} p$<$0.05,\textsuperscript{a} p$<$0.01 }
\end{tabular}
\end{table}



\subsection{Stayer Descriptives}


\begin{table}[H]
\centering
\caption{Descriptives for Stayers}\label{table:descriptives_stayers}
\vspace{-2mm}
\begin{tabular}{l*{1}{cccccc}}
\toprule
 & Mean & SD & Min & 25th & 75th & Max  \\
\midrule
 Usage (m3)  & 26.2  & 17.5  & 0.0  & 15.0  & 33.0  & 200.0  \\ 
 Bill  & 761  & 1,124  & -4,640  & 287  & 920  & 78,409  \\ 
 Unpaid Balance  & 2,416  & 5,070  & -4,995  & 261  & 2,346  & 79,904  \\ 
 Share of Months with Payment  & 0.60  & 0.49  & 0.00  & 0.00  & 1.00  & 1.00  \\ 
 Payment Size  & 1,214  & 1,498  & 0  & 426  & 1,482  & 61,298  \\ 
 Days Delinquent  & 84.9  & 155.4  & 0.0  & 0.0  & 91.0  & 720.0  \\ 
 Delinquency Visits per HH  & 1.32  & 0.61  & 1.00  & 1.00  & 2.00  & 6.00  \\ 
 Share of Months Disconnected  & 0.03  & 0.17  & 0.00  & 0.00  & 0.00  & 1.00  \\ 

\bottomrule
\multicolumn{7}{c}{Total Households: \input{tables/total_hhs_stayers}  Obs. per Household: \input{tables/obs_per_hh_stayers} Total Obs.: \input{tables/total_obs_stayers}}
\end{tabular}
\end{table}

\begin{figure}[H]
\centering
\caption{Stayers Share of Households that Receive a Delinquency Visit \\ depending on Days Delinquent in the Previous Month}\label{figure:dc_hazardstayers}
\includegraphics[scale=.7]{tables/connected_visit_hazard.pdf} \\
{ \footnotesize Only stayer households. }
\end{figure}

\subsection{Discussion of the Discount Rate}

\cite{andreoni2012estimating} estimate rates between 25\% and 35\% in an experimental setting and confirm exponential discounting.  \cite{laibson2007estimating} use a similar consumption-savings structural approach and recover a discount rate of around 15\%.  \cite{gourinchas2002consumption} use a similar structural approach finding a lower discount rate of around 5\%.


\nocite{*}
\singlespacing
\setlength{\bibsep}{7pt}
\bibliographystyle{abbrvnat}
\bibliography{ref}




\end{document}


