\documentclass[12pt]{article}

% TEMPLATE DEFAULT PACKAGES
\usepackage{amssymb,amsmath,amsfonts,eurosym,geometry,ulem,graphicx,color,setspace,sectsty,comment,natbib,pdflscape,array,adjustbox}

% ADDED PACKAGES FOR THIS MANUSCRIPT
\usepackage{palatino,newtxmath,multirow,titlesec,threeparttable,tabu,booktabs,titlesec,threeparttable,mathtools,bm,bbm,subcaption,pdflscape,tcolorbox,mathrsfs}
% endfloat,

\usepackage{afterpage}
\usepackage[hyphens]{url}
\usepackage[margin=1cm]{caption}

\usepackage[draft]{hyperref}
\newcommand{\tim}{$\,\times\,$}
% FIGURES & TABLES CAPTION STYLING
\captionsetup[figure]{labelfont={bf},name={Figure},labelsep=period}
\captionsetup[table]{labelfont={bf},name={Table},labelsep=period}

% SECTION TITLE SETTINGS
\titlelabel{\thetitle.\enskip}
\titleformat*{\section}{\large\bfseries}
\titleformat*{\subsection}{\normalsize\bfseries}

% COLUMN TYPES
\newcolumntype{L}[1]{>{\raggedright\let\newline\\\arraybackslash\hspace{0pt}}m{#1}}
\newcolumntype{C}{>{\centering\arraybackslash}p{5.2em}}
\newcolumntype{D}{>{\centering\arraybackslash}p{5em}}
\newcolumntype{R}[1]{>{\raggedleft\let\newline\\\arraybackslash\hspace{0pt}}m{#1}}


% MARGINS AND SPACING
\normalem
\geometry{left=1.1in,right=1.1in,top=1.0in,bottom=1.0in}
\setlength{\parskip}{2.5pt}

% SPECIAL CELL 
\newcommand{\specialcell}[2][c]{%
	\begin{tabular}[#1]{@{}l@{}}#2\end{tabular}}

% NO INDENT ON FOOTNOTES
\usepackage[hang,flushmargin]{footmisc}

\begin{document}

\begin{titlepage} 
\title{{Paying bills}\thanks{We are grateful to Andrew Foster}}
\author{\\[3em]
  William Violette\thanks{Federal Trade Commission, Washington, DC. E-mail: william.j.violette@gmail.com} \\
 \\ 
  }
\vspace{30mm}
\date{\vspace{5mm}This Version: \today}
\maketitle
\begin{abstract}

	paying bills

 %\\

%\vspace{0in}\\
%\textbf{Keywords:} traffic externalities; street livability; urban policy; housing market.\\
%\vspace{0in}\\
%\textbf{JEL Codes:} O18; H4; R2; R4.\\
\bigskip
\end{abstract}
\setcounter{page}{0}
\thispagestyle{empty}
\end{titlepage}
\pagebreak \newpage

\spacing{1}


\section{Pitch}

This paper looks at whether utilities in developing countries provide an important source of credit to households by letting them not pay their bill?  And how do these benefits compare to the costs of delinquency?

Motivation : (1) around the world, there are restrictions on when utilities can disconnect people (health, low-income, elderly), motivated by providing a sense of insurance;  (2) Also, in developing countries, pre-paid metering where utilities essentially put fancy meters that only distribute water when people pay first.  these shut off any credit channel but eliminate delinquency

Context : In Manila, water bills make up about 3\% of income on average, jumps to 5 to 10\% for low-income folks.  I have data from a large water provider serving half the city of Manila.  On average, people make a payment in three out of four months and are about a month behind in payments.  For example, if you don't pay for three months, that's like taking a three-month loan of around 9\% of your income.  

People might be paying infrequently because they are credit-constrained and consumption smoothing; or it might just be a pain to pay their bills so they just avoid the hassle by paying infrequently (and they have plenty of other opportunities to smooth).  

I use disconnection threats to better isolate the role of credit constraints.  Workers will occasionally visit delinquent customers and say if you don't pay your bill soon (within 12 days on average), we'll disconnect you.  Only 23\% of people say that's ``enough time'' to pay the bill.  After the threats, payments spike to double the average bill, delinquency drops to zero, and consumption drops by around 25\%.  If it were simply a hassle, people would pay their bill and consumption wouldn't change (they could get easy credit from other places and pay the bill); but if they are credit constrained, they have to deviate from consumption smoothing to make the payments.  The next step is to use theory to see what this decrease in consumption implies about the short-term interest rate that households face.


\section{Introduction}

%% disconnection 

% https://liheapch.acf.hhs.gov/Disconnect/disconnect.htm
\subsection{Question} 
Are utilities providing an important source of credit to households by letting them not pay their bill?

\subsection{Motivation}
\begin{itemize}
\item Disconnection policies as insurance (in the US) 
	\begin{itemize}
		\item weather, health, low-income
		\item mandate that utilities amortize arrears
		\item Utilities even tolerate greater non-payment
	\end{itemize}
\item At the same time, pre-paid metering is growing like crazy in the developing world (Sources)
\end{itemize}

\subsection{Descriptives}
% How much credit could households get from their water bill?
\begin{itemize}
\item Avg Income: 22,000 PhP (488 USD)   [ Bill 3\% ]
\item Avg Savings: 4,300 PhP (96 USD) 
\item 20\% Income: 8,300 PhP (184 USD)  [ Bill 7.6\% ]
\item 20\% Savings: 330 PhP (7 USD)  
\item Avg Bill: 630 PhP (14 USD)
\item Make payments 75\% of months
\item Avg delinquency: 30 days
\item Avg Payment Amount: 830 PhP (18 USD)
\end{itemize}

But : might just be inconvenient to pay every month (but its really easy to pay bills in this context)

How can I ballpark this against the consumption smoothing literature?

\subsection{Approach}

Disconnection : Don't pay bill; come and threaten disconnection
\begin{itemize}
\item avg days to pay : 12 days (only 23\% say that's enough time)
\item if you (agree to?) pay, you are reconnected after 2 days
\item 30\% of connections are threatened with disconnection
\item (small percent actually disconnect)
\item Pay (+1) 800 PhP (+2) 300 PhP = total 1100 PhP (24 USD) [ about two water bills on average ]
\item Consumption drops by about 20\% for two months (there is a pre-trend which can be interpreted as positive demand shocks)
\end{itemize}

\noindent Theory :
\begin{itemize}
\item suddenly this source of credit is cut off (loan with uncertain payback date)
\item concave utility predicts that households would want to smooth consumption (could get another loan, then fund consumption in that period)
\item 
\end{itemize}



\section{Interest rate}

% https://www.cgdev.org/blog/compartamos-context


\cite{karlan2009expanding} find money lenders regularly charge at least 20\% per month for credit.  \cite{gine2014group} offer small monthly loans of 1,000 PhP at 2.5\% monthly interest.

\cite{andreoni2012estimating} estimate rates between 25\% and 35\% in an experimental setting and confirm exponential discounting.  \cite{laibson2007estimating} use a similar consumption-savings structural approach and recover a discount rate of around 15\%.  \cite{gourinchas2002consumption} use a similar structural approach finding a lower discount rate of around 5\%.



%%% Karlan and Zinman

% Informal credit markets
% and serial borrowing from moneylenders charging 20% per month or more is common (e.g., more than
% 30% of our sample reported borrowing from moneylenders during the past year). Trade credit is quite
% uncommon. There are several microlenders operating in Metro Manila, but most MFIs operate on a
% small scale (as noted above) and charge high rates (see below).


\section{Results}

\begin{table}
\centering
\caption{Estimates}\label{table:estimates}
\begin{tabular}{lcc}
& Estimate & Standard Error \\
Interest Rate &0.038&0.00\\
Income Variance &0.270&0.00\\
Water Preference &0.019&0.00\\
Fixed Cost of Non-Piped Water &207.0&0.00\\
\end{tabular} 

\end{table}

\begin{table}
\centering
\caption{Fit}\label{table:fit}
\begin{tabular}{lcc}
& Data & Estimated \\
Mean Usage (m3) &24.9&24.6\\
SD Usage &11.1&2.3\\
Mean Water Debt (PhP) &1205&1412\\
SD Water Debt (PhP) &1258&1955\\
Corr. Usage and Water Debt &0.35&0.05\\
Mean Disc. for 1 month &0.061&0.073\\
Mean Disc. for 2 months &0.061&0.047\\
Mean Disc. for 3 months &0.048&0.030\\
Mean Disc. for 4 months &0.040&0.017\\
\end{tabular} 

\end{table}





\nocite{*}
\singlespacing
\setlength{\bibsep}{7pt}
\bibliographystyle{abbrvnat}
\bibliography{ref}



\end{document}


