\documentclass[12pt]{article}

% TEMPLATE DEFAULT PACKAGES
\usepackage{amssymb,amsmath,amsfonts,eurosym,geometry,ulem,graphicx,color,setspace,sectsty,comment,natbib,pdflscape,array,adjustbox}

% ADDED PACKAGES FOR THIS MANUSCRIPT
\usepackage{palatino,newtxmath,multirow,titlesec,threeparttable,tabu,booktabs,titlesec,threeparttable,mathtools,bm,bbm,subcaption,pdflscape,tcolorbox,mathrsfs,float}
% endfloat,

%\usepackage{kbordermatrix}% http://www.hss.caltech.edu/~kcb/TeX/kbordermatrix.sty
%\usepackage{amsmath}% http://ctan.org/pkg/amsmath

\usepackage{afterpage}
\usepackage[hyphens]{url}
\usepackage[margin=1cm]{caption}

\usepackage[draft]{hyperref}
\newcommand{\tim}{$\,\times\,$}
% FIGURES & TABLES CAPTION STYLING
\captionsetup[figure]{labelfont={bf},name={Figure},labelsep=period}
\captionsetup[table]{labelfont={bf},name={Table},labelsep=period}

% SECTION TITLE SETTINGS
\titlelabel{\thetitle.\enskip}
\titleformat*{\section}{\large\bfseries}
\titleformat*{\subsection}{\normalsize\bfseries}

% COLUMN TYPES
\newcolumntype{L}[1]{>{\raggedright\let\newline\\\arraybackslash\hspace{0pt}}m{#1}}
\newcolumntype{C}{>{\centering\arraybackslash}p{5.2em}}
\newcolumntype{D}{>{\centering\arraybackslash}p{5em}}
\newcolumntype{R}[1]{>{\raggedleft\let\newline\\\arraybackslash\hspace{0pt}}m{#1}}


% MARGINS AND SPACING
\normalem
\geometry{left=1.1in,right=1.1in,top=1.0in,bottom=1.0in}
\setlength{\parskip}{2.5pt}

% SPECIAL CELL 
\newcommand{\specialcell}[2][c]{%
	\begin{tabular}[#1]{@{}l@{}}#2\end{tabular}}

% NO INDENT ON FOOTNOTES
\usepackage[hang,flushmargin]{footmisc}

\begin{document}

\begin{titlepage} 
\title{{Paying bills}\thanks{We are grateful to Andrew Foster}}
\author{\\[3em]
  William Violette\thanks{Federal Trade Commission, Washington, DC. E-mail: william.j.violette@gmail.com} \\
 \\ 
  }
\vspace{30mm}
\date{\vspace{5mm}This Version: \today}
\maketitle
\begin{abstract}

	paying bills

 %\\

%\vspace{0in}\\
%\textbf{Keywords:} traffic externalities; street livability; urban policy; housing market.\\
%\vspace{0in}\\
%\textbf{JEL Codes:} O18; H4; R2; R4.\\
\bigskip
\end{abstract}
\setcounter{page}{0}
\thispagestyle{empty}
\end{titlepage}
\pagebreak \newpage

\spacing{1}


\section{Pitch}

Why structure?
\begin{itemize}
    \item cool counterfactuals
        \begin{enumerate}
            \item full welfare of prepaid meters (with revenue equaling counterfactual)
            \item how do interest rate changes affect water demand and delinquency? spillover effects!?
        \end{enumerate}
    \item measure consumption distortions from water loans [ shows up with first one.. ]
    \item estimate a revealed credit constraint parameter using utility data
        \begin{enumerate}
            \item show that they are substitutes!
        \end{enumerate}
\end{itemize}

This paper looks at whether utilities in developing countries provide an important source of credit to households by letting them not pay their bill?  And how do these benefits compare to the costs of delinquency?

Motivation : (1) around the world, there are restrictions on when utilities can disconnect people (health, low-income, elderly), motivated by providing a sense of insurance;  


(2) Also, in developing countries, prepaid metering where utilities essentially put fancy meters that only distribute water when people pay first.  these shut off any credit channel but eliminate delinquency

\cite{heymans2014limits}
 prepayment water systems are in use in more than
20 African countries, and in locations such as Turkey, parts
of the Balkans and Azerbaijan, and Colombia. Their scale
of use is ramping up rapidly. The Botswana Water Utilities
Corporation, for example, is reported to be planning to install
300,000 prepaid water meters in the near future, beyond
the existing small-scale installations. 

don't like : “Postpaid gives you more time to find the money.”  “Water is a need, but money is not always available.”


%% compared to prepaid electricity
% • Prepaid water is often seen as controversial. Payment for the supply of electricity is accepted more widely than
% payment for water, and access to electricity is not regarded a basic human right. T
% • The electricity sector is much less fragmented than the water sector, and has far greater clout to direct what
% manufacturers supply. 
% • Prepaid water meters face physical stresses that do not apply to electricity. There are more moving parts,
% most subjected to fluctuating pressures and flows, and wear, fatigue, and abrasion increase the likelihood of
% malfunction. 


% When discussing costs with service providers, the review
% team was told on several occasions that “prepaid meters cost
% about four times more than a conventional meter.” This is
% based on the typical cost of a prepaid metering device for an
% individual domestic connection, about US$210, compared
% to typical conventional mechanical meter, about US$50.   96.2\% of costs for large systems.  average of 7 year life span (assume the same for both)





Context : In Manila, water bills make up about 3\% of income on average, jumps to 5 to 10\% for low-income folks.  I have data from a large water provider serving half the city of Manila.  On average, people make a payment in three out of four months and are about a month behind in payments.  For example, if you don't pay for three months, that's like taking a three-month loan of around 9\% of your income.  

People might be paying infrequently because they are credit-constrained and consumption smoothing; or it might just be a pain to pay their bills so they just avoid the hassle by paying infrequently (and they have plenty of other opportunities to smooth).  

I use disconnection threats to better isolate the role of credit constraints.  Workers will occasionally visit delinquent customers and say if you don't pay your bill soon (within 12 days on average), we'll disconnect you.  Only 23\% of people say that's ``enough time'' to pay the bill.  After the threats, payments spike to double the average bill, delinquency drops to zero, and consumption drops by around 25\%.  If it were simply a hassle, people would pay their bill and consumption wouldn't change (they could get easy credit from other places and pay the bill); but if they are credit constrained, they have to deviate from consumption smoothing to make the payments.  The next step is to use theory to see what this decrease in consumption implies about the short-term interest rate that households face.


\section{Introduction}

REPRESENTATIVE HOUSEHOLD APPROACH

\cite{jack2015pay} small frequent purchases of electricity mean that prepaid meters provide a savings device for liquidity constrained households.  


\cite{jack2016charging} find decrease in usage of 13\%, what share of the bill 8-10\% of income?

  they acknowledge the credit mechanism


\cite{northeast2014} predict rapid increases in prepaid electricity meters throughout Sub-Saharan Africa.  

%% disconnection 

% https://liheapch.acf.hhs.gov/Disconnect/disconnect.htm
\subsection{Question} 
Are utilities providing an important source of credit to households by letting them not pay their bill?

\subsection{Motivation}
\begin{itemize}
\item Disconnection policies as insurance (in the US) 
	\begin{itemize}
		\item weather, health, low-income
		\item mandate that utilities amortize arrears
		\item Utilities even tolerate greater non-payment
	\end{itemize}
\item At the same time, prepaid metering is growing like crazy in the developing world (Sources)
\end{itemize}

\section{Data}

% Describe notice of disconnection in more detail here??? and  NO DO IT LATER~! PLEASE!!!!

Measuring credit through unpaid water bills requires information on monthly water bills at the household level.  This paper relies primarily on water-company records collected as part of a research partnership with one of the two regulated, private providers in Manila.  This company provided access to monthly billing records for each connection as well as detailed information covering the regulatory structure and costs of production.  The monthly billing records include meter readings, total bill due, and payments spanning January, 2010 to May, 2015.  Over this timeframe, the total number of connections increase from 900,000 to 1,500,000 as the water company expanded service access.  Water connections are split into four categories: residential (90\%), semi-business (4\%), commercial (5\%), and industrial (1\%).  Since this research focuses on household consumption and savings decisions, only connections coded as residential are included in the analysis.  

% Since the object of interest for welfare is household-level demand, 

Since multiple households often share the same connection in Manila, the billing data is merged to survey data that measures the number of households using each connection as well as demographics for the household that owns the connection.  Conducted every two years between 2008 and 2012, the survey data cover a total of close to 50,000 water connections.\footnote{Connection survey documentation indicates that surveyors used stratified random sampling where the number of connections surveyed was proportional to the census population in the smallest administrative census district (Barangay).}  Alongside providing demographic information, this survey data allows for limiting the analysis to connections that each serve a single household in order to map the billing data for each connection into the consumption and savings decisions of a single household.\footnote{\cite{wjv} finds that sharing a connection with multiple households affects household water demand through a variety of channels.}   However, previous research from \cite{wjv} finds that households using a connection individually tend to have higher demand for water, larger household sizes, and greater likelihood of living in single houses (as opposed to duplexes or apartments) in comparison to households sharing water connections.  These differences may limit the generalizability of the findings outside of this subpopulation in Manila.  This sample restriction also ensures that household-level data and connection-level data can be referred to interchangeably.  Table~\ref{table:sampleconstruction} in Appendix~\ref{appendix:sampleconstruction} includes more details on how the sample is constructed for the analysis.

Because the connection survey data does not include income measures, average monthly income for households in Manila is taken from the 2015 Family Income and Expenditure Survey.  The average interest rate for the Philippines is calibrated from the World Bank Databank (2010-2015).

% which help to calibrate the structural model.  

%  The water company data also flag when delinquent households experience visits from water company staff where the   MAYBE NOT! include this in the data section


%The connection survey records the total number of households and people using each connection, providing the primary measure of sharing behavior.  The connection survey also includes basic demographics for the owner of the connection.  The demographics used in the analysis include household size, dwelling type (whether apartment, duplex, or single house), number of employed household members, and whether the head of household is employed in a low-skill profession.



%consumption and water source choices at the household-level.  
%To examine both consumption decisions and water source choices at the household-level, I construct a dataset that merges (1) monthly water usage for around 50,000 connections, (2) demographics and water sharing measures for households using these connections, and (3) a cross-section of all households and their water source choices.  Put together, these data form a representative sample of households in half of Metro Manila.

%A full analysis of water usage in Manila also requires measuring households that are unconnected to the bulk water system and instead use water from local vendors.  The 2010 Census of Population and Housing records demographics for households that are using water from alternative sources.  Households are defined as connected to the piped network if they report using a ``faucet community water system'' for their cooking, laundry, or bathing needs.\footnote{The census was designed largely for rural populations so it does not directly ask whether households are connected to a piped water network.}  Alternatively, households are considered to be using from a water vendor if they report any other category, which includes deep wells, peddlers, or other sources.\footnote{Since demographic questions in the census and connection surveys are worded differently, Table~\ref{table:pawstocensus} of Appendix~\ref{appendix:pawstocensus} compares measures across the two data sources for households that own connections finding broadly similar patterns.}  Census data is merged geographically with the connection survey for 599 wards in Manila.

%To construct a profile of water source choices for the full population, I randomly sample from the pool of households using from vendors in 599 wards of the city so that the data form a representative sample when combined with households covered by the connection survey in each small area.
%\footnote{Regions are chosen to overlap with ward administrative/political boundaries while ensuring that I am able to obtain a large enough sample of vendor households for each area to perform the counterfactual analysis in Section~\ref{section:fixedcostestimation}.}

% For additional information on water sharing relationships between households, I fielded a small, non-representative survey of 600 households via mobile phone application.  This survey asks about issues associated with using different water sources, basic demographics, as well as how the bill and connection fee are paid for sharing households.  Appendix~\ref{appendix:pollfish} compares this survey to the connection survey finding roughly similar rates of sharing and demographic characteristics.

%includes additional details on this survey and finds roughly similar measures between the connection survey and the mobile survey, suggesting that the mobile survey may provide a reasonable descriptive picture of sharing in Manila.

% The water provider also maintains call records of all service complaints, which can be linked to individual connections.  This data is used to measure instances where households face unanticipated shocks to their water source, which are used to identify the marginal costs of using water from a neighbor.  Finally, the water provider tracks data on the marginal cost of pumping an additional cubic meter of water as well as the costs associated with installing new connections.  These data provide inputs into the provider's budget constraint, which are necessary for computing counterfactual pricing regimes.  

% With information on water source and monthly expenditures as well as household income, the 2011 Community Based Monitoring System data cover over 70\% of households in Pasay City --- a large area with a population of 500,000 in downtown metro Manila.  These data are used to impute income for households outside of the sample according to the following demographic measures: age of household head and indicators for low-skilled employment of household head, residing in a duplex, residing in a house, as well as all possible household sizes and number of employed household members.  The imputation also includes all possible interactions between these measures.



\section{Credit through Unpaid Water Bills}\label{section:descriptives}

% \footnote{The company upgraded t in the first year of the sample.  Previously, the company mailed the bill to each household at the end of the month.} 
% Each month, households consume \input{tables/usage_all}m3 of piped water on average, which totals an average monthly bill of \input{tables/bill_all}PhP as shown in Table~\ref{table:descriptives_all}.  Given an average monthly income in Manila of \input{tables/y_avg}PhP and savings rate of \input{tables/save_avg}PhP,  monthly water bills reach a modest average of \input{tables/bill_inc_all}\unskip\% of income and \input{tables/bill_sav_all}\unskip\% of savings.  


\begin{table}[h!] % PUT IN DEMOGRAPHICS PLEASE ?!?!?
\centering
\caption{Mean Characteristics}\label{table:descriptives_all}
\vspace{-2mm}
\begin{tabular}{l*{1}{cccccc}}
\toprule
 & Mean & SD & Min & 25th & 75th & Max  \\
\midrule
 Usage (m3)  & 24.3  & 15.4  & 0.0  & 14.0  & 31.0  & 200.0  \\ 
 Bill  & 671  & 734  & -4,640  & 265  & 843  & 19,999  \\ 
 Unpaid Balance  & 1,206  & 3,316  & -9,965  & 0  & 1,044  & 79,904  \\ 
 Share of Months with Payment  & 0.71  & 0.45  & 0.00  & 0.00  & 1.00  & 1.00  \\ 
 Payment Size  & 901  & 1,067  & 0  & 313  & 1,070  & 49,688  \\ 
 Days Delinquent  & 56.2  & 113.7  & 0.0  & 0.0  & 61.0  & 750.0  \\ 
 Delinquency Visits per HH  & 0.40  & 0.70  & 0.00  & 0.00  & 1.00  & 6.00  \\ 
 Share of Months Disconnected  & 0.04  & 0.18  & 0.00  & 0.00  & 0.00  & 1.00  \\ 

\bottomrule \\ [-.8em]
\multicolumn{7}{c}{Total Households: \input{tables/total_hhs_all}  Obs. per Household: \input{tables/obs_per_hh_all} Total Obs.: 2,844,481}
\end{tabular}
\end{table}

Unpaid water bills may provide a reliable source of low-cost credit to households because (1) the company tolerates high rates of delinquency before disconnecting them from service and (2) the water company is prohibited from charging any interest on outstanding balances. 

At the end of each month, the water company sends meter readers who record monthly consumption for each connection and then use a mobile device to print and deliver the bill to the household in person.  The household is then expected to pay the bill by the end of the month.  Households have many options to pay their bills with \input{tables/center}\unskip\% using small payment centers (mall kiosks, gas stations, convenient stores, etc.), \input{tables/maynilad}\unskip\% paying at local water company offices, and \input{tables/atm}\unskip\% paying over phone, online, or via ATM kiosks.\footnote{Figures are tabulated from the connection survey sample.}  Despite easy payment mechanisms, households rarely pay their bills on time.  Table~\ref{table:descriptives_all} provides summary statistics on the usage, billing, and payment patterns of households.  On average, households are \input{tables/days_delinquent_all}days behind in their payments.  Households also make large, infrequent payments.  While the average bill is \input{tables/bill_all}PhP per month, payment sizes average \input{tables/pay_size_all}PhP and households make payments in only \input{tables/pay_freq}\unskip\% of months.  These payment patterns leave an average total outstanding balance of \input{tables/balance_all}PhP per month.  

Given average monthly household incomes of \input{tables/y_avg}PhP and savings rates of \input{tables/save_avg}PhP in Manila, unpaid water bills reach \input{tables/balance_inc}\unskip\% of income and \input{tables/balance_save}\unskip\% of savings on average each month.  For households at the 20th income percentile, unpaid water bills jump to \input{tables/balance_inc_20_all}\unskip\% of monthly income.  As a benchmark, \cite{cull2009microfinance} survey microfinance institutions throughout the developing world and find median yearly loan sizes expressed as a share of the 20th percentile of household income at 48\% for nongovernmental organizations (NGOs), 160\% for nonbank financial institutions, and 224\% for banks.  These descriptives suggest that a yearly loan from an NGO could be reached with around 5 months of unpaid water bills for households at the 20th income percentile.\footnote{Deal with income effects on water consumption!}  Moreover, microfinance loans charge high yearly interest ranging from 25\% for NGOs to 13\% for banks.  In comparison, unpaid water bills are interest-free, but households face some risk of service disconnection for delinquency.

After bills have remained unpaid for at least 60 days, the government regulator permits the water company to visit delinquent households to disconnect their water service.\footnote{Regulations also require the water company to issue written statements to households, notifying them that their connections will be disconnected in 7 days if their outstanding balances remain unpaid.  In practice, only \input{tables/disc_notice}\unskip\% of disconnected households report receiving advanced notice.  All percentages are calculated from \input{tables/disc_count}respondents who report being visited for disconnection in the past year out of \input{tables/paws_accounts}total respondents in the connection survey.}  Likely due to time and travel costs, delinquency visits are relatively rare in practice, occurring in only \input{tables/tcd_id_mean}\unskip\% of household-month observations.  This probability increases to \input{tables/tcd_id_ar_cond}\unskip\% for household-months over 60 days overdue.  Figure~\ref{figure:dc_hazard} plots the share of months with delinquency visits according to days overdue.  The risk of delinquency visits spikes at 61 days overdue before settling to around 2.5\% at higher levels of delinquency.  Appendix Table~\ref{table:tcd_predict} predicts delinquency visits with days overdue, outstanding balance, and household characteristics.  This exercise finds that days overdue and unpaid balances are strongly and independently correlated with visits while demographic indicators have weaker associations. % more to say here about magnitudes?!

When company staff conduct delinquency visits, households often negotiate for additional time to pay their outstanding balances.  \input{tables/days_pay_under_30}\unskip\% of households report agreeing to pay within 30 days and the average grace period is \input{tables/days_pay_average}days; however, only \input{tables/enough_time}\unskip\% of connections report having ``enough time'' to make their payments.  For households who fail to pay, disconnection typically involves workers from the water company placing a metal lock on the water meter stopping the flow.  In order to reconnect, households must pay a small, one-time fee of 200 PhP on top of settling any outstanding balances.  The water company is then required to restore service within 48 hours of receiving full payment for reconnection, which is confirmed in survey data.\footnote{Households report being reconnected \input{tables/days_rec_average}days after payment.}  


\begin{figure}
\centering
\caption{Share of Households that Receive a Delinquency Visit \\ depending on Days Delinquent in the Previous Month}\label{figure:dc_hazard}
\includegraphics[scale=.7]{tables/connected_visit_hazard_all.pdf}
\end{figure}



\begin{table}[h!]
\centering
\caption{Mean Characteristics by Delinquency Visit Status}\label{table:descriptives_3g}
\vspace{-2mm}
\begin{tabular}{l*{1}{ccc}}
\toprule
 & Never Visited & Stayers & Leavers  \\
% &   &         &        \\
\midrule
 Usage (m3)  & 27.1  & 30.0  & 31.5  \\ 
 Bill  & 782  & 931  & 1,062  \\ 
 Unpaid Balance  & 917  & 3,009  & 8,308  \\ 
 Share of Months with Payment  & 0.78  & 0.62  & 0.38  \\ 
 Payment Size  & 978  & 1,448  & 1,553  \\ 
 Days Delinquent  & 20.7  & 90.9  & 259.1  \\ 
 Delinquency Visits per HH  & 0.00  & 1.35  & 1.45  \\ 
 Months Disconnected  & 0.01  & 0.04  & 0.33  \\ 
 HH Size  & 5.0  & 5.4  & 5.4  \\ 
 Age of HoH  & 47.2  & 44.7  & 45.6  \\ 
 Low Skilled HoH  & 0.15  & 0.19  & 0.21  \\ 
 &  &  &  \\ 
 Total Households   & 30,393  & 11,856  & 3,849  \\ 
 Total Observations  & 1,876,658  & 731,935  & 235,888  \\ 

\bottomrule
\multicolumn{4}{l}{\scriptsize ``Never Visited'' includes HHs that never receive a delinquency visit.}\\  [-.5em]
\multicolumn{4}{l}{\scriptsize ``Stayers'' includes HHs with $\geq$1 visit and are connected for the last 6 months.}\\ [-.5em]
\multicolumn{4}{l}{\scriptsize ``Leavers'' includes HHs with $\geq$1 visit and are disconnected for $\geq$1 of the last 6 months.}\\ [-.5em]
\multicolumn{4}{l}{\scriptsize Bill, Unpaid Balance, and Payment Size are in PhP/month.} 
\end{tabular}
\end{table}


Table~\ref{table:descriptives_3g} provides mean characteristics according to whether and how households respond to delinquency visits over the course of the sample.  The first column, ``Never Visited,'' includes households that never receive a delinquency visit.  Since delinquency visits are relatively rare, the majority of observations fall into this category.  The second and third columns include households that receive at least one delinquency visit.   The second column, ``Stayers,'' also requires that households are connected for the final 6 months of the sample while the third column, ``Leavers,'' includes households that are disconnected for at least one of the final 6 months of the sample.  

Leavers are predominantly composed of households that permanently disconnect over the sample period, which likely occurs when households move out of their current residences.  These households often leave large outstanding balances that are almost never repaid due to difficulties tracking households across locations.  By incentivizing households to pay, frequent delinquency visits provide a strategy for the company to minimize this lost revenue.

Stayers include households that remain connected at the end of the sample despite receiving at least one delinquency visit over the duration.\footnote{Stayers also excludes \input{tables/pcd_hh}households that the company has flagged as ``permanently disconnected.''}  Compared to households that are never visited, stayers have much higher outstanding balances and days delinquent.  Their payments also occur less frequently but have larger average sizes.  Stayers spend 3\% of the sample period disconnected from service.  Until they are able to pay for reconnection, these households likely substitute to alternative water sources including sharing with neighbors, using from deepwells, or purchasing from local water vendors.\footnote{Table~\ref{table:descriptives_3g} indicates that even for households that are never visited, they are disconnected during 1\% of months.  These disconnections include (1) households that received a delinquency visit before the start of the sample (but later reconnected) and (2) households that notify the water company about their moving plans in advance and therefore, do not need a delinquency visit.}  Stayers also have slightly larger household sizes, younger heads of household, and greater incidence of low-skilled employment than never visited households.  These demographic patterns are consistent with lower-income households having greater difficulty paying their bills promptly.


\begin{figure*}[hbtp]
    \centering
    \vspace{2mm}
    \begin{subfigure}[b]{0.49\textwidth}
        \centering
        \caption[]{\small Share Disconnected}  
        \vspace{-1mm}
        \includegraphics[width=\textwidth,trim={.2cm .2cm .2cm 0cm}, clip=true]{tables/line_disconnection}
        \label{fig:line_disc}
    \end{subfigure}
    \hfill
    \begin{subfigure}[b]{0.49\textwidth}  
        \centering 
        \caption[]{\small Unpaid Balance}
        \vspace{-1mm}
        \includegraphics[width=\textwidth,trim={.2cm .2cm .2cm 0cm}, clip=true]{tables/line_bal}
        \label{fig:line_bal}
    \end{subfigure}
    \vskip 1mm \vskip 0pt
    \begin{subfigure}[b]{0.49\textwidth}
        \centering
        \caption[]{\small Days Overdue}
        \vspace{-1mm}
        \includegraphics[width=\textwidth,trim={.2cm .2cm .2cm 0cm}, clip=true]{tables/line_ar}
        \label{fig:line_ar}
    \end{subfigure}
    \hfill
    \begin{subfigure}[b]{0.49\textwidth}  
        \centering
        \caption[]{\small Monthly Payment}  
        \vspace{-1mm}
        \includegraphics[width=\textwidth,trim={.2cm .2cm .2cm 0cm}, clip=true]{tables/line_pay}
        \label{fig:line_pay}
    \end{subfigure}
    \vskip 1mm \vskip 0pt
    \begin{subfigure}[b]{.49\textwidth}  
        \centering
        \caption[]{\small Average Consumption} 
        \vspace{-1mm}
        \includegraphics[width=\textwidth,trim={.2cm .2cm .2cm 0cm}, clip=true]{tables/line_c}  
        \label{fig:line_c}
    \end{subfigure}
    \hfill \hspace{.02\textwidth}
    \begin{minipage}{0.47\textwidth}   
    \vspace{-6cm}
    \caption[]
    {\small Mean Outcomes around First Delinquency Visit  \\  \\  Figures include all households with delinquency visits.  Stayers include only households who are connected for the last 6 months of the sample.  Mean monthly payments include zeros when no payment is made. } \label{fig:line_graphs}
    \end{minipage}
\end{figure*} 


To investigate the timing of household responses to delinquency visits, Figure~\ref{fig:line_graphs} plots monthly mean outcomes in months relative to the first delinquency visit for each household.  Outcomes are plotted separately for all households that experience delinquency visits as well as stayer households who experience visits and also remain connected for the last 6 months of the sample.  In the months just preceding a visit, monthly payments (Figure~\ref{fig:line_pay}) decrease suddenly, leading to corresponding increases in unpaid balances (Figure~\ref{fig:line_bal}) and days overdue (Figure~\ref{fig:line_ar}).  Average consumption (Figure~\ref{fig:line_c}) increases slightly as well.  Prior to disconnection, stayers follow similar patterns but with lower levels of delinquency than the population average.

Immediately following the first delinquency visit, monthly payments spike as many households pay their full outstanding balances to prevent any disconnection.  Despite these payments, the average share of households disconnected (Figure~\ref{fig:line_disc}) also spikes to around \input{tables/am_2}\unskip\% before decreasing, as some households pay for reconnection, and stabilizing at \input{tables/am_20}\unskip\%, which is likely composed of households who have permanently disconnected.  Stayer households pay more and disconnect less than the population average following a visit.  Disconnection rates for stayers spike to \input{tables/am6_2}\unskip\% before declining to around \input{tables/am6_20}\unskip\% two years later.\footnote{Although stayers are connected for the last 6 months of the sample by definition, disconnection rates do not decline to zero since some stayers experience multiple disconnection spells.}  The decline among stayers accounts for much of the average decline in disconnection rates across the sample.  Although many stayers quickly reconnect within 6 months, reconnections continue accumulating up to one year after a visit.  

This descriptive evidence indicates that many households choose to disconnect for long periods before finally paying for reconnection.  This behavior is consistent with households facing credit-constraints, which prevent them from taking a low-cost loan to fund immediate reconnection.  Instead, households substitute to lower-quality alternative sources of water until they can save enough income for reconnection.  Credit constraints would additionally predict that the most delinquent households at the time of the visit may be the most likely to disconnect following a visit.  The data support this hypothesis, finding that the share disconnected two months after a visit is \input{tables/amar_2}\unskip\% for stayers that are over 90 days delinquent at the time of the visit.  The corresponding share for stayers under 90 days delinquent shrinks to \input{tables/amar_2a}\unskip\%.

Another hypothesis may be that households choose to stop paying when they leave their homes for vacation or overseas work.  This mechanism would likely predict a large decrease in water consumption in the months preceding delinquency visits.  Instead, consumption rises consistently between 24 and 2 months before a visit.  Consumption then drops one month before, during, and after a visit, before quickly returning to an average level.  This dip in consumption likely corresponds to connections that are disconnected for less than a month before being reconnected and having their consumption recorded for that month.  Overall, this variation is small given a mean consumption for stayers of \input{tables/c_avg}m3 with a standard deviation of \input{tables/c_std}\unskip.  Moreover, delinquency visits are relatively rare, which may make their exact timing difficult for households to predict.  











%The administrative data indicate that \input{tables/share_pay_3_months}\unskip\% resolve their outstanding balances following a notice.

% At this time, households report paying 

%In practice, the water company often tolerates delinquency well over 60 days, which is consistent with (1) costs of sending and following up on notifications and (2) household demand for credit from the water company.  Households receive notice in \input{tables/disc_prob}\unskip\% of months.  

% This rate increases to \input{tables/disc_prob_60}\unskip\% for households with at least 60 days of delinquency.  

% Figure~\ref{figure:dc_hazard} graphs the probability of receiving a delinquency notice conditional on the level of delinquency for each account.  The water company appears to carefully follow government regulations preventing any disconnection for delinquency below 60 days.  Once households reach 61 days of delinquency, they face some probability of receiving a notice, which triples moving above 90 days unpaid balance.  Conditional on having greater than 90 days of delinquency 

%  In practice, the water company tolerates \input{tables/ar_dc}days of delinquency on average before issuing a notice.  The level of delinquency tolerated before issuing notices also appears uncorrelated with (list measures/demographics...).  ( SHOW REGRESSION IN APPENDIX )  

% Upon receiving a notice of disconnection, connection owners then negotiate for time to pay their outstanding balances.  In \input{tables/days_pay_under_30}\unskip\% of notices, owners are agree to pay within 30 days, making an average grace period of \input{tables/days_pay_average}days according to the water connection survey.  \input{tables/enough_time}\unskip\% of connections report having ``enough time'' to make their payments.  

% %The administrative data indicate that \input{tables/share_pay_3_months}\unskip\% resolve their outstanding balances following a notice.

% Disconnection typically involves workers from the water company placing a metal lock on the water meter stopping the flow.  Once disconnected, connection owners are charged a small, one-time fee of 200 PhP to restore their water service on top of settling any outstanding balances.  The water company is then required to restore service within 48 hours of receiving full payment for reconnection.

% \begin{itemize}
% \item No notice: great access to credit, or just not enough time to observe them hitting a
% \item Average notice rate!?!
% \end{itemize}
%\footnote{To account for potential mismeasurement in the administrative data, this measure allows connection owners to resolve their outstanding balances within 60 days of receiving notice.}
% In practice, \input{tables/dc_note}\unskip\% of owners report receiving advanced notice.  

\section{Model of Household Borrowing and Water Use}

%\subsection{Setting up the Household's Problem}
%Time horizon etc.  Ignore disconnecting households.. 


To model household borrowing and saving alongside water usage choices, this framework builds on a standard intertemporal utility maximization problem as developed by \cite{deaton1991saving}.  This approach assumes an infinite time horizon where households choose water usage, borrowing, and savings in each period to maximize current and expected future utility.  Household expected utility is given by the following equation
\begin{align}\label{eq:u}
E_t \Big[ \sum_{\tau = t}^{t-\tau} (1+\delta)^{t-\tau} u(w_{\tau},x_{\tau})   \Big]
\end{align}
where utility in each period $t$ is expressed as an increasing and concave function of water consumption, $w_{\tau}$, and consumption of all other goods, which are collapsed into a single term, $x_{\tau}$.  Households have a rate of time preference $\delta \in (0,1)$.  In each period, households face a budget constraint as follows
\begin{align}\label{eq:bc}
x_t \, + \, p(w_t) w_t \, &= \, y_t \, - D_{t+1} f  \, + \, S_t \, - \, S_{t+1}
% y_t &=
% \begin{cases}
% y_H \text{ if } s = H \\
% y_L \text{ if } s = L
% \end{cases} 
\end{align}
where $p(w_t)$ captures the price per unit of water and may change with water use according to the tariff structure.  The price of other goods, $x_t$, is normalized to one.  $y_t$ represents household income each period which takes a value of $(1+\theta)\bar{y}$ with probability $\pi \in (0,1)$ and a value of $(1-\theta)\bar{y}$ with probability $(1-\pi)$ where $\theta  \in (0,1)$.  At the beginning of each period, households can choose whether to temporarily disconnect from water service $D_{t+1}=1$ or remain connected $D_{t+1}=0$.  If households disconnect or remain disconnected, they pay a fixed cost $f$.  Since temporarily disconnected households are likely to share with neighbors or purchase from local vendors who resell piped water, they are assumed to face the same price schedule as connected households.  Temporary disconnection allows households to delay paying their outstanding water balance, especially after receiving a delinquency visit as described in more detail below.  

By avoiding payment of water bills, households are able to borrow against their future income.  $S_t$ captures total assets chosen in the previous period to be consumed in period $t$ while $S_{t+1}$ captures total assets set aside in period $t$ for consumption in $t+1$ %Total assets in each period can be expressed in terms of two types, $A_t$ and $B_t$ given by the following expression
\begin{align}
S_t &= A_t + B_t \\
S_{t+1} &=  \dfrac{A_{t+1}}{1+r^{a}_{t}}  + I_t \dfrac{B_{t+1}}{1+r^{b}_{t}} 
\end{align}
$A_t$ captures standard assets for borrowing and saving.  The real interest rate $r_a$ is assumed to take a value of $r_l$ when households are saving ($A_{t+1} \leq 0$) and a value of $r_h$ when households are borrowing ($A_{t+1} > 0$).  This wedge in interest rates is intended to capture an institutional context with underdeveloped financial markets where households face high borrowing costs from moneylenders as well as high transaction costs in loaning out their own savings.  This assumption is consistent with a recent literature in development economics that emphasizes how households often face institutional savings constraints.\footnote{In Kenya, \cite{dupas2013savings} and \cite{dupas2013don} estimate large willingness-to-pay for improved savings technologies.}  Households are unconstrained in their amount of borrowing or saving with this asset.  % possibly cite (\cite{karlan2009expanding}) here?

$B_t$ captures borrowing through unpaid water bills.  Households face a real interest rate $r_b$ for borrowing from water bills.  To prevent arbitrage cases where households borrow infinitely from water bills to invest in standard assets, $r_b$ is assumed to equal $r_h$ when households are saving through standard assets ($A_t>0$), and is equal to $r_w$ otherwise.  Since the water company is not allowed to charge interest on outstanding balances, $r_w$ is equal to zero in this setting.%This assumption is also consistent with some transaction costs in switching between asset types.

$I_t$ determines when households are able to borrow from their unpaid water bills and takes the following form
\begin{align}
I_t = D_{t+1} + (1-c_t) (1-D_t) (1-D_{t+1})
\end{align} %I_t = (D_{t+1} + c_t (1-D_{t+1})(1-D_t)) 
where $c_t$ indicates whether a household receives a delinquency visit, which occurs with probability $\lambda \in (0,1)$ in each period and $D_{t+1}$ indicates the household's disconnection choice given their disconnection status $D_t$.  If households receive a delinquency visit ($c_t=1$) and they choose to remain connected ($D_{t+1}=0$), then they must pay their full outstanding balance to remain connected, which means that households cannot continue borrowing against their unpaid water bill this period.  Similarly, if disconnected households ($D_t=1$) want to reconnect  ($D_{t+1}=0$), then they also need to pay their outstanding balance.  Choosing to disconnect ($D_{t+1}=1$) allows households to avoid paying their unpaid balances until they choose to reconnect.  

Given that households are able to borrow, the amount that they can borrow through unpaid bills is constrained as follows
\begin{align}\label{eq:borrowconstraint}
&B_t -  p(w_t) w_t (1-D_{t+1})(1+r_b) \leq B_{t+1} \leq 0 
\end{align}
By leaving bills unpaid, households are able to borrow up to their existing outstanding balance, $B_t$, plus their total bill from water consumption, $p(w_t) w_t$, given that they stay connected to service, $D_{t+1}=0$.  If households choose to disconnect, then they can maintain their existing outstanding balance, but cannot add to this balance through unpaid water use.  To capture the fact that water bills are almost never overpaid, households are assumed to be unable to save with this asset.

This framework creates four possible states depending on whether income, $y_t$, is high or low and whether households receive a delinquency visit given by the indicator variable, $c_t$.  Assuming that the probabilities of reaching each state are uncorrelated with each other yields the following transition matrix between states:


\resizebox{.85\linewidth}{!}{
  \begin{minipage}{\linewidth}
\begin{align}
\begin{split}
\label{eq:tmatrix}
T_{t,t+1} &= \bordermatrix{\text{} & (1+\theta)\bar{y}\,,\,\text{no visit} & (1-\theta)\bar{y}\,,\,\text{no visit} & (1+\theta)\bar{y}\,,\,\text{visit} & (1-\theta)\bar{y}\,,\,\text{visit} \cr
                (1+\theta)\bar{y}\,,\,\text{no visit} & \pi (1-\lambda) & (1-\pi) (1-\lambda) & \pi \lambda & (1-\pi) \lambda  \cr
                (1-\theta)\bar{y}\,,\,\text{no visit} & \pi (1-\lambda) & (1-\pi) (1-\lambda) & \pi \lambda & (1-\pi) \lambda  \cr
                (1+\theta)\bar{y}\,,\,\text{visit}    & \pi (1-\lambda) & (1-\pi) (1-\lambda) & \pi \lambda & (1-\pi) \lambda  \cr
                (1-\theta)\bar{y}\,,\,\text{visit}  & \pi (1-\lambda) & (1-\pi) (1-\lambda) & \pi \lambda &  (1-\pi) \lambda  }
\end{split}
\end{align}
  \end{minipage}
}


%\subsection{Solving the Model}

The problem can be reformulated using a recursive value function approach where the household solves for a function, $V(X_t,z_t)$, that maximizes both current utility and the expected continuation value given current asset-levels and disconnection status captured by $X_t$ as well as current states given by $z_t$ below
\begin{align}\label{eq:valmax}
\begin{split}
V(X_t,z_t) \,=\, &max_{x_t,w_t}  \,\,\, u(x_t,w_t) \,+\, (1+\delta)^{-1} E \Big[\, V(X_{t+1}|z_{t})\,\Big| z_{t+1}, T_{t,t+1} \Big]
\\
s.t.& \\
x_t & \, + \, p(w_t) w_t \, = \, y_t \, - D_{t+1} f  \, + \, S_t \, - \, S_{t+1} \\
-p&(w_t) w_t (1-D_{t+1}) \leq \frac{B_{t+1}-B_t}{(1+r_b)} \leq 0  \\
X_t &= [A_{t},B_{t},D_{t}] \\
z_t &= [y_t,c_t]
\end{split} 
\end{align}
This problem can be divided into two steps: (1) maximizing utility within a given period by choosing $w_t$ and $x_t$ holding asset levels fixed, and (2) choosing asset values to maximize utility over time.

Each period, households face a standard utility maximization problem except in cases where they need to overconsume water to raise enough revenue through unpaid water bills to meet their borrowing goals.  For example, households may drink tap water instead of soft drinks or other beverages in months where they are especially credit constrained.  Formally, this situation occurs when equation (\ref{eq:borrowconstraint}) is binding.  To solve this problem, let $L = \frac{B_{t+1} - B_t }{1+r^{b}_{t}}$ indicate the amount of revenue needed to be raised through water consumption to fund borrowing levels.  Likewise, let $Y = y_t  - D_{t+1} f   +  A_t + B_t  -  \frac{A_{t+1}}{1+r^{a}_{t}} - I_t \frac{B_t}{1+r^{b}_{t}}$ equal all other net income.  Also, let utility take a Cobb-Douglas shape in water and all other goods with preference parameter, $\alpha \in (0,1)$.  Cobb-Douglas preferences assume that households spend a constant share of their income on water and that the price-elasticity of demand for water is equal to one, which is close to recent estimates in the literature.\footnote{\cite{wjv} finds an average price elasticity of 0.84 in this setting while \cite{szabo2015value} finds an average price elasticity of 0.98 in South Africa.}  The Cobb-Douglas utility function is also assumed to take a log-log form, which determines how households smooth income over time.\footnote{CITE SOME LITERATURE HERE?!}

The price of water is parameterized as a linear function of water use, $p(w) = p_1 + p_2w$, to approximate the increasing block tariff present in Manila.  Suppressing most time subscripts for ease of exposition, the household maximization problem takes the following form within each period
\begin{align}\label{eq:hhmax}
\begin{split}
&max_{w_t,x_t} \,\,\,\, \alpha\, log(w) \, + \, (1-\alpha)\,log(x) \\
s.t.& \\
\,\,\,\, &(p_1 + p_2 w)\,w + x  =  Y - L \\
&(p_1 + p_2 w)\,w (1-D_{t+1}) \leq L  
\end{split}
\end{align}
Optimal consumption takes a piecewise form depending on whether households have to use more water to satisfy demand for borrowing each period
\begin{align}\label{eq:ws}
\begin{split}
w^{*} &= 
\begin{cases}
                                                                                                                                                                                                                                                                                                                                                             \frac{p_{1}-\sqrt{{p_{1}}^2-8\,L\,\alpha \,p_{2}+8\,Y\,\alpha \,p_{2}+4\,L\,\alpha ^2\,p_{2}-4\,Y\,\alpha ^2\,p_{2}}}{2\,p_{2}\,\left(\alpha -2\right)}
 &\text{ if } L \geq \widehat{L} \\
                                                                                                                                                                                                                                                                                                                                                                                                                                                                -\frac{p_{1}-\sqrt{{p_{1}}^2-4\,L\,p_{2}}}{2\,p_{2}}
 &\text{ if } L < \widehat{L}
\end{cases} \\
&\widehat{L} =                                                                                                                                                                                                                                                                                                                                 \frac{Y}{2\,\left(\alpha -1\right)}+\frac{\frac{Y\,p_{2}}{2}+\frac{{p_{1}}^2}{8}-\frac{p_{1}\,\sqrt{{p_{1}}^2-\alpha \,{p_{1}}^2+8\,Y\,\alpha \,p_{2}}}{8\,\sqrt{1-\alpha }}}{p_{2}}

\end{split}
\end{align}
where $\widehat{L}$ captures the point at which revenue demanded for borrowing is exactly generated by the household's optimal consumption.  When borrowing demand outpaces revenue from optimal consumption (ie. $L<\widehat{L}$), then households must deviate from their optimal consumption choice and instead consume enough water to exactly satisfy borrowing demand.  This overconsumption captures a possibly important inefficiency associated with using unpaid water bills as a source of credit.  Combining optimal consumption in equation (\ref{eq:ws}) with the budget constraint in equation (\ref{eq:bc}) and the utility function in equation (\ref{eq:hhmax}) provides an indirect utility function as a function of prices, income, and assets.  The full indirect utility function is given by equation (\ref{eq:vstar}) in Appendix~\ref{appendix:indirectutil}.

Given indirect utility in each period and a discount rate that is greater than the return on savings so that households do not save infinitely, $\delta> r_l$, households solve for a stationary value function in (\ref{eq:valmax}) that maximizes utility by mapping any combination of asset levels into period $t$ into future asset levels in period $t+1$.

\section{Estimation and Results}    % https://www.cgdev.org/blog/compartamos-context

The goal of the estimation strategy is to map variation in payment behavior, disconnection rates, and average monthly usage levels into estimates of income variation, preferences for water, and the degree of credit constraints faced by households.  

Table~\ref{table:calibratedparam} describes parameters that assumed or calibrated prior to estimation.  The monthly savings interest rate is calibrated to the prevailing interest rate in the Philippines over this time period.  The estimation uses a monthly discount rate of 2\%, which implies an annual discount rate of 26.8\% and falls in the range of recent structural and experimental estimates.\footnote{\cite{andreoni2012estimating} estimate rates between 25\% and 35\% in an experimental setting and confirm exponential discounting.  \cite{laibson2007estimating} use a similar consumption-savings structural approach and recover a discount rate of around 15\%.  \cite{gourinchas2002consumption} use a similar structural approach finding a lower discount rate of around 5\%.}  Since the discount rate of 2\% exceeds the calibrated savings interest rate of 0.3\%, households are able to solve for a well-defined value function.

To measure prices, the highly non-linear tariff structure is approximated by a linear function of monthly usage as described in more detail in Appendix~\ref{appendix:tariff}.  This simplification allows for computational tractability within this dynamic model.\footnote{\cite{wjv} and \cite{szabo2015value} carefully capture non-linear pricing incentives with static models that are very computationally expensive.}  This approach also parallels average pricing models of consumer demand for utilities as suggested by \cite{ito2014consumers}.  

Without detailed household income data in the water connection survey, the estimation includes average income in Manila as measured by the Family Income and Expenditure Survey.  This method may overestimate true average household income because the estimation excludes shared water connections, which tend to used by poorer households.  At the same time, this approximation may underestimate true household income to the extent that the Family Income and Expenditure Survey records households that are unconnected to piped water and tend to be significantly poorer.\footnote{The 2010 Census of Population and Housing finds around 5 to 6\% of households were unconnected to piped water (\cite{wjv}).}  While the size of monthly income shocks are estimated, households are assumed to face equal probabilities of high and low income shocks in any particular month.

\begin{table}[H]
\centering
\caption{Calibrated and Assumed Parameters}\label{table:calibratedparam}
\vspace{-2mm}
\resizebox{\columnwidth}{!}{%
\begin{tabular}{l*{1}{ccl}}
\toprule
%Parameter  &   &  Value & Source \\
%\midrule
Savings Interest Rate & $r_l$ & \input{tables/irate}\unskip\%  & {\footnotesize Philippines data from World Bank Databank (2010-2015) } \\
Water Interest Rate & $r_w$ & 0\% & {\footnotesize Regulators prevent charging interest } \\
Discount Rate & $\delta$ & 2\% & {\footnotesize Mean of structural estimates from 
$\text{literature}^{\dagger}$} \\
Tariff    & $(p_1 + p_2 w)$ & $(\input{tables/p_int} + \input{tables/p_slope}w)$ & {\footnotesize Estimated price by water usage  (See Appendix~\ref{appendix:tariff} for details) } \\
Mean Inc. (PhP) & $\bar{y}$ & \input{tables/y_avg} & {\footnotesize Family Income Expenditure Survey (2015) for Manila} \\
High Inc. Risk& $\pi$ & 50\% & {\footnotesize Assumed to ensure symmetric income shocks} \\
Visit Risk & $\lambda$ & \input{tables/prob_caught}\unskip\% & {\footnotesize \% of months with a visit among stayers with $>$31 days overdue} \\
\bottomrule
\multicolumn{4}{l}{\scriptsize All measures are monthly.  Annual rates are converted to monthly rates as follows: Monthly Rate = $(1+\text{Annual Rate})^{1/12}-1$} \\[-.5em]
\multicolumn{4}{l}{\scriptsize $\text{}^{\dagger}$ See \cite{andreoni2012estimating}, \cite{laibson2007estimating}, and \cite{gourinchas2002consumption} for structural $\delta$ estimates.}
\end{tabular}
}
\end{table}

The last term needed for estimation is the risk of receiving a delinquency visit each month.  Since key parameters are identified by the share of households that disconnect in response to delinquency visits, the estimation sample is limited to stayers and risk of delinquency visits is calculated for this sample.  Table~\ref{table:descriptives_stayers} in Appendix~\ref{appendix:stayerdescriptives} includes detailed descriptive statistics for stayer households and Figure~\ref{figure:dc_hazardstayers} provides the probability of receiving a delinquency visit conditional on the days delinquent in the previous month.  Focusing on stayers may limit the generalizability of the results, especially given that stayers differ demographically from the population of households as shown in Table~\ref{table:descriptives_3g}.   % M = (1 + A)^(1/12) - 1 % A = (1 + M)^12 - 1

Given assumed and calibrated parameters from Table~\ref{table:calibratedparam}, the estimation strategy is able to recover parameters described in Table~\ref{table:estimparam} using moments in the data.  Due to computational limitations, the estimation assumes a representative household and therefore, recovers a single set of parameters that apply to behavior for all households.  Average consumption primarily identifies household water preferences.  By assuming Cobb-Douglas preferences, identification rests on the assumption that household price elasticity is equal to one so that the preference parameter, $\alpha$, can be interpreted as the constant share of the budget that households use on water.

Since surveys do not measure monthly household income variation in Manila, the estimation strategy is designed to recover the size of monthly income variation that rationalizes the amount of unpaid water bills observed in the data.  Under the structure of the model, greater income variation increases household demand for credit, which in turn increases the amount of unpaid water bills.  This approach assumes that households choose not to pay their water bills in order to smooth their consumption over time.  This assumption may not be valid in cases where households pay their bills infrequently because they face fixed travel or other hassle costs in making each payment.  As discussed in Section~\ref{section:descriptives}, households have a variety of payment options available to them including through local convenience stores, online, and over the phone, which suggests that the fixed costs of making each payment may be small. % ARE COMPLAINTS ALSO ZERO!?

The fixed cost of disconnecting and using from an alternative water source is recovered from the share of households that disconnect in response to a delinquency visit.  Disconnecting allows households to avoid immediately paying their outstanding water balances.  Therefore, high disconnection rates suggest low fixed costs of being disconnected.

The borrowing rate from standard assets is identified from differential disconnections rates in response to delinquency visits for households over and under 90 days overdue at the time of a visit.  This identification strategy leverages the intuition that households with large debts at the time of a visit would require large loans from standard assets in order to remain connected.  Therefore, high borrowing rates may disproportionately drive these households to disconnect in response to a visit.  This identification strategy requires the assumption that households over and under 90 days overdue at the time of a visit face equal fixed costs of remaining disconnected.  For example, if more delinquent households have better outside options for water, then this approach would wrongfully attribute their high disconnection rates to high borrowing rates.  Also for this approach to be valid, households must be unable to predict the exact timing of delinquency visits.  While Appendix~\ref{table:tcd_predict} provides some evidence that the timing of delinquency visits is correlated with demographics and payment behavior, delinquency visits are relatively rare events, and only \input{tables/disc_notice}\unskip\% of disconnected households reported receiving advanced warning from the water company.
%\footnote{TEST THE ASSUMPTION THAT OUTSIDE OPTIONS ARE THE SAME PLEASE?!}

\begin{table}[H]
\centering
\caption{Parameters to be Estimated}\label{table:estimparam}
\vspace{-2mm}
%\resizebox{\columnwidth}{!}{%
\begin{tabular}{l*{1}{cl}}
\toprule
Parameters  &   & Main Identifying Moments  \\
\midrule
Water Preference  & $\alpha$ & Mean Usage \\[.1em]
Income Shock Magnitude & $\theta$ & Mean Outstanding Balance  \\[.1em]
Fixed Cost of being Disconnected  & $f$ &  \% Disconnected 1-4 months post visit \\[.1em]
Borrowing Rate from Standard Assets & $r_h$ & \% Disconnected 1-4 months post visit \\
 & & given 90+ days overdue when visited \\
\bottomrule
%\multicolumn{3}{l}{\scriptsize All rates are monthly.  Annual rates are converted to monthly as follows: Monthly Rate = $(1+\text{Annual Rate})^{1/12}-1$} 
\end{tabular}
%}
\end{table}

The estimation routine solves the household's problem in equation (\ref{eq:valmax}) through value function iteration over across a grid of asset values.  Households can choose over \input{tables/par_nA}values of the standard asset, $A_{t+1}$, evenly spaced across a normal distribution with a standard deviation of \input{tables/par_sigA}\unskip, a minimum of \input{tables/par_Amin}\unskip, and a maximum of \input{tables/par_Amax}\unskip.  Households can also choose how much to borrow from unpaid water bills, $B_{t+1}$, over \input{tables/par_nB}values evenly spaced across a normal distribution truncated above at 0 with a standard deviation of \input{tables/par_sigB} and a minimum of \input{tables/par_Bmin}\unskip.  The additional choice of whether to stay connected each period, $D_{t+1}$, brings the total possible number of asset combinations to \input{tables/par_totalsize}\unskip.  

The estimation strategy uses a simulated method of moments approach, which chooses parameters to minimize the sum of squared distances between simulated and true moments, weighted by their average values in the data.  To generate simulated moments, the estimator creates a random \input{tables/par_n_iter}month chain of states according to the transition matrix (equation (\ref{eq:tmatrix})) and calculates the household's predicted asset and consumption choices across these states (assuming asset levels of zero to start).\footnote{See \cite{laibson2007estimating} and \cite{gourinchas2002consumption} for similar approaches to estimation.}

Table~\ref{table:estimates} provides the estimation results.  The Cobb-Douglas preference for water consumption is estimated to be \input{tables/est_alpha}\unskip, which is consistent with households' average budget share dedicated to water in Manila.  The estimated income shock of \input{tables/est_theta}implies that household incomes either increase or decrease by \input{tables/est_theta_per}\unskip\% of average income with 50\% probability each month.  This estimate can also be interpreted as measuring the coefficient of variation (CV) of income and falls on the lower end of previous estimates in the literature.\footnote{The coefficient of variation (CV) measures the standard deviation of monthly household income divided by average household income (\cite{hannagan2015income}).}  \cite{hannagan2015income} use monthly financial diaries in the US to calculate CVs of 0.39 for average households and 0.55 for households below the poverty line.  Using household surveys from Mexico, \cite{amuedo2011remittances} calculate CVs between 0.29 and 0.46, which more closely resemble the estimate in Table~\ref{table:estimates}.

The estimation recovers a fixed cost of being disconnected of \input{tables/est_fc}PhP/month.  Previous research uses a static, structural approach to estimate a long-term monthly fixed cost from using alternative water sources of 130 PhP/month (\cite{wjv}).  While these estimates fall in a similar range, this paper produces a larger estimate of the fixed-cost likely because being suddenly disconnected from piped water leaves little time for households to search for or invest in low-cost alternative sources for water.

The borrowing rate from standard assets is estimated to be \input{tables/est_irate_per}\unskip\% per month, which implies an annual interest rate of \input{tables/est_irate_annual_per}\unskip\%.  This estimate is substantially lower than the 20\% per month interest rate that \cite{karlan2009expanding} document as being commonly charged by moneylenders in Manila.  Despite this high interest rate, \cite{karlan2009expanding} document that at least 30\% of their sample of microentrepreneurs report taking credit from moneylenders at these rates.  The estimated borrowing rate of \input{tables/est_irate_per}\unskip\% is more similar to microloans of 1,000 PhP at 2.5\% monthly interest offered to rural Filipino households as part of a microfinance experiment conducted by \cite{gine2014group}.\footnote{The annual interest rate of \input{tables/est_irate_annual_per}\unskip\% is well exceeds than the average 13 to 25\% range offered by microfinance providers worldwide surveyed by \cite{cull2009microfinance}.  Two possible reasons for this discrepancy are that (1) institutional reasons unique to Manila may limit lenders' abilities to offer low rates and (2) subsidies may allow many microfinance providers to offer below-market interest rates.}

\begin{table}[h!]
\centering
\caption{Estimates}\label{table:estimates}
\vspace{-2mm}
%\resizebox{\columnwidth}{!}{%
\begin{tabular}{l*{1}{cc}}
\toprule
Parameters  &   & Estimates \\
\midrule
Water Preference & $\alpha$ & \input{tables/est_alpha} \\
 &  & (\input{tables/est_sd_alpha}\unskip) \\[.4em]
Income Shock Magnitude & $\theta$ & \input{tables/est_theta} \\
 &  & (\input{tables/est_sd_theta}\unskip) \\[.4em]
Fixed Cost of being Disconnected (PhP) & $f$ &  \input{tables/est_fc} \\
 &  &  (\input{tables/est_sd_fc}\unskip) \\[.4em]
Borrowing Rate from Standard Assets & $r_h$ & \input{tables/est_irate} \\
 &  & (\input{tables/est_sd_irate}\unskip) \\[.8em]
Households & & \input{tables/est_hhs} \\
Household-Months & & 2,118,861 \\
\bottomrule
\multicolumn{3}{l}{\scriptsize Standard errors in parentheses are bootstrapped at the household-level.} % with \input{tables/breps}repetitions 
\end{tabular}
%}
\end{table}

Table~\ref{table:fit} provides both moments in the data used for estimation as well as other moments to help evaluate model fit.  While the model is able to almost exactly match average usage and outstanding balance, the model has more difficulty matching the slow decline in disconnection rates observed after delinquency visits.  The model instead predicts that households who disconnect in response to a delinquency visit will quickly reconnect over the following four months.  One explanation for this discrepancy may be that the distribution of income shocks does not allow for serial correlation so that households quickly recover from negative shocks.  In reality, households may be disconnecting in response to longer-term negative income shocks like job loss or illness.  A similar pattern exists for disconnection conditional being over 90 days overdue when visited.


\begin{table}[H]
\centering
\caption{Model Fit}\label{table:fit}

\begin{tabular}{l*{1}{cc}}
\toprule
 Moments & Data  & Predicted \\[.5em]
Used in Estimation & & \\
\midrule
Mean Usage (m3) &26.20&26.58\\
Mean Outstanding Balance (PhP) &2415.8&2450.0\\
[.5em]
\% Disconnected Post-Visit  & & \\[.2em]
1 month &0.13&0.12\\
2 months &0.14&0.10\\
3 months  &0.12&0.06\\
4 months &0.10&0.05\\[.8em]
\% Disconnected Post-Visit & & \\
given 90+ days overdue when visited & & \\[.2em]
1 month &0.30&0.28\\
2 months &0.32&0.23\\
3 months &0.26&0.13\\
4 months &0.23&0.11\\
& & \\
Unused in Estimation & & \\
\midrule
SD Usage &12.3&2.4\\
SD Outstanding Balance  &3588.7&2634.7\\
Corr. Usage and Out. Bal. &0.31&-0.02\\

\bottomrule
\multicolumn{3}{l}{\scriptsize ``SD'' stands for standard deviation and ``Corr.'' stands for correlation. }  \\[-.5em]
\multicolumn{3}{l}{\scriptsize ``Out. Bal.'' stands for outstanding balance. Standard deviations in the data }  \\[-.5em]
\multicolumn{3}{l}{\scriptsize are calculated with variation within households.}
\end{tabular}
%\begin{tabular}{lcc}
& Data & Estimated \\
Mean Usage (m3) &24.9&24.6\\
SD Usage &11.1&2.3\\
Mean Water Debt (PhP) &1205&1412\\
SD Water Debt (PhP) &1258&1955\\
Corr. Usage and Water Debt &0.35&0.05\\
Mean Disc. for 1 month &0.061&0.073\\
Mean Disc. for 2 months &0.061&0.047\\
Mean Disc. for 3 months &0.048&0.030\\
Mean Disc. for 4 months &0.040&0.017\\
\end{tabular} 

\end{table}

The model has difficultly matching moments that were not used in the estimation.  Since log-utility encourages households to smooth their consumption over time, this model predicts very little variation in usage levels.  By contrast, high observed variation in usage is likely driven by the fact that households are likely to face idiosyncratic shocks to their water demand each month as household members travel for work, other families come to visit, or Manila experiences a heat wave.  A more complete model may include a term for indiosyncratic water shocks although it is unclear whether these shocks would substantively affect the model's predictions over income smoothing across time.  While having difficulty matching usage variation, the model is able to generate over half of the observed variation in outstanding balances.  

In terms of the correlation between usage and outstanding balances, the data find a positive relationship, which suggests that households may take on water debt to fund extra consumption in months where they face large, positive shocks to their water demand.  In the model, positive income shocks reduce demand for water debt and increase demand for water. Negative income shocks reduce demand for water while increasing demand for water debt.  Therefore, in some cases, households use more water during negative income shocks in order to increase their borrowing limits.  On net, the model finds zero average correlation between outstanding balances and usage.    

\begin{figure}[H]
\centering
\caption{100 Months of Simulated Data around \\ a Period of Disconnection}\label{figure:deaton}
\includegraphics[scale=1.1]{tables/new_deaton_graph.pdf} \\
{\scriptsize  Note: 100 months are chosen to center around the first disconnection event in the \input{tables/par_n_iter}month random sequence of states used in estimation.  ``Visit'' indicates months with a delinquency visit with a diamond.  ``Disconnect'' indicates months disconnected with a thick line.  Cum. $\Delta y_t$ measures cumulative shocks to income in PHP, $A_{t+1}$ indicates the optimal standard asset position in PhP, $B_{t+1}$ indicates the optimal water borrowing in PhP, and $\text{Usage}_t$ indicates water consumption in m3.}
\end{figure}


To build intuition, Figure~\ref{figure:deaton} provides 100 time periods of simulated data from the model.  These 100 time periods are chosen to center around the first disconnection occurrence in the \input{tables/par_n_iter}month random sequence of states used in the estimation.  The first panel in Figure~\ref{figure:deaton} indicates the cumulative, exogenous income shocks faced by the household.  This sample features an extreme period of income loss with negative shocks greatly outweighing positive shocks to income.  Positive shocks only begin to outweigh negative shocks at around 50 months.  Indicators for when households receive delinquency visits as well as whether they choose to remain disconnected are also nested in this first panel.  Over the course of 100 months, the household experiences three visits, the second of which leads the household to disconnect for around 12 months.  This disconnection corresponds to the period where the household has accumulated the least income.

The second panel indicates the household's choice of asset position, $A_{t+1}$, in each month.  Asset position closely tracks income realizations as households increasingly borrow (moving into very negative positions) following a long series of negative income shocks.  At around the time of disconnection, the household chooses to borrow the maximum allowed by the grid of assets chosen for the simulation.  Positive income shocks then allow the household to borrow less and begin saving at around 60 months.

The third panel indicates the household's borrowing through unpaid water bills, $B_{t+1}$.  The household increases its borrowing more slowly than with standard assets since each month's total borrowing is limited by the household's current water bill.  Matching the downturn in income, households continue borrowing before quickly reaching the maximum borrowing allowed by the grid of assets chosen for the simulation.  With few positive income shocks, households remain at this maximum borrowing level for at least 24 months.  When the second delinquency visit occurs around 40 months, households are still borrowing the maximum from unpaid water bills and therefore, instead of choosing to pay their outstanding balance, these households choose to disconnect until they the receive enough positive income shocks to pay their full water bill and reconnect to service around month 55.  During the third and last delinquency visit, the household's outstanding balance happens to be relatively small so the household pays the balance to remain connected.

The fourth panel indicates households water usage patterns over the same 100 months.  Usage begins to spike as the household increases it usage to fund borrowing through unpaid water bills.  The largest spikes in usage occur when the household moves to the maximum level of borrowing allowed by the grid of assets chosen for the simulation.  Because of the step-size chosen for the asset grid, moving to the largest borrowing level requires a jump in unpaid bills of around 1,000 PhP.  Since the average bill is around 600 PhP, the household needs to almost double its usage to fund this jump in unpaid borrowing.  These spikes in usage measure the extent to which borrowing from unpaid water bills may distort water usage choices, adding an additional friction associated with borrowing from water bills.  After maximizing water borrowing at around 24 months, usage begins to stabilize at lower levels, mirroring the long string of negative income shocks faced by the household.  

\section{Counterfactual Policies}

To measure how much households value credit from unpaid water bills, the model is able to examine household welfare in a counterfactual setting where households are unable to borrow from their water bills.  In practice, this counterfactual setting may correspond to a policy of strict enforcement by the water company (ie. monthly delinquency visits).  Table~\ref{table:counter} includes outcomes for the current setting in Manila in Column (1) and for a counterfactual setting without water borrowing in Column (2).  The first row calculates compensating variation equal to \input{tables/U_nl_abs}PhP/month associated with losing access to water credit.  This estimate suggests that households would require at least \input{tables/U_nl_abs}PhP/month (or about 1 USD) greater income to eliminate access to water credit.  Given an average water bill of \input{tables/bill_all}PhP/month, this estimate would translate into households paying around \input{tables/U_nl_per}\unskip\% smaller bills each month.  

Eliminating credit access also decreases mean usage by \input{tables/c_h_nl_drop}\unskip\% as shown by columns (1) and (2) in the second row of Table~\ref{table:counter}.  Reducing credit access lowers the extent to which households can smooth their consumption over time while also removing the incentive for households to overconsume water in order to finance water borrowing.  Given that households spend around \input{tables/bill_inc_all}\unskip\% of their income on water, this estimate is roughly proportional with similar evidence from South Africa where restricting credit access with prepaid electricity meters produced a 13\% reduction in usage and where households spend around 8-10\% of their income on electricity (\cite{jack2016charging}).\footnote{\cite{jack2016charging} also propose other mechanisms that may account for reductions in usage such as transaction costs and intra-household bargaining constraints.}

\begin{table}[H]
\centering
\caption{Counterfactual Policies}\label{table:counter}
\begin{tabular}{l*{1}{ccc}}
\toprule
 & (1) & (2) & (3)  \\
 & Current & No Water  & Prepaid  \\
 &         &     Borrowing               & Metering  \\
\midrule   
Compensating Variation (PhP) & & \input{tables/U_nl}  & \input{tables/U_pp} \\
Mean Usage (m3) & \input{tables/c_h} & \input{tables/c_nl} & \input{tables/c_pp} \\
 &         &                    &  \\
%Adjustments to stay revenue neutral  & & \\
Price Intercept $p_1$ (PhP/m3) & \input{tables/p_int} &  & \input{tables/p_int_pp} \\
Disconnection Rebate (PhP) & \input{tables/delinquency_cost} &  & 0 \\
%Price Intercept (PhP)  &20.2& &26.6\\

%\input{tables/counter_fixed_nomid}
\bottomrule
\multicolumn{4}{l}{ \scriptsize  All values are monthly. The price intercept increases under prepaid metering to cover meter replacement } \\[-.5em]
\multicolumn{4}{l}{ \scriptsize  costs.  The disconnection rebate measures the average outstanding balance left unpaid by permanently } \\[-.5em]
\multicolumn{4}{l}{ \scriptsize  disconnected households in terms of household-months.  Price intercepts and disconnection rebates are } \\[-.5em]
\multicolumn{4}{l}{ \scriptsize  unchanged for the no water borrowing counterfactual.}
\end{tabular}
\end{table}


\subsection{Prepaid Metering}

This structural model also provides a useful opportunity to evaluate the total welfare effects of implementing prepaid metering technologies, which have become increasingly popular for both electricity and water utilities throughout the developing world.\footnote{See \cite{jack2016charging} and \cite{northeast2014} for electricity utilities and \cite{heymans2014limits} for water utilities.}  By requiring households to pay upfront for their water usage, these meters eliminate any access to credit through unpaid water bills.  

Prepaid meters also provide potential savings to the water company by preventing households from permanently disconnecting and leaving large outstanding balances that are never paid.  On average, \input{tables/dc_per_month_account}households permanently disconnect per household-month.  These households that permanently disconnect leave average outstanding balances of \input{tables/out_bal}PhP.  These estimates imply household savings equal to an average of \input{tables/delinquency_cost}PhP/month per household.  In practice, households enjoy all of these savings in the final few months that they remain connected.  However, since households use water indefinitely in the model, the counterfactual exercise captures these savings by assuming that households receive a monthly fixed rebate of \input{tables/delinquency_cost}PhP.  By allowing households to spread these savings over time, this assumption is likely to overstate the true benefits to households from leaving unpaid bills.  Also, by assuming that households receive fixed rebates, this approach ignores any incentives that households may face to overconsume in their final months connected (since households face an effective price of zero in these months).  Appendix~\ref{appendix:permanentdc} finds some evidence of overconsumption before permanent disconnection.  This counterfactual exercise also assumes that prepaid meters do not affect households decision's over when and whether to permanently disconnect from service.  With prepaid meters, the water company would also no longer need to conduct delinquency visits.  Appendix~\ref{appendix:visitcosts} provides evidence that these cost savings are likely to be negligible.

The welfare effects of installing prepaid meters also depend on the costs of purchasing and installing this new technology.  \cite{heymans2014limits} surveyed eight large water providers that implemented prepaid meters in developing countries and found that each prepaid meter costs about four times as much as a standard meter and requires replacement every 7 years.  In the context of Manila, each standard meter costs around 1,500 PhP and is replaced around every 6 years and 3 months, bringing the monthly cost to 20 PhP/month.\footnote{The water company provided additional documentation of costs and frequency of meter replacement for residential households.}  Assuming that a prepaid meter costs 4 times as much as a standard meter with a replacement rate of 7 years, the estimated monthly cost of a prepaid meter would be 71 PhP/month.  Therefore, prepaid meters imply an additional cost of 51 PhP/month per household.\footnote{\cite{heymans2014limits} also report that the fixed administrative costs of installing and monitoring new meters account for less than 4\% of the total costs of switching to prepaid meters while the bulk of the expenses come from purchasing new meters.  By focusing on meter replacement, this exercise is likely to capture the majority of switching costs associated with prepaid metering.}  Moving to prepaid meters may also affect total water consumption, which would in turn affect the water company's costs of water production.  The water company reports a marginal cost of 5 PhP/m3.  

The regulatory structure in Manila as well as many other developing cities ensures that prices for water are regulated to exactly cover all production costs (\cite{hoque2013state}).  To simulate this regulatory environment, the counterfactual exercise calculates the changes in prices necessary to ensure that the water company remains revenue neutral after installing prepaid water meters.  This exercise also assumes that the government regulator is able to perfectly forecast water demand among all households.  Since prices are non-linear, only the price intercept, $p_1$, which applies to all levels of usage, is adjusted in this exercise.  Savings from permanent disconnections of \input{tables/delinquency_cost}PhP per household-month are exceeded by installation costs of 51 PhP per household-month, meaning that prices need to raise enough revenue to cover \input{tables/del_raised}PhP per household-month.  Counterfactual prices must also account for changes in usage so that they are able to exactly fund any marginal costs of water production.

Table~\ref{table:counter} includes the results of the prepaid metering counterfactual in column (3).  To cover the higher costs of prepaid meters, the price intercept, $p_1$, increases by around \input{tables/p_increase_per}\unskip\% as indicated by the third row.  At the same time, households no longer receive a disconnection rebate of \input{tables/delinquency_cost}PhP per month.  With higher prices and without water credit, households lower their consumption by \input{tables/c_h_pp_drop}\unskip\% under prepaid metering.  In order to be indifferent between the current setting and a world with prepaid metering, households would need to receive an additional \input{tables/U_pp_abs}PhP/month in compensation.  Taken together, these results provide suggestive evidence that prepaid metering would be welfare reducing in this context.

%%% also include counterfactual where we increase the interest rate and see substitution? may have to rewrite some of the identification section....

\section{Conclusion}







\pagebreak

\section{Appendix}


\subsection{Predicting Delinquency Visits}

\begin{table}[H]
\small
\centering
\caption{ Linear Probability of Receiving a Delinquency Visit }\label{table:tcd_predict}
\vspace{-2mm}
\begin{tabular}{lCCC}
\toprule
& \small (1) & \small (2) & \small (3)  \\
\midrule 
Usage t-1           &  -0.0000150\textsuperscript{a}&  -0.0000186\textsuperscript{a}&  -0.0000225\textsuperscript{a}\\
                    & (0.0000052)                   & (0.0000052)                   & (0.0000079)                   \\[0.5em]
Days Delinquent t-1 &   0.0000524\textsuperscript{a}&   0.0000581\textsuperscript{a}&   0.0000499\textsuperscript{a}\\
                    & (0.0000016)                   & (0.0000016)                   & (0.0000020)                   \\[0.5em]
Unpaid Balance t-1  &   0.0000009\textsuperscript{a}&   0.0000009\textsuperscript{a}&   0.0000011\textsuperscript{a}\\
                    & (0.0000001)                   & (0.0000001)                   & (0.0000001)                   \\[0.5em]
Single House        &  -0.0001480                   &  -0.0001468                   &                               \\
                    & (0.0002032)                   & (0.0002050)                   &                               \\[0.5em]
Apartment           &  -0.0004173\textsuperscript{b}&  -0.0005555\textsuperscript{a}&                               \\
                    & (0.0001861)                   & (0.0001869)                   &                               \\[0.5em]
Age of HoH          &  -0.0000398\textsuperscript{a}&  -0.0000373\textsuperscript{a}&                               \\
                    & (0.0000040)                   & (0.0000040)                   &                               \\[0.5em]
HoH Low Skill Empl. &   0.0005415\textsuperscript{a}&   0.0005333\textsuperscript{a}&                               \\
                    & (0.0001827)                   & (0.0001830)                   &                               \\[0.5em]
HH Size             &   0.0003462\textsuperscript{a}&   0.0003482\textsuperscript{a}&                               \\
                    & (0.0000363)                   & (0.0000364)                   &                               \\[0.5em]
Employed HH Members &  -0.0002883\textsuperscript{a}&  -0.0002957\textsuperscript{a}&                               \\
                    & (0.0000573)                   & (0.0000574)                   &                               \\[0.5em]
Location            &                               &  \checkmark                   &                               \\
Year \tim Month \textsc{FE}&                               &  \checkmark                   &  \checkmark                   \\
Household \textsc{FE}&                               &                               &  \checkmark                   \\
N                   &   1,951,543                   &   1,948,783                   &   1,951,543                   \\
Mean Visits Per Month&      0.0072                   &      0.0072                   &      0.0072                   \\

\bottomrule
\multicolumn{4}{l}{\footnotesize Std. errors clustered at the HH-level. \textsuperscript{c} p$<$0.10,\textsuperscript{b} p$<$0.05,\textsuperscript{a} p$<$0.01 }
\end{tabular}
\end{table}


\subsection{Sample Construction}\label{appendix:sampleconstruction}

Merging the full sample from the connection survey to the billing data yields an initial population of 3,343,644connection-months as described in Table~\ref{table:sampleconstruction}.  Non-residential accounts are first removed to ensure that results apply to household-level decisions.  Due to some data inconsistencies, payment records are missing for some connections, which are excluded.  Due to leaks, meter replacements, and meter reading errors, connections occasionally experience extremely high meter readings and bills.  Consumption records above 200 m3 as well as bills and payments above 80,000 are censored to address these issues.  Large negative payments and outstanding balances (due to reimbursements of billing errors) are also excluded due to likely measurement error.  Households that connect during the sample or have large stretches of missing records are excluded by including only connections with over 30 months of data.  Keeping only connections serving single households brings the final sample size to 2,073,884household-months.


\begin{table}[H]
\centering
\caption{Sample Construction}\label{table:sampleconstruction}
\vspace{-2mm}
\resizebox{\columnwidth}{!}{%
\begin{tabular}{l*{1}{cc}}
\toprule
 & Observations & Observations Removed  \\
\midrule
Initial sample   &         3,343,644  &     \\
Keep residential connections (excluding commercial)   &            &    \input{tables/class_drop}  \\
Keep connections with payment records &  &   \input{tables/pay_drop}   \\
Keep months with usage under 200 m3    &            &  \input{tables/chigh_drop}     \\
Keep bills $>$ -5,000 PhP and $<$ 80,000 PhP  &  & \input{tables/amount_drop}  \\
Keep unpaid bills $>$ -5,000 PhP and $<$ 80,000 PhP &             &     \input{tables/bal_drop}  \\
Keep payments $>$ -80,000 PhP and $<$ 80,000 PhP    &            &    \input{tables/paylh_drop}   \\
Keep connections with over 30 months of records   &            &   \input{tables/month_drop}    \\
Keep connections serving a single household   &            &   \input{tables/SHH_drop}    \\
%Drop connections that are disconnected for final yr.  &            &   \input{tables/dc_drop}    \\
Final sample & 2,073,884 & \\
\bottomrule
\end{tabular}
}
\end{table}



\subsection{Indirect Utility Function}\label{appendix:indirectutil}

\begin{align}\label{eq:vstar}
\begin{split}
v^{*} &= 
\begin{cases}
\alpha \,\ln(\frac{p_{1}-\sqrt{{p_{1}}^2-8\,L\,\alpha \,p_{2}+8\,Y\,\alpha \,p_{2}+4\,L\,\alpha ^2\,p_{2}-4\,Y\,\alpha ^2\,p_{2}}}{2\,p_{2}\,(\alpha -2)})- \\ \ln(\frac{(\alpha -1)\,(8\,L-8\,Y-4\,L\,\alpha +4\,Y\,\alpha )}{2\,{(\alpha -2)}^2}+ \\
\frac{(p_{1}\,\sqrt{{p_{1}}^2-8\,L\,\alpha \,p_{2}+8\,Y\,\alpha \,p_{2}+4\,L\,\alpha ^2\,p_{2}-4\,Y\,\alpha ^2\,p_{2}}-{p_{1}}^2)\,(\alpha -1)}{2\,p_{2}\,{(\alpha -2)}^2})\,(\alpha -1)\\
 &\text{ if } L \geq \widehat{L} \\
                                                                                                                                                                                                                                                                                                                                                                                             \alpha \,\ln\left(-\frac{p_{1}-\sqrt{{p_{1}}^2-4\,L\,p_{2}}}{2\,p_{2}}\right)-\ln\left(Y\right)\,\left(\alpha -1\right)
 &\text{ if } L < \widehat{L}
\end{cases} \\
&\widehat{L} =                                                                                                                                                                                                                                                                                                                                 \frac{Y}{2\,\left(\alpha -1\right)}+\frac{\frac{Y\,p_{2}}{2}+\frac{{p_{1}}^2}{8}-\frac{p_{1}\,\sqrt{{p_{1}}^2-\alpha \,{p_{1}}^2+8\,Y\,\alpha \,p_{2}}}{8\,\sqrt{1-\alpha }}}{p_{2}}

\end{split}
\end{align}



\subsection{Tariff Structure and Approximation}\label{appendix:tariff}

% \begin{figure}
% \caption{Example Residential Tariff Presented to Consumers}\label{figure:tarifftrue}
% \begin{center}
% \includegraphics[scale=.8]{tables/tariff_pic_png.png} \\
% \footnotesize{``conn.'' refers to connection, ``cu.m.'' refers to cubic meters of usage \\ per month, and \textbf{P} refers to Philippine Pesos.  Includes Tariff as of February, 2019.}
% \end{center}
% \end{figure}


\begin{table}[H]
\centering
\caption{Example Residential Tariff As Presented to Consumers}\label{table:tarifftrue}
\vspace{-2mm}
\begin{tabular}{l*{1}{r}}
\toprule
Usage (m3) & Price (PhP) \\
\midrule
Under  10    &   \input{tables/f_10}/conn. \\
Over  10     &   \input{tables/f_11}/conn. \\
Next   10    &   \input{tables/c_12}/cu.m. \\
Next    20   &   \input{tables/c_21}/cu.m. \\
Next   20    &   \input{tables/c_41}/cu.m. \\
Next    20   &   \input{tables/c_61}/cu.m. \\
Next    20   &   \input{tables/c_81}/cu.m. \\
Next    50   &   \input{tables/c_101}/cu.m. \\
Next   50    &   \input{tables/c_151}/cu.m. \\
Over  200    &   \input{tables/c_200}/cu.m. \\
\bottomrule
\multicolumn{2}{l}{\scriptsize Mean tariff 2010-2015 with value added tax. }\\[-.5em]
\multicolumn{2}{l}{\scriptsize ``conn.'' refers to connection. }\\[-.5em]
\multicolumn{2}{l}{\scriptsize ``cu.m.'' refers to m3/month.  50 PhP$\sim$1 USD }
\end{tabular}
\end{table}

Table~\ref{table:tarifftrue} provides the monthly tariff structure as it is presented to consumers.  Consumers face a fixed price as well as marginal prices for any usage above 10 m3.  The government regulator gradually adjusts prices at roughly yearly intervals in order to ensure that the water company is able to exactly cover its costs.  The marginal price is highly non-linear, accelerating quickly at low usage levels before slowly increasing at high usage levels.  To achieve a tractable approximation of this price schedule, Table~\ref{table:tcd_predict} fits a simple regression model predicting average price as a function of an intercept, $p_1$, and monthly usage levels, $p_2$.  This model predicts that a increase in monthly usage of 10 m3 results in an increase in average price of 2.2 PhP/m3.  

\begin{table}[H]
\small
\centering
\caption{Average Price and Monthly Usage}\label{table:tcd_predict}
\vspace{-2mm}
\begin{tabular}{lc}
\toprule
& \small Avg. Price: $\frac{\text{Bill (PhP)}}{\text{Usage (m3)}}$    \\
\midrule 
Usage (m3)          &        0.29\textsuperscript{a}\\
                    &      (0.00)                   \\[0.5em]
Intercept           &       17.57\textsuperscript{a}\\
                    &      (0.05)                   \\[0.5em]
Household-Months    &   1,861,200                   \\

\bottomrule
\multicolumn{2}{l}{\scriptsize \textsuperscript{c} p$<$0.10,\textsuperscript{b} p$<$0.05,\textsuperscript{a} p$<$0.01 }
\end{tabular}
\end{table}



\subsection{Stayer Descriptives}\label{appendix:stayerdescriptives}


\begin{table}[H]
\centering
\caption{Descriptives for Stayers}\label{table:descriptives_stayers}
\vspace{-2mm}
\begin{tabular}{l*{1}{cccccc}}
\toprule
 & Mean & SD & Min & 25th & 75th & Max  \\
\midrule
 Usage (m3)  & 26.2  & 17.5  & 0.0  & 15.0  & 33.0  & 200.0  \\ 
 Bill  & 761  & 1,124  & -4,640  & 287  & 920  & 78,409  \\ 
 Unpaid Balance  & 2,416  & 5,070  & -4,995  & 261  & 2,346  & 79,904  \\ 
 Share of Months with Payment  & 0.60  & 0.49  & 0.00  & 0.00  & 1.00  & 1.00  \\ 
 Payment Size  & 1,214  & 1,498  & 0  & 426  & 1,482  & 61,298  \\ 
 Days Delinquent  & 84.9  & 155.4  & 0.0  & 0.0  & 91.0  & 720.0  \\ 
 Delinquency Visits per HH  & 1.32  & 0.61  & 1.00  & 1.00  & 2.00  & 6.00  \\ 
 Share of Months Disconnected  & 0.03  & 0.17  & 0.00  & 0.00  & 0.00  & 1.00  \\ 

\bottomrule
\multicolumn{7}{c}{Total Households: \input{tables/total_hhs_stayers}  Obs. per Household: \input{tables/obs_per_hh_stayers} Total Obs.: \input{tables/total_obs_stayers}}
\end{tabular}
\end{table}

\begin{figure}[H]
\centering
\caption{Stayers Share of Households that Receive a Delinquency Visit \\ depending on Days Delinquent in the Previous Month}\label{figure:dc_hazardstayers}
\includegraphics[scale=.7]{tables/connected_visit_hazard.pdf} \\
{ \footnotesize Only stayer households. }
\end{figure}

% \subsection{Discussion of the Discount Rate}


\subsection{Usage Before Permanent Disconnection}\label{appendix:permanentdc}

Figure~\ref{figure:dc_permanent} plots average usage across households according to the number of months before these households permanently disconnect. Average consumption increases in the months leading up to permanent disconnection.  This increase is likely due to some households using water as if they faced a zero marginal price since they know that they will never pay their bills.  With an average bill of \input{tables/bill_all}PhP/month, leaving an outstanding balance of \input{tables/out_bal}PhP at permanent disconnection is equal to enjoying an average of around 18 months of free water consumption.

\begin{figure}[H]
\centering
\caption{Average Usage in Months Before Permanent Disconnection}\label{figure:dc_permanent}
\includegraphics[scale=.7]{tables/line_disc_graph.pdf} \\
{ \footnotesize Negative months indicate months before permanent disconnections. }
\end{figure}

\subsection{Cost of Conducting Delinquency Visits}\label{appendix:visitcosts}

Conversations with the water company suggest that travel costs compose the majority of the total costs for any service performed on a water meter.  Since the company requires a 200 PhP reconnection fee for disconnected households, I assume that delinquency visits cost the same amount to the water company.  Since disconnection visits occur in only \input{tables/tcd_id_mean}\unskip\% of household-month observations, the company is likely to spend a negligible 1.4 PhP/month to conduct delinquency visits.



% \subsection{Sampling of the Connection Survey}\label{appendix:pawssampling}

% Connection survey documentation indicates that surveyors used stratified sampling where the number of connections surveyed was proportional to the census population in each small administrative district (Barangay).  Surveyors randomly interviewed owners of connections until reaching these targets.  


% Although this approach ensures that connections are randomly sampled within areas, it may also lead to oversampling of areas with few connections.  To measure the extent of possible oversampling, the following regression expresses the share of surveyed connections in each Meter Reading Unit (MRU) --- the water company's smallest unit of geography including around 200 households per unit on average --- as a function of the demographics of households owning connections, $Z_{MRU,i}$, in each of these MRUs:

% To test this hypothesis, the following equation predicts the share of connections surveyed in each Meter Reading Unit (MRU) --- the smallest geographical unit used by the water provider including a couple hundred water connections on average --- according to demographics of households owning connections, $Z_{MRU,i}$, in each of these MRUs:
% \begin{align*}
% ShareSurveyed_{MRU} = \beta Z_{MRU,i} + \epsilon_{MRU,i}
% \end{align*}
% Table~\ref{table:pawssampling} provides the results for this regression.  Although many coefficients are statistically significant consistent with large sample sizes for this estimation, the magnitudes of the effects are small in economic terms across most measures.  For example, increasing the average number of households per connection by a standard deviation (0.65) results in a 0.0021 percentage point decrease in the probability of being surveyed; given a mean survey probability of 6 percentage points, this change represents only a 3.3 percent decrease.  Dwelling types are the strongest predictor of coverage, indicating that surveyors may have oversampled apartments possibly due to easier availability to survey.  Taken together, these results suggest that to the extent that non-random sampling may affect results, it will lead to an underestimation of the importance of sharing networks in Manila.

% \begin{table}[H]
% \centering
% \caption{ Predicting Share of Connections Surveyed with \\ Connection Owner Demographics }\label{table:pawssampling}
% \begin{tabular}{lc} \hline
 & (1) \\
VARIABLES & Percent Surveyed \\ \hline
 &  \\
Mean Consumption (m3) & 0.000286*** \\
 & (3.94e-05) \\
HHs per Connection & -0.00334*** \\
 & (0.000775) \\
HH Size & -0.000285 \\
 & (0.000190) \\
Age HoH & 0.000138*** \\
 & (2.27e-05) \\
Total Empl. & 4.04e-05 \\
 & (0.000326) \\
Low Skill Emp. & 0.00158* \\
 & (0.000909) \\
Apartment & 0.00704*** \\
 & (0.00157) \\
Single House & -0.0114*** \\
 & (0.00117) \\
Constant & 0.117*** \\
 & (0.00250) \\
 &  \\
Observations & 45,907 \\
R-squared & 0.018 \\
Mean Coverage & .06 \\
Std. Dev. Coverage & .06 \\
 Cluster MRU & Yes \\ \hline
\multicolumn{2}{c}{ Robust standard errors in parentheses} \\
\multicolumn{2}{c}{ *** p$<$0.01, ** p$<$0.05, * p$<$0.1} \\
\multicolumn{2}{c}{ 2,947 Meter Reading Units (MRUs)} \\
\end{tabular}

% \end{table}

\nocite{*}
\singlespacing
\setlength{\bibsep}{7pt}
\bibliographystyle{abbrvnat}
\bibliography{ref}



\end{document}


