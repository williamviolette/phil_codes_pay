\documentclass[12pt]{article}

\usepackage{geometry,setspace,palatino,enumitem,hyperref}
\geometry{left=1.1in,right=1.1in,top=1.0in,bottom=1.0in}
\setlength{\parskip}{6pt}
\setlength{\parindent}{0pt}

\hypersetup{colorlinks=true,linkcolor=blue,urlcolor=blue,citecolor=blue}

% Custom list styles
\newlist{refpoints}{itemize}{2}
\setlist[refpoints,1]{label=\textbullet, leftmargin=1.5em, itemsep=6pt, parsep=2pt}
\setlist[refpoints,2]{label=$\circ$, leftmargin=2em, itemsep=3pt, parsep=1pt}

% Command for referee quotes
\newcommand{\refquote}[1]{\textit{``#1''}}

\begin{document}

\begin{center}
{\Large \textbf{Response to Editor and Referees}} \\[0.5em]
{\large ``Microcredit from Delaying Bill Payments''} \\[0.3em]
William Violette \\[0.3em]
\today
\end{center}

\vspace{1em}

Dear Editor,

Thank you for the opportunity to revise and resubmit this paper. I am grateful for your thoughtful guidance and for the careful, constructive reports from both referees. Their comments have significantly improved the paper. Below, I respond to each point raised by the editor and referees, with referee comments in italics and my responses in sub-bullets.

\vspace{1em}
\hrule
\vspace{0.5em}
\section*{Editor Comments}

\begin{refpoints}

\item \refquote{Please do your best to complete the welfare comparison by modeling prepaid smoothing strategies.}
  \begin{refpoints}
  \item [Response here.]
  \end{refpoints}

\item \refquote{Consider broadening the policy space beyond ``current vs prepaid'' to incorporate features such as late fees (or the equivalent early-payment discount) instead of interest-free debt via the utility bill and/or a design in which a separate entity provides the loan while the utility focuses on water provision.}
  \begin{refpoints}
  \item [Response here.]
  \end{refpoints}

\item \refquote{Provide more institutional context and discussion of external validity (e.g., How common are shutoffs elsewhere? Are utilities typically allowed to charge interest on arrears? What is distinctive about Manila and what may generalize?).}
  \begin{refpoints}
  \item [Response here.]
  \end{refpoints}

\end{refpoints}

\vspace{1em}
\hrule
\vspace{0.5em}
\section*{Referee 1}

\textbf{Overview and recommendation:} Referee~1 recommends acceptance without revisions, describing the paper as ``well executed'' and ``a substantial contribution.'' We are grateful for this positive assessment. Below we address the substantive concern raised.

\vspace{0.5em}
\textbf{Main comment:}

\begin{refpoints}

\item \refquote{A greater concern of mine relates to what I feel is an incompleteness in the Welfare calculations with respect to prepaid and postpaid metering.}

  \begin{refpoints}
  \item \refquote{With prepaid meters, however, households can also consumption smooth in a more prospective fashion, through purchasing strategically ahead of time.}
    \begin{refpoints}
    \item [Response here.]
    \end{refpoints}
  \end{refpoints}

\item \refquote{Where price follows an inclining block tariff, as it does in this paper's setting in Manila, additional welfare improvements for households are accessible through prepaid metering, but not postpaid metering. Prepaid metered households are able to minimise their total expenditure on the utility over the period they are connected by purchasing their average monthly consumption\ldots This strategy minimises the total number of litres bought at the higher marginal price blocks of the inclining block schedule over the period that the household is connected.}
  \begin{refpoints}
  \item [Response here.]
  \end{refpoints}

\item \refquote{Just by purchasing ahead, households reduce the risk of receiving demand letters and disconnection notices, as well as reducing the risk of being disconnected if they plan to stay in the utility's service area. Demand letters and disconnection notices reduce utility and so do actual disconnections in the author's model\ldots Avoiding these improves welfare, all else equal.}
  \begin{refpoints}
  \item [Response here.]
  \end{refpoints}

\end{refpoints}

\vspace{1em}
\hrule
\vspace{0.5em}
\section*{Referee 2}

\textbf{General comment:} Referee~2 suggests the paper may be a better fit for a development economics journal.

\begin{refpoints}
\item \refquote{Given the research question, in my opinion, the paper would be a better fit for a development economics journal.}
  \begin{refpoints}
  \item [Response here.]
  \end{refpoints}
\end{refpoints}

\vspace{0.5em}
\textbf{Main issues:}

\begin{refpoints}

\item \textbf{1.1.~Considering all relevant effects.}

  \begin{refpoints}
  \item \refquote{I'm not sure whether the paper considers all relevant effects when making the policy conclusions. In this setting, the water utility company has two roles: providing interest-free loans and providing water. In my opinion, it is not a typical IO problem.}
    \begin{refpoints}
    \item [Response here.]
    \end{refpoints}

  \item \refquote{Access to clean water is a public health issue in developing countries. If the water company takes on dual roles, providing loans and water, it may face difficulties making the necessary investments to supply clean water to the entire population\ldots the paper currently does not account for the water company's investment decisions. In my opinion, it is likely that the current revenue is too low to support optimal investment.}
    \begin{refpoints}
    \item [Response here.]
    \end{refpoints}

  \item \refquote{If the government wants to provide interest-free loans, why do it via the water utility company? Why not create another entity that provides loans or build an institutional framework that allows the market to do so? In my opinion, for the counterfactual policy analysis, this implies that we shouldn't consider only an epsilon improvement, but also consider what is optimal.}
    \begin{refpoints}
    \item [Response here.]
    \end{refpoints}
  \end{refpoints}

\item \textbf{1.2.~Household discount rate.}

  \begin{refpoints}
  \item \refquote{The parameter for the household discount rate is assumed to be the same as found in the literature for the US. However, since the institutional details and behavior in Manila differ from those in the US, I believe it is important to use parameters that are estimated for developing countries.}
    \begin{refpoints}
    \item [Response here.]
    \end{refpoints}

  \item \refquote{It seems like a very strong assumption to assume that the annual discount rate in Manila is 6\%. This is reasonable in the US, where the annual interest rate could be considered 6\%. But in Manila, as the paper states, the annual interest rate for most households is about 200\% (9.5\% monthly rate). Perhaps it would be useful to look at discount rate estimates in the Philippines (Ashraf, Karlan, and Yin, 2006 QJE).}
    \begin{refpoints}
    \item [Response here.]
    \end{refpoints}
  \end{refpoints}

\end{refpoints}

\vspace{0.5em}
\textbf{Other comments:}

\begin{refpoints}

\item \textbf{2.1.~More details on the institution and comparison with other countries.}

  \begin{refpoints}
  \item \refquote{The paper studies a strange pattern of behavior: a large share of the population regularly allows its water to be turned off. To better understand what is going on, it would be helpful to have some background information for comparison with other countries. In other countries, how common is it for utilities to be turned off? How common is it for utility companies to be prohibited from charging interest on late payments?}
    \begin{refpoints}
    \item [Response here.]
    \end{refpoints}
  \end{refpoints}

\item \textbf{2.2.~Additional counterfactuals.}

  \begin{refpoints}
  \item \refquote{It would be interesting to add another counterfactual: the utility company charging interest on late payments, as is done in many other countries. If it is politically costly to impose a late fee, one could frame it as an early payment discount instead.}
    \begin{refpoints}
    \item [Response here.]
    \end{refpoints}
  \end{refpoints}

\item \textbf{2.3.~Intra-household bargaining.}

  \begin{refpoints}
  \item \refquote{What is the role of intra-household bargaining? In development economics, there is substantial literature on household bargaining and the inefficiencies that result from it. Who makes the decisions and pays the bills, and who bears most of the cost when the water is disconnected and the household has to obtain water from alternative sources? Is the person who pays the regular bill the same as the one who pays the one-time reconnection fee?}
    \begin{refpoints}
    \item [Response here.]
    \end{refpoints}
  \end{refpoints}

\item \textbf{2.4.~Summary statistics.}

  \begin{refpoints}
  \item \refquote{It would be informative if Table~1 presented, in addition to the mean, other statistics, at least the median.}
    \begin{refpoints}
    \item [Response here.]
    \end{refpoints}
  \end{refpoints}

\end{refpoints}

\end{document}
