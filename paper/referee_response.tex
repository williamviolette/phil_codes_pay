\documentclass[12pt]{article}

\usepackage{geometry,setspace,palatino,enumitem,hyperref}
\geometry{left=1.1in,right=1.1in,top=1.0in,bottom=1.0in}
\setlength{\parskip}{6pt}
\setlength{\parindent}{0pt}

\hypersetup{colorlinks=true,linkcolor=blue,urlcolor=blue,citecolor=blue}

\usepackage{xr-hyper}
\externaldocument[paper-]{water_bill_paper_wjv_rev}

% Custom list styles
\newlist{refpoints}{itemize}{2}
\setlist[refpoints,1]{label=\textbullet, leftmargin=1.5em, itemsep=6pt, parsep=2pt}
\setlist[refpoints,2]{label=$\circ$, leftmargin=2em, itemsep=3pt, parsep=1pt}

% Command for referee quotes
\newcommand{\refquote}[1]{\textit{``#1''}}

\begin{document}

\begin{center}
{\Large \textbf{Response to Editor and Referees}} \\[0.5em]
{\large ``Microcredit from Delaying Bill Payments''} \\[0.3em]
William Violette \\[0.3em]
\today
\end{center}

\vspace{1em}

Dear Ryan,

Thank you for the opportunity to revise and resubmit this paper and for the thoughtful suggestions.  Please find responses below, and please let me know if you have any follow-up questions.

Thanks again,

Will

\vspace{1em}
\hrule
\vspace{0.5em}
\section*{Editor Comments}

\begin{refpoints}

\item \refquote{Please do your best to complete the welfare comparison by modeling prepaid smoothing strategies.}
  \begin{refpoints}
  \item  The revised paper addresses both welfare channels raised by Referee~1.  First, a robustness exercise replaces the increasing block tariff with a single marginal price and finds similar welfare results (Section~\ref{paper-section:prepaidmeters}, p.~\pageref{paper-section:prepaidmeters}; Appendix~\ref{paper-appendix:linearprices}, p.~\pageref{paper-appendix:linearprices}).\footnote{Constant marginal prices are likely realistic given that increasing block tariffs are difficult to implement with prepaid meters.}  Second, the revised discussion of prepaid metering clarifies that this technology eliminates the need for payment enforcement because water flow stops automatically when credits run out (Section~\ref{paper-section:prepaidmeters}, p.~\pageref{paper-section:prepaidmeters}).  The model also allows households to accumulate precautionary savings through the standard asset.
  \end{refpoints}

\item \refquote{Consider broadening the policy space beyond ``current vs prepaid'' to incorporate features such as late fees (or the equivalent early-payment discount) instead of interest-free debt via the utility bill and/or a design in which a separate entity provides the loan while the utility focuses on water provision.}
  \begin{refpoints}
  \item The revised paper includes counterfactuals for both late fees and interest on unpaid bills (Section~\ref{paper-section:latepenalty}, p.~\pageref{paper-section:latepenalty}).  External liquidity provision is addressed by calculating the reduction in the borrowing rate necessary to leave households indifferent to prepaid meters (Section~\ref{paper-section:prepaidmeters}, p.~\pageref{paper-section:prepaidmeters}), which corresponds to a setting where a separate entity provides discounted loans while the utility focuses on water provision.
  \end{refpoints}

\item \refquote{Provide more institutional context and discussion of external validity (e.g., How common are shutoffs elsewhere? Are utilities typically allowed to charge interest on arrears? What is distinctive about Manila and what may generalize?).}
  \begin{refpoints}
  \item The revised paper includes a new Section~\ref{paper-section:background} (``Background,'' p.~\pageref{paper-section:background}) that addresses each of these points before the data section.  This section discusses (1) the prevalence of utility disconnections across developing and developed countries, (2) the variation in policies on interest and penalties for utility arrears, (3) what is distinctive about Manila (limited formal credit at very high rates, zero interest on arrears, detailed administrative data), and (4) where the core mechanism is likely to generalize.
  \end{refpoints}

\end{refpoints}

\vspace{1em}
\hrule
\vspace{0.5em}
\section*{Referee 1}

\begin{refpoints}

\item \refquote{Where price follows an inclining block tariff, as it does in this paper's setting in Manila, additional welfare improvements for households are accessible through prepaid metering, but not postpaid metering. Prepaid metered households are able to minimise their total expenditure on the utility over the period they are connected by purchasing their average monthly consumption\ldots This strategy minimises the total number of litres bought at the higher marginal price blocks of the inclining block schedule over the period that the household is connected.}
  \begin{refpoints}
  \item Thank you.  A robustness exercise replaces the increasing block tariff with a single marginal price and finds similar welfare effects (Section~\ref{paper-section:prepaidmeters}, p.~\pageref{paper-section:prepaidmeters}; Appendix~\ref{paper-appendix:linearprices}, p.~\pageref{paper-appendix:linearprices}).  In practice, tariffs tend to become flatter---if not single marginal prices---when prepaid meters are introduced, both because of this arbitrage concern and because of the difficulty of tracking monthly usage on prepaid meters.
  \end{refpoints}

\item \refquote{Just by purchasing ahead, households reduce the risk of receiving demand letters and disconnection notices, as well as reducing the risk of being disconnected if they plan to stay in the utility's service area. Demand letters and disconnection notices reduce utility and so do actual disconnections in the author's model\ldots Avoiding these improves welfare, all else equal.}
  \begin{refpoints}
  \item The revised discussion of prepaid metering (Section~\ref{paper-section:prepaidmeters}, p.~\pageref{paper-section:prepaidmeters}) clarifies that this technology eliminates the need for warnings and disconnections: water flow stops automatically as soon as the meter reaches zero.  The welfare comparison between prepaid and postpaid metering therefore does not involve differences in warning or disconnection hassle costs.
  \end{refpoints}

\end{refpoints}

\vspace{1em}
\hrule
\vspace{0.5em}
\section*{Referee 2}

\begin{refpoints}
\item \refquote{Given the research question, in my opinion, the paper would be a better fit for a development economics journal.}
  \begin{refpoints}
  \item While the setting is a developing country, the core questions are squarely within industrial organization: how should a regulated public utility design its billing and enforcement policies to balance cost recovery against consumer welfare?  The counterfactual exercises---prepaid metering, enforcement levels, late penalties, and interest rates on arrears---are policy design questions about the optimal regulation of a public utility.  The paper contributes to a growing IO literature on utility policy in developing countries (McRae 2015; Jack and Smith 2020) and uses a structural dynamic model of household demand that builds on standard IO methods.  The paper also engages with the development economics literature on credit constraints and consumption smoothing, which strengthens rather than undermines its fit for an IO audience.
  \end{refpoints}
\end{refpoints}

\vspace{0.5em}
\textbf{1.1.~Considering all relevant effects.}

\begin{refpoints}

\item \refquote{I'm not sure whether the paper considers all relevant effects when making the policy conclusions. In this setting, the water utility company has two roles: providing interest-free loans and providing water. In my opinion, it is not a typical IO problem.}
  \begin{refpoints}
  \item The paper's framework explicitly accounts for both roles: the utility provides water at regulated prices (the tariff structure and cost model in Section~\ref{paper-section:costmodel}, p.~\pageref{paper-section:costmodel}) and implicitly extends interest-free credit by tolerating unpaid bills (the borrowing constraint and delinquency warning process in Section~\ref{paper-section:model}, p.~\pageref{paper-section:model}).  The revenue-neutral approach holds utility profits at zero, isolating the net welfare effects on households while ensuring the utility remains viable.  When households use delayed payments as credit, this changes their demand for water, the utility's cost structure, and the optimal enforcement and pricing policy---a question that is relevant for IO because it fundamentally affects optimal utility regulation.  The revised paper broadens the policy space to include late penalties and interest rates on arrears (Columns~4 and~5 of Tables~\ref{paper-table:countercosts} and~\ref{paper-table:counter}, p.~\pageref{paper-table:countercosts}), which directly address the interaction between the utility's lending and service-provision roles.
  \end{refpoints}

\item \refquote{Access to clean water is a public health issue in developing countries. If the water company takes on dual roles, providing loans and water, it may face difficulties making the necessary investments to supply clean water to the entire population\ldots the paper currently does not account for the water company's investment decisions. In my opinion, it is likely that the current revenue is too low to support optimal investment.}
  \begin{refpoints}
  \item The paper addresses this concern in two ways.  First, the revenue-neutral framework ensures that total revenues cover all costs---including fixed infrastructure costs---under each counterfactual, with prices adjusting endogenously so the utility can always finance its operations (Section~\ref{paper-section:costmodel}, p.~\pageref{paper-section:costmodel}).  Second, the conclusion (Section~\ref{paper-section:conclusion}, p.~\pageref{paper-section:conclusion}) now discusses this limitation more explicitly, noting that McRae (2015) documents how high nonpayment rates can weaken investment incentives elsewhere.  In Manila, however, the regulatory structure has ensured universal access through a public-private partnership, and over 95\% of households rate service as ``good'' or ``very good.''  To the extent that delinquency undermines investment incentives in other settings, the welfare benefits estimated here would represent an upper bound.
  \end{refpoints}

\item \refquote{If the government wants to provide interest-free loans, why do it via the water utility company? Why not create another entity that provides loans or build an institutional framework that allows the market to do so? In my opinion, for the counterfactual policy analysis, this implies that we shouldn't consider only an epsilon improvement, but also consider what is optimal.}
  \begin{refpoints}
  \item The revised conclusion (Section~\ref{paper-section:conclusion}, p.~\pageref{paper-section:conclusion}) discusses why the utility may have a comparative advantage in providing credit: (1) it can use disconnection as implicit collateral, addressing constraints that restrict lending to poor households (Jack and Smith 2016); (2) its billing infrastructure lowers the transaction costs of screening and monitoring; and (3) its broad customer base allows it to diversify default risk.  A separate lending entity would need to replicate these capabilities at additional cost.  The revised paper also broadens the counterfactual policy space beyond epsilon improvements to include a late penalty (Column~4) and interest charges on arrears (Column~5) in Tables~\ref{paper-table:countercosts} and~\ref{paper-table:counter} (p.~\pageref{paper-table:countercosts}), spanning policies from eliminating implicit credit entirely to expanding it substantially.
  \end{refpoints}

\end{refpoints}

\vspace{0.5em}
\textbf{1.2.~Household discount rate.}

\begin{refpoints}

\item \refquote{The parameter for the household discount rate is assumed to be the same as found in the literature for the US. However, since the institutional details and behavior in Manila differ from those in the US, I believe it is important to use parameters that are estimated for developing countries. It seems like a very strong assumption to assume that the annual discount rate in Manila is 6\%. This is reasonable in the US, where the annual interest rate could be considered 6\%. But in Manila, as the paper states, the annual interest rate for most households is about 200\% (9.5\% monthly rate). Perhaps it would be useful to look at discount rate estimates in the Philippines (Ashraf, Karlan, and Yin, 2006 QJE).}
  \begin{refpoints}
  \item Thank you.  Table~\ref{paper-table:robustestimates} (p.~\pageref{paper-table:robustestimates}) now includes robustness checks with a higher discount rate, citing Ashraf, Karlan, and Yin (2006, QJE) and using the 2.4\% monthly discount rate implied by the Matousek et al.\ (2022) meta-analysis of the experimental literature.  Higher discount rates magnify welfare effects, as discussed in Section~\ref{paper-section:costmodel} (p.~\pageref{paper-section:costmodel}).
  \end{refpoints}

\end{refpoints}

\vspace{0.5em}
\textbf{2.1.~More details on the institution and comparison with other countries.}

\begin{refpoints}

\item \refquote{The paper studies a strange pattern of behavior: a large share of the population regularly allows its water to be turned off. To better understand what is going on, it would be helpful to have some background information for comparison with other countries. In other countries, how common is it for utilities to be turned off? How common is it for utility companies to be prohibited from charging interest on late payments?}
  \begin{refpoints}
  \item The revised paper includes a new Section~\ref{paper-section:background} (``Background,'' p.~\pageref{paper-section:background}) that addresses these questions with cross-country evidence on disconnection prevalence, interest and penalty policies for utility arrears, what is distinctive about Manila (limited formal credit at high borrowing rates, zero interest on arrears, detailed administrative data), and where the core mechanism is likely to generalize.
  \end{refpoints}

\end{refpoints}

\vspace{0.5em}
\textbf{2.2.~Additional counterfactuals.}

\begin{refpoints}

\item \refquote{It would be interesting to add another counterfactual: the utility company charging interest on late payments, as is done in many other countries. If it is politically costly to impose a late fee, one could frame it as an early payment discount instead.}
  \begin{refpoints}
  \item Thank you---this suggestion has substantially improved the paper.  The revised paper includes two new counterfactuals (Section~\ref{paper-section:latepenalty}, p.~\pageref{paper-section:latepenalty}).  Column~(4) of Tables~\ref{paper-table:countercosts} and~\ref{paper-table:counter} (p.~\pageref{paper-table:countercosts}) introduces a late penalty equal to 10\% of the average unpaid balance (123.5~PhP per month when carrying a balance), consistent with surcharge policies used by other Philippine water providers.  Column~(5) introduces a 4.9\% monthly interest rate on unpaid bills, so that outstanding balances grow each month.  Both policies are paired with revenue-neutral price adjustments.  These counterfactuals span the policy space from penalizing delinquency (late fees and interest) to encouraging it (reduced enforcement), with prepaid metering as an extreme case of eliminating credit entirely.  The suggestion of framing penalties as early payment discounts is noted as a politically feasible alternative in Section~\ref{paper-section:latepenalty} (p.~\pageref{paper-section:latepenalty}).
  \end{refpoints}

\end{refpoints}

\vspace{0.5em}
\textbf{2.3.~Intra-household bargaining.}

\begin{refpoints}

\item \refquote{What is the role of intra-household bargaining? In development economics, there is substantial literature on household bargaining and the inefficiencies that result from it. Who makes the decisions and pays the bills, and who bears most of the cost when the water is disconnected and the household has to obtain water from alternative sources? Is the person who pays the regular bill the same as the one who pays the one-time reconnection fee?}
  \begin{refpoints}
  \item The paper models the household as a unitary decision-maker, which is standard in the consumption smoothing literature following Deaton (1991).  The billing data record payments and consumption at the connection level and do not identify which household member makes payment decisions or bears the costs of disconnection.  Jack and Smith (2020) note intra-household bargaining as a potential mechanism for why prepaid meters reduce consumption---if the bill-payer and the primary water user differ, prepaid meters may shift bargaining power.  To the extent that intra-household frictions cause additional welfare losses from disconnection, the estimates of the welfare costs of prepaid metering and the benefits of reduced enforcement could be understated.  Exploring intra-household dynamics in utility payment decisions is a valuable direction for future research but requires within-household data that are not available in this setting.
  \end{refpoints}

\end{refpoints}

\vspace{0.5em}
\textbf{2.4.~Summary statistics.}

\begin{refpoints}

\item \refquote{It would be informative if Table~1 presented, in addition to the mean, other statistics, at least the median.}
  \begin{refpoints}
  \item Table~\ref{paper-table:descriptives_all} (p.~\pageref{paper-table:descriptives_all}) now includes the median alongside the mean.
  \end{refpoints}

\end{refpoints}

\end{document}
