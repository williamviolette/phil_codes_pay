\documentclass[12pt]{article}

\usepackage{geometry,setspace,palatino,enumitem,hyperref}
\geometry{left=1.1in,right=1.1in,top=1.0in,bottom=1.0in}
\setlength{\parskip}{6pt}
\setlength{\parindent}{0pt}

\hypersetup{colorlinks=true,linkcolor=blue,urlcolor=blue,citecolor=blue}

\usepackage{xr-hyper}
\externaldocument[paper-]{water_bill_paper_wjv_rev_highlighted}

% Paragraph-number citation command  (prints e.g. "[¶3]")
\newcommand{\parref}[1]{[\P\ref{paper-#1}]}

% Custom list styles
\newlist{refpoints}{itemize}{2}
\setlist[refpoints,1]{label=\textbullet, leftmargin=1.5em, itemsep=6pt, parsep=2pt}
\setlist[refpoints,2]{label=$\circ$, leftmargin=2em, itemsep=3pt, parsep=1pt}

% Command for referee quotes
\newcommand{\refquote}[1]{\textit{``#1''}}

\begin{document}

\begin{center}
{\Large \textbf{Response to Editor and Referees}} \\[0.5em]
{\large ``Microcredit from Delaying Bill Payments''} \\[0.3em]
William Violette \\[0.3em]
\today
\end{center}

\vspace{1em}

Dear Ryan,

Thank you for the opportunity to revise and resubmit this paper and for the thoughtful suggestions.  Please find responses below, and please let me know if you have any follow-up questions.

Thanks again,

Will

\vspace{1em}
\hrule
\vspace{0.5em}
\section*{Editor Comments}

\begin{refpoints}

\item \refquote{Please do your best to complete the welfare comparison by modeling prepaid smoothing strategies.}
  \begin{refpoints}
  \item  To address the concern that prepayment may help avoid higher tariff blocks, a new robustness exercise replaces the increasing block tariff with a single marginal price and finds similar welfare results (p.~\pageref{paper-par:linearrobust}, \parref{par:linearrobust}; p.~\pageref{paper-par:linearpricesappendix}, \parref{par:linearpricesappendix}).\footnote{Constant marginal prices are likely realistic given that increasing block tariffs are difficult to implement with prepaid meters.}  To address the concern that prepayment may avoid extra hassle costs of disconnection enforcement, counterfactual now assumes that this technology eliminates the need for payment enforcement consistent with water flow stopping automatically when credits run out (p.~\pageref{paper-par:prepaidexplain}, \parref{par:prepaidexplain}).  Finally, additional discussion observes how the standard asset provides a way for the model to capture prepaid smoothing to some extent (p.~\pageref{paper-par:prepaidsmoothing}, \parref{par:prepaidsmoothing}).
  \end{refpoints}

\item \refquote{Consider broadening the policy space beyond ``current vs prepaid'' to incorporate features such as late fees (or the equivalent early-payment discount) instead of interest-free debt via the utility bill and/or a design in which a separate entity provides the loan while the utility focuses on water provision.}
  \begin{refpoints}
  \item The revised paper includes counterfactuals for both late fees and interest on unpaid bills (p.~\pageref{paper-par:latepenaltysection}, \parref{par:latepenaltysection}).  External liquidity provision is addressed by calculating the reduction in the borrowing rate necessary to leave households indifferent to prepaid meters (p.~\pageref{paper-par:microcreditoffset}, \parref{par:microcreditoffset}), which corresponds to a setting where a separate entity provides discounted loans while the utility focuses entirely on water provision.
  \end{refpoints}

\item \refquote{Provide more institutional context and discussion of external validity (e.g., How common are shutoffs elsewhere? Are utilities typically allowed to charge interest on arrears? What is distinctive about Manila and what may generalize?).}
  \begin{refpoints}
  \item A new paragraph in the Introduction (p.~\pageref{paper-par:manilacontext}, \parref{par:manilacontext}) describes what is distinctive about Manila---having a relatively high-functioning water utility for a lower-middle income country---and discusses how lessons may not generalize to developing cities with lower capacity infrastructure, but may likely be useful as these cities undertake their own large infrastructure investments.   A new paragraph in Section~\ref{paper-section:descriptives} (p.~\pageref{paper-par:crosscountry}, \parref{par:crosscountry}) demonstrates that disconnections/arrears enforcement in Manila are generally typical of other settings, using cross-country evidence from South Africa, the United States, and other countries.
  \end{refpoints}

\end{refpoints}

\vspace{1em}
\hrule
\vspace{0.5em}
\section*{Referee 1}

\begin{refpoints}

\item \refquote{Where price follows an inclining block tariff, as it does in this paper's setting in Manila, additional welfare improvements for households are accessible through prepaid metering, but not postpaid metering. Prepaid metered households are able to minimise their total expenditure on the utility over the period they are connected by purchasing their average monthly consumption\ldots This strategy minimises the total number of litres bought at the higher marginal price blocks of the inclining block schedule over the period that the household is connected.}
  \begin{refpoints}
  \item Helpful observation, thanks!  A robustness exercise replaces the increasing block tariff with a single marginal price and finds similar welfare effects (Section~\ref{paper-section:prepaidmeters}, \parref{par:linearrobust}; Appendix~\ref{paper-appendix:linearprices}, \parref{par:linearpricesappendix}).  It also looks like, at least in South Africa, people rarely accumulate units across months even with increasing block tariffs (Jack and Smith, 2020).  I also added discussion about how the standard asset might provide a way for the model to capture prepaid smoothing to some extent (p.~\pageref{paper-par:prepaidsmoothing}, \parref{par:prepaidsmoothing}).
  \end{refpoints}

\item \refquote{Just by purchasing ahead, households reduce the risk of receiving demand letters and disconnection notices, as well as reducing the risk of being disconnected if they plan to stay in the utility's service area. Demand letters and disconnection notices reduce utility and so do actual disconnections in the author's model\ldots Avoiding these improves welfare, all else equal.}
  \begin{refpoints}
  \item I think this is mainly a technological question: I could see this being a concern if prepaid meters required manual disconnection, but the technologies that I've seen eliminate the need for warnings and disconnections because water flow stops automatically as soon as the meter reaches zero.  I added this assumption in the revised discussion of prepaid metering (Section~\ref{paper-section:prepaidmeters}, \parref{par:prepaidexplain}).
  \end{refpoints}

\end{refpoints}

\vspace{1em}
\hrule
\vspace{0.5em}
\section*{Referee 2}

\begin{refpoints}
\item \refquote{Given the research question, in my opinion, the paper would be a better fit for a development economics journal.}
  \begin{refpoints}
  \item Thank you for raising this.  The revised Introduction now includes a new sentence (p.~\pageref{paper-par:iocite}, \parref{par:iocite}) connecting the paper to an IO literature on utility regulation in developing countries---in particular, work on the consequences of treating utility services as implicit entitlements (Burgess et al.\ 2020) and on how disconnection and penalty policies distribute costs across households (Cicala et al.\ 2021).
  \end{refpoints}
\end{refpoints}

\vspace{0.5em}
\textbf{1.1.~Considering all relevant effects.}

\begin{refpoints}

%\item \refquote{I'm not sure whether the paper considers all relevant effects when making the policy conclusions. In this setting, the water utility company has two roles: providing interest-free loans and providing water. In my opinion, it is not a typical IO problem.}
  % \begin{refpoints}
  % \item The revised paper broadens the policy space to include late penalties and interest rates on arrears in Columns~4 and~5 of Tables~\ref{paper-table:countercosts} and~\ref{paper-table:counter} (p.~\pageref{paper-par:costtable}, \parref{par:costtable}~and p.~\pageref{paper-par:outcometable}, \parref{par:outcometable}), with discussion in a new subsection (p.~\pageref{paper-par:latepenaltysection}, \parref{par:latepenaltysection}).   External liquidity provision is addressed by calculating the reduction in the borrowing rate necessary to leave households indifferent to prepaid meters (p.~\pageref{paper-par:microcreditoffset}, \parref{par:microcreditoffset}), which imagines a setting where a microcredit program is paired with prepaid meters so the utility can focus entirely on water provision.
  % \end{refpoints}

\item \refquote{Access to clean water is a public health issue in developing countries. If the water company takes on dual roles, providing loans and water, it may face difficulties making the necessary investments to supply clean water to the entire population\ldots the paper currently does not account for the water company's investment decisions. In my opinion, it is likely that the current revenue is too low to support optimal investment.}
  \begin{refpoints}
  \item A new paragraph in the Introduction (p.~\pageref{paper-par:manilacontext}, \parref{par:manilacontext}) discusses the Manila context, which includes high-capacity water provision and relatively high levels of service quality.  This paragraph further discusses the extent to which this context would generalize to other settings.  It also cites Wu and Malaluan [2008], which discusses the public-private parternship regulatory structure in Manila where prices are set to recover costs including investments.
  \end{refpoints}

\item \refquote{If the government wants to provide interest-free loans, why do it via the water utility company? Why not create another entity that provides loans or build an institutional framework that allows the market to do so? In my opinion, for the counterfactual policy analysis, this implies that we shouldn't consider only an epsilon improvement, but also consider what is optimal.}
  \begin{refpoints}
  \item The revised paper calculates how much borrowing costs would need to fall to leave households indifferent to prepaid meters (p.~\pageref{paper-par:microcreditoffset}, \parref{par:microcreditoffset}), providing a benchmark for the cost of external credit provision.  The revised counterfactual policy space now spans from eliminating implicit credit entirely (prepaid metering) to penalizing delinquency (late fees and interest, p.~\pageref{paper-par:latepenaltysection}, \parref{par:latepenaltysection}) to expanding it substantially (reduced enforcement).
  \end{refpoints}

\end{refpoints}

\vspace{0.5em}
\textbf{1.2.~Household discount rate.}

\begin{refpoints}

\item \refquote{The parameter for the household discount rate is assumed to be the same as found in the literature for the US. However, since the institutional details and behavior in Manila differ from those in the US, I believe it is important to use parameters that are estimated for developing countries. It seems like a very strong assumption to assume that the annual discount rate in Manila is 6\%. This is reasonable in the US, where the annual interest rate could be considered 6\%. But in Manila, as the paper states, the annual interest rate for most households is about 200\% (9.5\% monthly rate). Perhaps it would be useful to look at discount rate estimates in the Philippines (Ashraf, Karlan, and Yin, 2006 QJE).}
  \begin{refpoints}
  \item Thanks for the suggestion.  Table~\ref{paper-table:robustestimates} (p.~\pageref{paper-par:discountrate}, \parref{par:discountrate}) now includes robustness checks with a higher discount rate of 2.4\% per month, citing Ashraf, Karlan, and Yin (2006, QJE) and using the implied monthly discount rate from the Matousek et al.\ (2022) meta-analysis of the experimental literature.  Higher discount rates magnify welfare effects.  Table~\ref{paper-table:robustestimates} also includes compensating variations for the new late penalty and interest rate counterfactuals under each preference specification (p.~\pageref{paper-par:robustnewrows}, \parref{par:robustnewrows}).
  \end{refpoints}

\end{refpoints}

\vspace{0.5em}
\textbf{2.1.~More details on the institution and comparison with other countries.}

\begin{refpoints}

\item \refquote{The paper studies a strange pattern of behavior: a large share of the population regularly allows its water to be turned off. To better understand what is going on, it would be helpful to have some background information for comparison with other countries. In other countries, how common is it for utilities to be turned off? How common is it for utility companies to be prohibited from charging interest on late payments?}
  \begin{refpoints}
  \item The revised paper addresses these questions in two places.  A new paragraph in the Introduction (p.~\pageref{paper-par:manilacontext}, \parref{par:manilacontext}) describes what is distinctive about Manila and discusses where the context is informative for other developing cities.  A new paragraph in Section~\ref{paper-section:descriptives} (p.~\pageref{paper-par:crosscountry}, \parref{par:crosscountry}) provides cross-country evidence on disconnection prevalence and penalty policies for utility arrears, covering South Africa and the United States.
  \end{refpoints}

\end{refpoints}

\vspace{0.5em}
\textbf{2.2.~Additional counterfactuals.}

\begin{refpoints}

\item \refquote{It would be interesting to add another counterfactual: the utility company charging interest on late payments, as is done in many other countries. If it is politically costly to impose a late fee, one could frame it as an early payment discount instead.}
  \begin{refpoints}
  \item Thank you---this suggestion has substantially improved the paper.  The revised paper includes two new counterfactuals in a new subsection (p.~\pageref{paper-par:latepenaltysection}, \parref{par:latepenaltysection}).  Column~(4) of Tables~\ref{paper-table:countercosts} and~\ref{paper-table:counter} (p.~\pageref{paper-par:costtable}, \parref{par:costtable}~and p.~\pageref{paper-par:outcometable}, \parref{par:outcometable}) introduces a late penalty equal to 10\% of the average unpaid balance, consistent with surcharge policies used by other Philippine water providers.  Column~(5) introduces a 4.9\% monthly interest rate on unpaid bills.  These results are also summarized in the Introduction (p.~\pageref{paper-par:latepenaltyintro}, \parref{par:latepenaltyintro}).
  \end{refpoints}

\end{refpoints}

\vspace{0.5em}
\textbf{2.3.~Intra-household bargaining.}

\begin{refpoints}

\item \refquote{What is the role of intra-household bargaining? In development economics, there is substantial literature on household bargaining and the inefficiencies that result from it. Who makes the decisions and pays the bills, and who bears most of the cost when the water is disconnected and the household has to obtain water from alternative sources? Is the person who pays the regular bill the same as the one who pays the one-time reconnection fee?}
  \begin{refpoints}
  \item Thank you for highlighting this, and the revised paper now states this limitation explicitly (p.~\pageref{paper-par:intrahh}, \parref{par:intrahh}): the analysis assumes that prepaid meters do not affect intra-household bargaining over water consumption, for example if the bill-payer and primary water users differ.  The billing data record payments and consumption at the connection level and do not identify which household member makes payment decisions or bears the costs of disconnection.
  \end{refpoints}

\end{refpoints}

\vspace{0.5em}
\textbf{2.4.~Summary statistics.}

\begin{refpoints}

\item \refquote{It would be informative if Table~1 presented, in addition to the mean, other statistics, at least the median.}
  \begin{refpoints}
  \item Table~\ref{paper-table:descriptives_all} (p.~\pageref{paper-par:medianstats}, \parref{par:medianstats}) now includes the median alongside the mean.
  \end{refpoints}

\end{refpoints}

\end{document}
