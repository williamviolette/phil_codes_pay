\documentclass[12pt]{article}

\usepackage{geometry,setspace,palatino,enumitem,hyperref}
\geometry{left=1.1in,right=1.1in,top=1.0in,bottom=1.0in}
\setlength{\parskip}{6pt}
\setlength{\parindent}{0pt}

\hypersetup{colorlinks=true,linkcolor=blue,urlcolor=blue,citecolor=blue}

\usepackage{xr-hyper}
\externaldocument[paper-]{water_bill_paper_wjv_rev} 

% Custom list styles
\newlist{refpoints}{itemize}{2}
\setlist[refpoints,1]{label=\textbullet, leftmargin=1.5em, itemsep=6pt, parsep=2pt}
\setlist[refpoints,2]{label=$\circ$, leftmargin=2em, itemsep=3pt, parsep=1pt}

% Command for referee quotes
\newcommand{\refquote}[1]{\textit{``#1''}}

\begin{document}

\begin{center}
{\Large \textbf{Response to Editor and Referees}} \\[0.5em]
{\large ``Microcredit from Delaying Bill Payments''} \\[0.3em]
William Violette \\[0.3em]
\today
\end{center}

\vspace{1em}

Dear Ryan,

Thank you for the opportunity to revise and resubmit this paper and for the thoughtful suggestions.  Please find responses below, and please let me know if you have any follow-up questions.

Thanks again,

Will

\vspace{1em}
\hrule
\vspace{0.5em}
\section*{Editor Comments}

\begin{refpoints}

\item \refquote{Please do your best to complete the welfare comparison by modeling prepaid smoothing strategies.}
  \begin{refpoints}
  \item [Response here.]
  \end{refpoints}

\item \refquote{Consider broadening the policy space beyond ``current vs prepaid'' to incorporate features such as late fees (or the equivalent early-payment discount) instead of interest-free debt via the utility bill and/or a design in which a separate entity provides the loan while the utility focuses on water provision.}
  \begin{refpoints}
  \item [Response here.]
  \end{refpoints}

\item \refquote{Provide more institutional context and discussion of external validity (e.g., How common are shutoffs elsewhere? Are utilities typically allowed to charge interest on arrears? What is distinctive about Manila and what may generalize?).}
  \begin{refpoints}
  \item Thank you.  The revised paper now includes a new Section~\ref{paper-section:background} (``Background'') that addresses each of these points before the data section.  This section discusses (1) the prevalence of utility disconnections across developing and developed countries, (2) the variation in policies on interest and penalties for utility arrears, (3) what is distinctive about Manila (limited formal credit at very high rates, zero interest on arrears, detailed administrative data), and (4) where the core mechanism is likely to generalize.
  \end{refpoints}

\end{refpoints}

\vspace{1em}
\hrule
\vspace{0.5em}
\section*{Referee 1}

\textbf{Overview and recommendation:} Referee~1 recommends acceptance without revisions, describing the paper as ``well executed'' and ``a substantial contribution.'' We are grateful for this positive assessment. Below we address the substantive concern raised.

\vspace{0.5em}
\textbf{Main comment:} \refquote{A greater concern of mine relates to what I feel is an incompleteness in the Welfare calculations with respect to prepaid and postpaid metering.}

\begin{refpoints}

\item \refquote{With prepaid meters, however, households can also consumption smooth in a more prospective fashion, through purchasing strategically ahead of time.}
  \begin{refpoints}
  \item [Response here.]
  \end{refpoints}

\item \refquote{Where price follows an inclining block tariff, as it does in this paper's setting in Manila, additional welfare improvements for households are accessible through prepaid metering, but not postpaid metering. Prepaid metered households are able to minimise their total expenditure on the utility over the period they are connected by purchasing their average monthly consumption\ldots This strategy minimises the total number of litres bought at the higher marginal price blocks of the inclining block schedule over the period that the household is connected.}
  \begin{refpoints}
  \item [Response here.]
  \end{refpoints}

\item \refquote{Just by purchasing ahead, households reduce the risk of receiving demand letters and disconnection notices, as well as reducing the risk of being disconnected if they plan to stay in the utility's service area. Demand letters and disconnection notices reduce utility and so do actual disconnections in the author's model\ldots Avoiding these improves welfare, all else equal.}
  \begin{refpoints}
  \item [Response here.]
  \end{refpoints}

\end{refpoints}

\vspace{1em}
\hrule
\vspace{0.5em}
\section*{Referee 2}

\begin{refpoints}
\item \refquote{Given the research question, in my opinion, the paper would be a better fit for a development economics journal.}
  \begin{refpoints}
  \item Thank you for this comment.  While the setting is a developing country, the core questions are squarely within industrial organization: how should a regulated public utility design its billing and enforcement policies to balance cost recovery against consumer welfare?  The counterfactual exercises---prepaid metering, enforcement levels, late penalties, and interest rates on arrears---are all policy design questions about the optimal regulation of a public utility.  The paper contributes to a growing IO literature on utility policy in developing countries (McRae 2015; Jack and Smith 2020) and uses a structural dynamic model of household demand that builds on standard IO methods.  At the same time, the paper engages with the development economics literature on credit constraints and consumption smoothing, which we believe strengthens rather than undermines the paper's fit for an IO audience.
  \end{refpoints}
\end{refpoints}

\vspace{0.5em}
\textbf{1.1.~Considering all relevant effects.}

\begin{refpoints}

\item \refquote{I'm not sure whether the paper considers all relevant effects when making the policy conclusions. In this setting, the water utility company has two roles: providing interest-free loans and providing water. In my opinion, it is not a typical IO problem.}
  \begin{refpoints}
  \item We agree that the dual role is a key feature of this setting and appreciate the referee highlighting it.  The paper's framework explicitly accounts for both roles: the utility provides water at regulated prices (captured by the tariff structure and cost model) and implicitly extends interest-free credit by tolerating unpaid bills (captured by the borrowing constraint and delinquency warning process).  The revenue-neutral approach holds utility profits at zero, which isolates the net welfare effects on households while ensuring the utility remains viable.  The paper contributes precisely by showing how these two roles interact---a question that is relevant for IO because it fundamentally affects how we think about optimal utility regulation.  When households use delayed payments as credit, this changes their demand for water, the utility's cost structure, and the optimal enforcement and pricing policy.  The revised paper now broadens the policy space to include late penalties and interest rates on arrears (Columns 4 and 5 of Tables~\ref{paper-table:countercosts} and~\ref{paper-table:counter}), which further addresses the interaction between the utility's lending and service-provision roles.
  \end{refpoints}

\item \refquote{Access to clean water is a public health issue in developing countries. If the water company takes on dual roles, providing loans and water, it may face difficulties making the necessary investments to supply clean water to the entire population\ldots the paper currently does not account for the water company's investment decisions. In my opinion, it is likely that the current revenue is too low to support optimal investment.}
  \begin{refpoints}
  \item This is an important concern.  The paper addresses it in two ways.  First, the revenue-neutral framework ensures that the utility's total revenues cover all costs---including fixed infrastructure costs---under each counterfactual.  Prices adjust endogenously so the utility can always finance its operations.  Second, the conclusion now discusses this limitation more explicitly, noting that McRae (2015) documents how high nonpayment rates can weaken investment incentives in other settings.  However, the regulatory structure in Manila has successfully ensured universal access and maintained service quality through a public-private partnership---over 95\% of households rate the utility's service as ``good'' or ``very good.''  This institutional context suggests that investment incentives are functioning well in this particular setting, making the partial equilibrium approach a reasonable starting point.  To the extent that delinquency undermines investment incentives in other settings, the welfare benefits of tolerating unpaid bills estimated in this paper would represent an upper bound.
  \end{refpoints}

\item \refquote{If the government wants to provide interest-free loans, why do it via the water utility company? Why not create another entity that provides loans or build an institutional framework that allows the market to do so? In my opinion, for the counterfactual policy analysis, this implies that we shouldn't consider only an epsilon improvement, but also consider what is optimal.}
  \begin{refpoints}
  \item Thank you---this is a useful framing.  The paper now discusses why the utility may have a comparative advantage in providing credit relative to a separate lending entity.  As discussed in the revised conclusion (Section~\ref{paper-section:conclusion}), the utility overcomes three key barriers that limit formal lending in low-income settings: (1) it can use the threat of disconnection as implicit collateral, addressing the collateral constraints that restrict lending to poor households (Jack and Smith 2016); (2) its existing metering and billing infrastructure lowers the transaction costs of screening and monitoring debtors; and (3) its broad customer base allows it to spread default risk more effectively than small-scale moneylenders or peer-to-peer groups.  A separate lending entity would need to replicate these capabilities at additional cost.  Regarding the policy space, the revised paper now broadens the set of counterfactuals beyond ``epsilon improvements'' to include a late penalty (Column 4) and interest charges on arrears (Column 5), which represent fundamentally different approaches to managing the utility's dual role.  These additions, together with the prepaid metering and reduced enforcement counterfactuals, span a wide range of policies from eliminating implicit credit entirely to expanding it substantially.
  \end{refpoints}

\end{refpoints}

\vspace{0.5em}
\textbf{1.2.~Household discount rate.}

\begin{refpoints}

\item \refquote{The parameter for the household discount rate is assumed to be the same as found in the literature for the US. However, since the institutional details and behavior in Manila differ from those in the US, I believe it is important to use parameters that are estimated for developing countries. It seems like a very strong assumption to assume that the annual discount rate in Manila is 6\%. This is reasonable in the US, where the annual interest rate could be considered 6\%. But in Manila, as the paper states, the annual interest rate for most households is about 200\% (9.5\% monthly rate). Perhaps it would be useful to look at discount rate estimates in the Philippines (Ashraf, Karlan, and Yin, 2006 QJE).}
  \begin{refpoints}
  \item Great suggestion; see updated Table~\ref{paper-table:robustestimates} which cites Ashraf, Karlan, and Yin, 2006 QJE to justify a higher discount rate and uses a monthly discount rate of 2.4\% implied by Matousek et al. [2022] in their meta-analysis of the experimental literature.  Overall, I find that high discount rates magnify welfare effects as discussed in Section~\ref{paper-section:costmodel}.
  \end{refpoints}

\end{refpoints}

\vspace{0.5em}
\textbf{2.1.~More details on the institution and comparison with other countries.}

\begin{refpoints}

\item \refquote{The paper studies a strange pattern of behavior: a large share of the population regularly allows its water to be turned off. To better understand what is going on, it would be helpful to have some background information for comparison with other countries. In other countries, how common is it for utilities to be turned off? How common is it for utility companies to be prohibited from charging interest on late payments?}
  \begin{refpoints}
  \item Thank you for this suggestion.  The revised paper now includes additional institutional context.  Disconnections for nonpayment are common in both developing and developed countries.  In the US, the NAACP and other groups have documented widespread utility shutoffs, and many states restrict disconnections during extreme temperatures (US Department of Health and Human Services, LIHEAP).  Disconnections were also widely suspended globally during COVID-19, underscoring their prevalence.  In South Africa, Jack and Smith (2020) document high nonpayment rates that motivated the adoption of prepaid electricity meters.  Regarding interest on arrears: many utilities worldwide are prohibited or restricted from charging interest on late payments, particularly for water, which is often treated as a basic human right.  In the Philippines, the prohibition on interest reflects this stance.  However, some utilities in the Philippines and elsewhere do impose late penalty surcharges---typically 10\% of the outstanding amount---which motivates the late penalty counterfactual in our revised analysis.  What is distinctive about Manila is the combination of (1) very limited formal credit access with high borrowing rates (around 9.5\% monthly), (2) zero interest on arrears, and (3) detailed administrative billing data.  The core mechanism---households timing bill payments to smooth consumption---is likely to generalize wherever credit-constrained households face utilities that tolerate delinquency, though the magnitudes will depend on local credit conditions and enforcement practices.
  \end{refpoints}

\end{refpoints}

\vspace{0.5em}
\textbf{2.2.~Additional counterfactuals.}

\begin{refpoints}

\item \refquote{It would be interesting to add another counterfactual: the utility company charging interest on late payments, as is done in many other countries. If it is politically costly to impose a late fee, one could frame it as an early payment discount instead.}
  \begin{refpoints}
  \item Thank you---this is a very helpful suggestion that has substantially improved the paper.  The revised paper now includes two new counterfactuals that directly address this point.  First, Column~(4) of Tables~\ref{paper-table:countercosts} and~\ref{paper-table:counter} introduces a late penalty equal to 10\% of the average unpaid balance (123.5 PhP per month when carrying a balance), which is consistent with late surcharge policies used by other Philippine water providers.  Second, Column~(5) introduces a 4.9\% monthly interest rate on unpaid bills, so that outstanding balances grow each month.  Both policies are paired with revenue-neutral price adjustments.  These counterfactuals span the policy space from penalizing delinquency (late fees and interest) to encouraging it (reduced enforcement), with prepaid metering as an extreme case of eliminating credit entirely.  We also appreciate the suggestion about framing penalties as early payment discounts, which we note in the revised discussion as a politically feasible implementation.
  \end{refpoints}

\end{refpoints}

\vspace{0.5em}
\textbf{2.3.~Intra-household bargaining.}

\begin{refpoints}

\item \refquote{What is the role of intra-household bargaining? In development economics, there is substantial literature on household bargaining and the inefficiencies that result from it. Who makes the decisions and pays the bills, and who bears most of the cost when the water is disconnected and the household has to obtain water from alternative sources? Is the person who pays the regular bill the same as the one who pays the one-time reconnection fee?}
  \begin{refpoints}
  \item This is a thoughtful point.  The paper models the household as a unitary decision-maker, which is standard in the consumption smoothing literature following Deaton (1991).  The billing data record payments and consumption at the connection level and do not identify which household member makes payment decisions or bears the costs of disconnection.  Jack and Smith (2020) note intra-household bargaining as a potential mechanism for why prepaid meters reduce consumption---if the bill-payer and the primary water user differ, prepaid meters may shift bargaining power.  We acknowledge this as a limitation of our unitary household framework.  To the extent that intra-household frictions cause additional welfare losses from disconnection (e.g., if the costs fall disproportionately on members who do not control payment decisions), our estimates of the welfare costs of prepaid metering and the benefits of reduced enforcement could be understated.  Exploring intra-household dynamics in utility payment decisions would be a valuable direction for future research but requires within-household data that are not available in this setting.
  \end{refpoints}

\end{refpoints}

\vspace{0.5em}
\textbf{2.4.~Summary statistics.}

\begin{refpoints}

\item \refquote{It would be informative if Table~1 presented, in addition to the mean, other statistics, at least the median.}
  \begin{refpoints}
  \item Please see updated Table~\ref{paper-table:descriptives_all} including the median as well.
  \end{refpoints}

\end{refpoints}

\end{document}
