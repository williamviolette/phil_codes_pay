%%%%%%%%%%%%%%%%%%%%%%%%%%%%%%%%%%%%%%%%%
% Beamer Presentation
% LaTeX Template
% Version 1.0 (10/11/12)
%
% This template has been downloaded from:
% http://www.LaTeXTemplates.com
%
% License:
% CC BY-NC-SA 3.0 (http://creativecommons.org/licenses/by-nc-sa/3.0/)
%
%%%%%%%%%%%%%%%%%%%%%%%%%%%%%%%%%%%%%%%%%

%----------------------------------------------------------------------------------------
%	PACKAGES AND THEMES
%----------------------------------------------------------------------------------------

\documentclass[aspectratio=149]{beamer}
\usefonttheme[onlymath]{serif}


\mode<presentation> {

% The Beamer class comes with a number of default slide themes
% which change the colors and layouts of slides. Below this is a list
% of all the themes, uncomment each in turn to see what they look like.

\usetheme{default}
%\usetheme{AnnArbor}
%\usetheme{Antibes}
%\usetheme{Bergen}
%\usetheme{Berkeley}
%\usetheme{Berlin}
%\usetheme{Boadilla}
%\usetheme{CambridgeUS}
%\usetheme{Copenhagen}
%\usetheme{Darmstadt}
%\usetheme{Dresden}
%\usetheme{Frankfurt}
%\usetheme{Goettingen}
%\usetheme{Hannover}
%\usetheme{Ilmenau}
%\usetheme{JuanLesPins}
%\usetheme{Luebeck}
%\usetheme{Malmoe}
%\usetheme{Marburg}
%\usetheme{Montpellier}
%\usetheme{PaloAlto}
%\usetheme{Pittsburgh}
%\usetheme{Rochester}
%\usetheme{Singapore}
%\usetheme{Szeged}
%\usetheme{Warsaw}

% As well as themes, the Beamer class has a number of color themes
% for any slide theme. Uncomment each of these in turn to see how it
% changes the colors of your current slide theme.

%\usecolortheme{albatross}
\usecolortheme{beaver}
%\usecolortheme{beetle}
%\usecolortheme{crane}
%\usecolortheme{dolphin}
%\usecolortheme{dove}
%\usecolortheme{fly}
%\usecolortheme{lily}
%\usecolortheme{orchid}
%\usecolortheme{rose}
%\usecolortheme{seagull}
%\usecolortheme{seahorse}
%\usecolortheme{whale}
%\usecolortheme{wolverine}

%\setbeamertemplate{footline} % To remove the footer line in all slides uncomment this line
%\setbeamertemplate{footline}[page number] % To replace the footer line in all slides with a simple slide count uncomment this line

%\setbeamertemplate{navigation symbols}{} % To remove the navigation symbols from the bottom of all slides uncomment this line
}

\usepackage{graphicx} % Allows including images
\usepackage{booktabs} % Allows the use of \toprule, \midrule and \bottomrule in tables
\usepackage{verbatim}

\usepackage{mathtools} 
\usepackage{amssymb}
\usepackage{mathrsfs}
\usepackage{amsmath}

\usepackage{ragged2e}
\usepackage{etoolbox}
\usepackage{lipsum}

\usepackage{siunitx,booktabs}
\usepackage{pifont}




\setbeamertemplate{enumerate items}[circle]
\usepackage{tikz}

\newcommand\mynum[1]{
  \usebeamercolor{enumerate item}
  \tikzset{beameritem/.style={circle,inner sep=0,minimum size=2ex,text=enumerate item.bg,fill=enumerate item.fg,font=\footnotesize}}%
  \tikz[baseline=(n.base)]\node(n)[beameritem]{#1};
}

\newcommand\mynumm[1]{
  \usebeamercolor{enumerate item}
  \tikzset{beameritem/.style={rectangle,inner sep=0,minimum size=2ex,text=enumerate item.bg,fill=enumerate item.fg,font=\footnotesize}}%
  \tikz[baseline=(n.base)]\node(n)[beameritem]{#1};
}

\def\Put(#1,#2)#3{\leavevmode\makebox(0,0){\put(#1,#2){#3}}}

\newcommand\Wider[2][3em]{%
\makebox[\linewidth][c]{%
  \begin{minipage}{\dimexpr\textwidth+#1\relax}
  \raggedright#2
  \end{minipage}%
  }%
}

\setbeamertemplate{footline}[frame number]

%----------------------------------------------------------------------------------------
%	TITLE PAGE
%----------------------------------------------------------------------------------------

\title{ Microcredit from Delayed Bill Payments } % The short title appears at the bottom of every slide, the full title is only on the title page

\author{Will Violette} 

 % Your institution as it will appear on the bottom of every slide, may be shorthand to save space

\date{\footnotesize The views expressed herein do not necessarily reflect those of the Federal Trade Commission or any of its commissioners.} % Date, can be changed to a custom date
%\today

\begin{document}

\beamertemplatenavigationsymbolsempty

\begin{frame}
\titlepage % Print the title page as the first slide
\end{frame}

%\begin{frame}
%\frametitle{Overview} % Table of contents slide, comment this block out to remove it
%\tableofcontents % Throughout your presentation, if you choose to use \section{} and \subsection{} commands, these will automatically be printed on this slide as an overview of your presentation
%\end{frame}

%----------------------------------------------------------------------------------------
%	PRESENTATION SLIDES
%----------------------------------------------------------------------------------------
\section{Introduction}
%------------------------------------------------

% \begin{frame}
% \frametitle{ Motivation }

% \begin{itemize}
% \item Households experience large month-to-month changes in income
% \begin{itemize}
%   \item Coef. of variation $\sim$0.4 in US and Mexico \\ {\scriptsize (Hannagan and Morduch [2015]; Amuedo-Dorantes and Pozo [2011]) }
% \end{itemize}

% \vspace{2mm}

% \item Short-term credit to smooth consumption is costly 
%   \begin{itemize}
%     \item High interest rates from payday loans, credit cards, informal moneylenders, etc.
%     \item In Manila, only 4\% have credit cards, 19\% have bank accounts
%   \end{itemize}
% \end{itemize}
% \end{frame}


\begin{frame}
\frametitle{ Motivation }


\begin{itemize}
\item Households (HHs) have variable, uncertain incomes
\vspace{2mm}
\item Smoothing consumption is costly 
  \begin{itemize}
    \item High interest rates from payday loans, credit cards, informal moneylenders, etc.
    \item In Manila, only 4\% have credit cards, 19\% have bank accounts
  \end{itemize}

\vspace{2mm}

\item Public utilities (water, electricity, gas, etc.) may provide efficient, second-best credit by letting HHs delay their bill payments
  \vspace{1mm}

  % \begin{enumerate}
  %   \item \textbf{Collateral} from disconnections for nonpayment
  %    \vspace{1mm}
  %   \item \textbf{Low monitoring costs} since workers already service connections
  %    \vspace{1mm}
  %   \item \textbf{Wide risk pool} due to nearly universal coverage 
  %    \vspace{1mm}
  %   \item \textbf{Few consumption distortions} due to inelastic demand for utilities
  % \end{enumerate}
\end{itemize}

\end{frame}

%------------------------------------------------

\begin{frame}
\frametitle{ New policies reduce delinquency }


\begin{itemize}
  \item Growing use of prepaid meters that ensure upfront payments
  \vspace{.1cm}
  \begin{itemize}
    \item benefits $\rightarrow$ may lower prices and increase investments in quality
    \item costs $\rightarrow$ no more credit from delayed bill payments
  \end{itemize}
\begin{figure}
\centering
\caption{Prepaid water meter in South Africa}
\includegraphics[scale=.2]{prepaid-water-meter.jpg}
\end{figure}
\end{itemize}

% \begin{itemize}
%   \item In the US, states often encourage utility credit by limiting disconnections for delinquency
%     \vspace{1mm}
%     \begin{itemize}
%       \item Often cite health reasons {\footnotesize (Focus for future research, LIHEAP [2019]))}
%     \end{itemize}
% \end{itemize}



\end{frame}

%------------------------------------------------

\begin{frame}
\frametitle{This Paper}

\begin{itemize}
\item \textbf{Question} \hspace{.5mm} how much do HHs value delaying their bills?
\vspace{.1cm}
\begin{itemize}
  \item How credit-constrained are HHs?
  \item What are the welfare effects of other payment policies \\ (ie. prepaid meters)?
\end{itemize}
\vspace{2mm}
\item \textbf{Context} \hspace{.5mm} a regulated piped water utility in Manila
\vspace{2mm}
\item \textbf{Data} \hspace{.5mm} monthly billing records from 2010-15 for 1.5 mil. connections
\vspace{2mm}
\item \textbf{Approach} \hspace{.5mm} estimate a consumption/savings model where HHs choose when to pay their water bills
% \vspace{1mm}
% \begin{itemize}
%   \item Estimate credit constraints from billing delinquency
%   \item Simulate counterfactuals
% \end{itemize}

\end{itemize}

\end{frame}

%------------------------------------------------
%------------------------------------------------

\begin{frame}
\frametitle{Preview of Results}

\begin{itemize}
  \item Estimated monthly interest rate is 2.2\%  (30\% annually)
  \vspace{1mm}
    \begin{itemize}
      % \item In Manila, moneylenders offer 20\% monthly { \footnotesize (Karlan and Zinman [2009])}
      %   \vspace{1mm}
      \item Globally, microfinance offers 13 to 25\% annually { \footnotesize (Cull et al. [2009])}
    \end{itemize}

\vspace{2mm}

  \item Willingness-to-pay for delaying bills is $\sim$70 PhP (or \$1.5) per month 
  \vspace{1mm}
  \begin{itemize}
    \item Equal to 9\% of an avg water bill
    % and 0.2\% of HH income
  \end{itemize}

\vspace{2mm}

  \item Prepaid metering (adjusting prices to cover costs) reduces welfare

\end{itemize}

\end{frame}

% monthly = ((1+annual)^(1/12)) - 1
% annual = ((monthly + 1)^(12)) - 1

%------------------------------------------------
%------------------------------------------------


\begin{frame}
\frametitle{Contributions to the Literature}

\begin{enumerate}
\item Bring consumption smoothing to public utility regulation \\
{\footnotesize (McRae [2015]; Szab\'o [2015]; Jack and Smith [2015,2016]; Szab\'o and Ujhelyi [2015])}
\vspace{2mm}


\item Estimate HH consumption/savings model with utility billing data \\
{\footnotesize (Deaton [1991]; Gourinchas and Parker [2002]; Laibson et al. [2007])}
\vspace{2mm}

\item Measure credit constraints from billing delinquency \\
{ \footnotesize (\textit{RCTs}: Karlan and Zinman [2009]; Gin\'e and Karlan [2014],  \textit{Village surveys}: Townsend [1994]; Townsend and Kinnan [2012]; Ligon [1994], \textit{Natural Experiments}: Banerjee and Duflo [2012]) }

% Numerous case studies and empirical analyses in a variety of countries have revealed that informal credit markets often display patterns and features not commonly found in institutional lending: (i) loans are often advanced on the basis of oral agreements rather than written contracts, with little or no collateral, making default a seemingly attractive option (ii) the credit market is usually highly segmented, marked by long-term exclusive relationships and repeat lending (iii) interest rates are much higher on average than bank interest rates, and also show significant dispersion, presenting apparent arbitrage opportunities (iv) there is frequent interlinkage with other markets, such as land, labor or crop (v) significant credit rationing, whereby borrowers are unable to borrow all they want, or some loan applicants are unable to borrow at all. There are a number of different theoretical approaches that attempt to explain some or all of these features. Though differing in specific mechanisms proposed, they share a common general theme: that the world of informal credit is one of missing markets, asymmetric information, and incentive problems. There are a number of broad strands in the literature, focusing respectively on adverse selection (hidden information), moral hazard (hidden action), and contract enforcement problems. This article provides a sample of the latter two approaches, and argues that they are fundamentally similar in terms of their underlying logic and policy implications.

%  {\footnotesize (Morduch [1999]; Dupas and Robinson [2013]; Karlan and Zinman [2009]; Gin\'e and Karlan [2014]; Townsend [1994])}


\end{enumerate}

\end{frame}


%------------------------------------------------

\begin{frame}
\frametitle{Paying water bills in Manila}

\begin{enumerate}


\item The avg HH is 85 days behind on their payments
  \begin{itemize}
    \item Avg HH's unpaid water bills = 5\% monthly HH income
 %   \item Two main reasons: Inconvenient to pay every month and Consumption smoothing
  \end{itemize}
\vspace{3mm}

\item No interest is charged on delinquent bills 
% and many options for paying \\
% { \footnotesize (gas stations, convenient stores, phone, online, or via ATM kiosks) }

\vspace{3mm}

% \item After 60 days of delinquency, regulations permit disconnection
\item The utility visits delinquent HHs and makes a take-it or leave-it offer: \\ pay now or become disconnected % sudden take-it-or-leave-it offer
\vspace{.5mm}
  \begin{itemize}
    \item Visits are rare (4\% of HH-months given $>$60 days delinquent)
    % \item HHs pay immediately to avoid disconnection
    % \item Others disconnect until they can afford to reconnect
  \end{itemize}

\vspace{3mm}

\item To reconnect, HHs pay a small one-time fee and all unpaid bills
\vspace{.5mm}
  \begin{itemize}
    \item When HHs change residences, they rarely pay their outstanding bills 
  \end{itemize}

\end{enumerate}


\end{frame}


%------------------------------------------------

%------------------------------------------------

\begin{frame}
\frametitle{Data and Sample}

\begin{itemize}
  \item Data
    \begin{itemize}
      \item Monthly billing records per connection 2010-15 \\
      (usage, payments, and delinquency visits)
      \vspace{2mm}
      \item Merge to survey data on $\sim$50,000 connections \\
      (number of HHs sharing a connection and demographics for the owner)
    \end{itemize}
    \vspace{2mm}
  \item Sample
    \begin{itemize}
      \item Model single HH decisions
        \begin{itemize}
          \item Keep residential connections that serve a single HH (67\%) 
          \vspace{1mm}
        \end{itemize}
      \item Use delinquency visits for identification
        \begin{itemize}
          \item Keep HHs with visits (31\%) 
        \end{itemize}
         \vspace{1mm}
      \item Drop HHs that move
        \begin{itemize}
          \item Drop if disconnected for the last 6 months of the sample (10\%) 
        \end{itemize}
    \end{itemize}
\end{itemize}

\end{frame}

%------------------------------------------------

\begin{frame}
\frametitle{Descriptives}

% \Wider{
% \begin{table}[H]
% \centering
% %\caption{Descriptives}\label{table:descriptives_stayers}
% \vspace{-2mm}
% % \resizebox{1.05\linewidth}{!}{
% \begin{tabular}{l*{1}{cccccc}}
% \toprule
%  & Mean & SD & Min & 25th & 75th & Max  \\
% \midrule
%  Usage (m3)  & 26.2  & 17.5  & 0.0  & 15.0  & 33.0  & 200.0  \\ 
 Bill  & 761  & 1,124  & -4,640  & 287  & 920  & 78,409  \\ 
 Unpaid Balance  & 2,416  & 5,070  & -4,995  & 261  & 2,346  & 79,904  \\ 
 Share of Months with Payment  & 0.60  & 0.49  & 0.00  & 0.00  & 1.00  & 1.00  \\ 
 Payment Size  & 1,214  & 1,498  & 0  & 426  & 1,482  & 61,298  \\ 
 Days Delinquent  & 84.9  & 155.4  & 0.0  & 0.0  & 91.0  & 720.0  \\ 
 Delinquency Visits per HH  & 1.32  & 0.61  & 1.00  & 1.00  & 2.00  & 6.00  \\ 
 Share of Months Disconnected  & 0.03  & 0.17  & 0.00  & 0.00  & 0.00  & 1.00  \\ 

% \bottomrule
% \multicolumn{7}{c}{Total Households: \input{tables/total_hhs_stayers}  Obs. per Household: \input{tables/obs_per_hh_stayers} Total Obs.: \input{tables/total_obs_stayers}}
% \end{tabular}
% %}
% \end{table}
% }
\begin{table}[H]
\centering
\vspace{-2mm}
\begin{tabular}{l*{1}{cc}}
% \toprule
 & Mean & SD  \\
\midrule
 Usage (m3)  & 26.2  & 17.5  \\ 
 Bill  & 761  & 1,124  \\ 
 Unpaid Balance  & 2,416  & 5,070  \\ 
 Share of Months with Payment  & 0.60  & 0.49  \\ 
 Days Delinquent  & 84.9  & 155.4  \\ 
 Delinquency Visits per HH  & 1.32  & 0.61  \\ 
 Share of Months Disconnected  & 0.03  & 0.17  \\ 

% \bottomrule
\end{tabular}
% }
\end{table}

Total HHs:  \input{tables/total_hhs_stayers}  \hspace{3mm} Obs per HH: \input{tables/obs_per_hh_stayers}  \hspace{3mm} Total Obs: \input{tables/total_obs_stayers} \\[.8em]
45 Philippine Peso (PhP) = 1 US Dollar \\
Avg monthly HH Income \input{tables/y_avg}PhP \\
% \begin{itemize}
% \item Why do households pay late? (convenience, consumption smoothing)
% \end{itemize}

\end{frame}


%------------------------------------------------
%------------------------------------------------

\begin{frame}
\frametitle{Avg share connected around 1st delinquency visit}
\Wider[4em]{
\begin{figure}
\centering
%\includegraphics[scale=.47]{tables/line1_pay.pdf}
\includegraphics[scale=.7]{tables/line_conn.pdf}
\end{figure}
}
\end{frame}



%------------------------------------------------
%------------------------------------------------

% \begin{frame}
% \frametitle{By days overdue at the time of first visit}
% \Wider[4em]{
% \begin{figure}
% \centering
% \includegraphics[scale=.47]{tables/line_pay.pdf}
% \includegraphics[scale=.47]{tables/line_disconnection.pdf}
% \end{figure}
% }
% \end{frame}

% \begin{frame}
% \frametitle{Avg consumption around 1st delinquency visit}
% \Wider[4em]{
% \begin{figure}
% \centering
% \includegraphics[scale=.7]{tables/line_c.pdf}
% %\includegraphics[scale=.47]{tables/line_disconnection.pdf}
% \end{figure}
% }
% \end{frame}

%------------------------------------------------
%------------------------------------------------

% \begin{frame}
% \frametitle{Model}

% \begin{itemize}
%   \item HHs have Cobb-Douglas utility over water, $w_{\tau}$, and all other goods $x_{\tau}$
%   \item HHs maximize expected utility over an infinite time horizon discounted at rate $\delta$
% \begin{align*}\label{eq:u}
% max_{w_t,x_t} \, \, \, E_t \Big[ \, \, \sum_{\tau = t}^{\infty} (1+\delta)^{t-\tau} \, w_{\tau}^{\alpha}x_{\tau}^{(1-\alpha)}  \, \, \Big]
% \end{align*}
% \item subject to a budget constraint in each period
% \begin{align*}
% x_t \, + \, p(w_t) w_t \, &= \, y_t \, + S_t
% \end{align*}
% \item where $p(w_t)$ is the relative (non-linear) water price, \\
%   $y_t$ is income, and $S_t$ is net savings
% \end{itemize}
% \end{frame}

\begin{frame}
\frametitle{Model of HH consumption and savings}

% u(w_{\tau},x_{\tau})
\begin{align*}\label{eq:u}
&max \, \, \, E_t \Big[ \, \, \sum_{\tau = t}^{\infty} (1+\delta)^{t-\tau} \,\, u(w_{\tau},x_{\tau}) \, \, \Big] \\
\forall t &\,\,\,\,\,x_t \, + \, p(w_t) w_t \, = \, y_t \, + A_t - \frac{ A_{t+1} }{1+r_a} + S_t
\end{align*}

\begin{itemize}
\item Utility, $u(w_{\tau},x_{\tau}) =\alpha log( w_{\tau}) + (1-\alpha) log(x_{\tau})$ is over water, $w_t$, and all other goods, $x_t$, with discount rate, $\delta$
\vspace{.5mm}
\item Budget constraint has water price, $p(w_t)$, and income, $y_t$, which takes values $(1+\theta)\bar{y}$ and $(1-\theta)\bar{y}$ with 0.5 probability
\vspace{.5mm}
\item HHs borrow and save with asset $A_{t+1}$ where $A_{t+1}\geq -\bar{A}$ and interest rate, $r_a$, is  equal to $r_h$ if borrowing ($A_{t+1}\leq 0$) and $r_l$ else
\vspace{.5mm}
\item $S_t$ allows for borrowing from water bills (cont.)
\end{itemize}

% \begin{itemize}
% \item $w_t$ : water
% \vspace{.5mm}
% \item $x_t$ : all other goods
% \vspace{.5mm}
% \item $u(w_{\tau},x_{\tau}) = x_{\tau}^{\alpha}x_{\tau}^{(1-\alpha)}$
% \vspace{.5mm}
% \item $\delta$ : discount rate
% \vspace{.5mm}
% \item $p(w_t)$ : price 
% \vspace{.5mm}
% \item $y_t$ : income, $(1+\theta)\bar{y}$ or $(1-\theta)\bar{y}$ with 0.5 probability
% \vspace{.5mm}
% \item $A_t, A_{t+1}$ : assets where $A_{t+1}\geq \bar{A}$
% \vspace{.5mm}
% \item $r_a$ : interest rate, $r_h$ if borrowing ($A_{t+1}\leq 0$) and $r_l$ else
% \vspace{.5mm}
% \item $S_t$ : water borrowing (cont.)
% \end{itemize}
\end{frame}



\begin{frame}
\frametitle{Borrowing from water bills, $S_t$}

% \begin{align*}
% S_t &= B_t + I_t \dfrac{B_{t+1}}{1+r^{b}_{t}}  \\
% I_t &= D_{t+1} + (1-c_t) (1-D_t) (1-D_{t+1}) \\
% &B_t -  p(w_t) w_t (1-D_{t+1})(1+r_b) \leq B_{t+1} \leq 0 
% \end{align*}

% ($D_t$=0) ($c_t$=0)

\begin{itemize}
\item Each period, HH faces probability $\pi$ of receiving a delinquency visit
\item If no visit occurs, HHs can borrow from their current bill
\begin{align*}
S_t &= B_{t-1} -  B_{t} \\
B_{t-1} &-  p(w_t) w_t \leq B_{t} \leq 0 
\end{align*}

\begin{itemize}
  \item $B_{t-1}$ : last month's unpaid bill { \footnotesize ($\leq0$) }
  \item $B_{t}$ : this month's unpaid bill { \footnotesize ($=0$ if $A_t>0$ to prevent arbitrage)}
  % \item $r_{b}$ : bill interest rate (equal to $r_h$ if $A_t>0$ to prevent arbitrage)
  % \item HHs can borrow up to their unpaid bills plus their current usage
\end{itemize}
\vspace{2mm}


\item If a visit occurs, HHs can choose to disconnect ($D_{t}=1$), avoid paying their bills ($S_t  = 0$), and pay a fixed cost ($f$) per month for other water until they reconnect

\item Otherwise, HHs pay off any unpaid bills ($S_t=B_{t-1}$) and this month's bill ($B_{t}=0$) to stay connected

\end{itemize}

\end{frame}


%------------------------------------------------
%------------------------------------------------

\begin{frame}
\frametitle{Solving the model with a value function approach}

\begin{align*}
V(X_t,z_t) \,=\, &max_{x_t,w_t}  \,\,\, u(x_t,w_t) \,+\,\, (1+\delta)^{-1} \, E \Big[\, V(X_{t+1}|z_{t})\,\Big| z_{t+1}, T_{t,t+1} \Big]
\\
s.t.& \\
x_t & \, + \, p(w_t) w_t \, = \, y_t  \, + \, S_t \\
B_{t-1}-p&(w_t) w_t (1-D_{t}) \leq B_{t} \leq 0  \\
\\
X_t &= [x_{t},w_{t},A_{t},B_{t},D_{t}] \hspace{2mm}\text{ chosen by HH }  \\
 z_t &= [y_t,visit_t]  \\
T_{t,t+1} &= [ 0.5\pi \,\,\,\, 0.5(1-\pi)\,\,\,\,  0.5\pi\,\,\,\,  0.5(1-\pi) ] \times [ 1 \,\,\,\, 1\,\,\,\,  1 \,\,\,\, 1 ]^{\text{T}}
% \text{simple transition matrix } \\ 
\end{align*}


% \vspace{2mm}
% \begin{itemize}
% \item Choices observed in the data : usage ($w_t$), bills ($B_t$), and disc. ($D_t$)
% \end{itemize}

\end{frame}

%------------------------------------------------
%------------------------------------------------


\begin{frame}

\frametitle{Calibrated Parameters}
\Wider[4em]{

\begin{table}[H]
\centering
% \caption{Calibrated and Assumed Parameters}\label{table:calibratedparam}
\begin{tabular}{l*{1}{ll}}
%\toprule
%Parameter  &   &  Value & Source \\
%\midrule
Calibrated & & Source \\[.5em]
\toprule
Discount rate  & $\delta = 0.015$ & {\footnotesize Structural macro literature} \\
 Savings interest rate & $r_l = 0.003$ & {\footnotesize World Bank} \\
 Visit risk & $\pi = 0.04$ & {\footnotesize Billing data }  \\
Price  & $p = 20.2 + 0.2w$ & {\footnotesize Billing data } \\
Mean inc. (PhP) & $\bar{y} = 31,910$ & {\footnotesize HH inc. survey} \\
Borrowing limit & $\bar{A} = -32,250$ & {\footnotesize HH inc. survey (95 pctile. of loans)} \\
Unpaid bills limit & $\bar{B} = -10,109$ & {\footnotesize Billing data (95 pctile. of unpaid bills)} \\
%\bottomrule 
\\
% \end{tabular}
% \begin{tabular}{l*{1}{l}}

% \textbf{To be estimated}  & & \textbf{Identifying moments}\\[.5em]
% %\multicolumn{3}{l}{\scriptsize All measures are monthly. } \\[-.5em]
% Water preference & $\alpha $ & {\footnotesize   Avg usage   } \\[.1em]
% Income shock size  & $\theta $ & {\footnotesize  Avg unpaid bills } \\[.1em]
% Cost of other water  &  $f$ & {\footnotesize  \% Disc. 1-2 months post visit } \\[.1em]
% Borrowing rate & $r_h$ & \% {\footnotesize  Disc. 1-2 months post visit } \\
%  & &  {\footnotesize \hspace{4mm} given $>$90 days overdue} \\
\end{tabular}
\end{table}
}

{\footnotesize All terms are monthly}

\end{frame}







\begin{frame}

\frametitle{Estimation with simulated method of moments}

\begin{table}[H]
\centering
% \caption{Calibrated and Assumed Parameters}\label{table:calibratedparam}
\begin{tabular}{l*{1}{ll}}
Estimated Parameters  & & Moments \\[.5em]
\toprule
%\multicolumn{3}{l}{\scriptsize All measures are monthly. } \\[-.5em]
Water preference & $\alpha $ & {\footnotesize   Avg usage   } \\[.1em]
Income shock size  & $\theta $ & {\footnotesize  Avg unpaid bills } \\[.1em]
Fixed cost of other water  &  $f$ & {\footnotesize  \% Disc. 1-2 months post visit } \\[.1em]
Borrowing rate from standard assets & $r_h$ & \% {\footnotesize  Disc. 1-2 months post visit } \\
 & &  {\footnotesize \hspace{4mm} given $>$90 days overdue} \\
\end{tabular}
\end{table}

\begin{itemize}
\item Solve for the optimum of a grid of 28 asset and 28 billing values
\item Compute simulated moments (avg usage, unpaid bills, etc.) with a random sequence of 10,000 states
\item Choose parameters to minimize the sum of squared distances between the data and the simulated moments
\end{itemize}

\end{frame}


\begin{frame}
\frametitle{Estimates}

\begin{table}[h!]
\centering
%\caption{Estimates}\label{table:estimates}
\vspace{-2mm}
%\resizebox{\columnwidth}{!}{%
\begin{tabular}{l*{1}{cc}}
%\toprule
Parameters  &   & Estimates \\
\midrule
Water Preference & $\alpha$ & \input{tables/est_alpha} \\
 &  & (\input{tables/est_sd_alpha}\unskip) \\[.4em]
Income shock size & $\theta$ & \input{tables/est_theta} \\
 &  & (\input{tables/est_sd_theta}\unskip) \\[.4em]
Fixed cost of other water (PhP) & $f$ &  \input{tables/est_fc} \\
 &  &  (\input{tables/est_sd_fc}\unskip) \\[.4em]
Borrowing rate from standard assets & $r_h$ & \input{tables/est_irate} \\
 &  & (\input{tables/est_sd_irate}\unskip) \\[.8em]
Households & & \input{tables/est_hhs} \\
Household-Months & & 2,118,861 \\
\bottomrule
\multicolumn{3}{l}{\scriptsize Standard errors in parentheses are bootstrapped at the household-level.} % with \input{tables/breps}repetitions 
\end{tabular}
%}
\end{table}

\end{frame}


\begin{frame}
\frametitle{Counterfactuals}

\Wider[4em] { %%%% widen the whole thing!

\begin{table}[H]
\centering
%\caption{Counterfactual Policies}\label{table:counter}
\resizebox{\columnwidth}{!}{%
\begin{tabular}{lcc<{\onslide<2->}c<{\onslide<3->}c<{\onslide}}
\toprule
 & (1) & (2) & (3) & (4) \\
 & Current & No Water   & No Water Borrowing    & Prepaid Metering \\
 &         & Borrowing  & and Covering Costs      & and Covering Costs   \\
\midrule   
Compensating Variation (PhP) &  & \input{tables/U_nl} & \input{tables/U_ppe}  & \input{tables/U_pp} \\
Mean Usage (m3) & \input{tables/c_h2} & \input{tables/c_nl2}  & \input{tables/c_ppe2} & \input{tables/c_pp2} \\[.2em]
 &         &           &         &  \\
% Water Borrowing  & \checkmark & X & X & X \\[.2em]
%Adjustments to stay revenue neutral  & & \\
\onslide<2-> Price Intercept (PhP/m3) & \input{tables/p_int2} &   &  \input{tables/p_int_ppe2} & \input{tables/p_int_pp2} \\[.2em]
\onslide<2->  Credit supply costs (PhP) & 31.3 &   & 0 & 0 \\[.2em]
% \onslide<2-> Disconnection Rebate (PhP) & \input{tables/delinquency_cost} &   & 0 & 0 \\[.2em]
% \onslide<2-> Delinquency Visit Cost (PhP) & \input{tables/visit_cost} &  & 0 & 0 \\[.2em]
% \onslide<2-> Opp. Cost of Lending (PhP) & \input{tables/coste} &  & 0 & 0 \\[.2em]

\onslide<2-> Marginal cost (PhP/m3) & 5 &   & 5 & 5 \\[.2em]
\onslide<3-> Additional metering cost (PhP) &  0 &  & \onslide<3-> 0 & 51 \\[.2em]
%Price Intercept (PhP)  &20.2& &26.6\\

%\input{tables/counter_fixed_nomid}
\bottomrule
\multicolumn{5}{l}{  All values are at the household-month level. }
%The price intercept increases under prepaid metering to cover meter replacement } % \\[-.5em]
%\multicolumn{5}{l}{ \scriptsize  costs.  The disconnection rebate measures the average outstanding balance left unpaid by permanently disconnected households } \\[-.5em]
%\multicolumn{5}{l}{ \scriptsize  in terms of household-months.  Price intercepts and disconnection rebates are unchanged for the no water } \\[-.5em]
%\multicolumn{5}{l}{ \scriptsize  borrowing counterfactual.}
\end{tabular}
}
\end{table}

} %%%% widen the whole thing!


\begin{itemize}
 \onslide<2>  \item Credit supply costs include (1) cost of delinquency visits, (2) lost revenue from HHs that move, and (3) opportunity cost of credit
\end{itemize}

\end{frame}



\begin{frame}

\frametitle{Next Steps}

\begin{itemize}
  \item Estimate heterogeneity by income
  \item Model HHs decision to move out of Manila (and leave outstanding bills)
  \item Optimal delinquency visit policy for Manila
\end{itemize}
\vspace{2mm}
Thank you!

\end{frame}



\begin{frame}
\frametitle{Other outcomes relative to 1st visit}
\Wider[4em]{
\begin{figure}
\centering
\includegraphics[scale=.47]{tables/line_pay.pdf}
\includegraphics[scale=.47]{tables/line_c.pdf}
\end{figure}
\begin{itemize}
  \item Avg payments only include positive payments
\end{itemize}
}
\end{frame}



\end{document} 
